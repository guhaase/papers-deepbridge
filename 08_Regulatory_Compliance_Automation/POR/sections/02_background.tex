\section{Fundamentacao e Panorama Regulatorio}

\subsection{EEOC: Equal Employment Opportunity Commission}

\subsubsection{Title VII - Civil Rights Act (1964)}
Proibe discriminacao em emprego baseada em:
\begin{itemize}
    \item Raca
    \item Cor
    \item Religiao
    \item Sexo (incluindo orientacao sexual, identidade de genero)
    \item Origem nacional
\end{itemize}

\subsubsection{Four-Fifths Rule (80\% Rule)}
Uniform Guidelines on Employee Selection Procedures (1978) estabelecem teste quantitativo para adverse impact:

\begin{equation}
\text{Impact Ratio} = \frac{\text{Taxa Selecao Grupo Protegido}}{\text{Taxa Selecao Grupo Referencia}}
\end{equation}

Se Impact Ratio $<$ 0.80, ha \textbf{prima facie evidence} de discriminacao.

\textbf{Exemplo}: Se 50\% de candidatos brancos sao contratados, pelo menos 40\% (50\% $\times$ 0.80) de candidatos negros devem ser contratados.

\subsubsection{Statistical Significance Testing}
Alem de four-fifths rule, EEOC recomenda testes estatisticos:
\begin{itemize}
    \item \textbf{Fisher's Exact Test}: Para amostras pequenas ($n < 30$)
    \item \textbf{Chi-squared Test}: Para amostras grandes
    \item \textbf{Z-test para proporcoes}: Comparacao de taxas de selecao
\end{itemize}

Threshold: $p < 0.05$ indica disparidade estatisticamente significativa.

\subsection{ECOA: Equal Credit Opportunity Act}

\subsubsection{Regulation B (12 CFR Part 1002)}
Proibe discriminacao em transacoes de credito baseada em:
\begin{itemize}
    \item Raca, cor, religiao, origem nacional
    \item Sexo, estado civil
    \item Idade (exceto para determinar capacidade de contratar)
    \item Recepcao de assistencia publica
    \item Exercicio de direitos sob Consumer Credit Protection Act
\end{itemize}

\subsubsection{Prohibited Basis}
Regulacao B especifica que credores \textbf{nao podem}:
\begin{enumerate}
    \item Usar caracteristicas protegidas diretamente em decisoes
    \item Usar proxies correlacionadas (ex: codigo postal como proxy para raca)
    \item Aplicar standards diferentes por grupo (disparate treatment)
\end{enumerate}

\subsubsection{Adverse Action Notices}
ECOA requer que credores forneçam:
\begin{itemize}
    \item Notificacao de decisao adversa dentro de 30 dias
    \item Razoes especificas para rejeicao (top-k features)
    \item Informacao sobre direito de solicitar detalhes
\end{itemize}

Para modelos de ML: features mais importantes devem ser explicaveis e nao-discriminatorias.

\subsection{Fair Housing Act}

Proibe discriminacao em:
\begin{itemize}
    \item Venda ou locacao de habitacao
    \item Financiamento habitacional (mortgages)
    \item Publicidade de habitacao
\end{itemize}

Grupos protegidos: raca, cor, religiao, sexo, deficiencia, status familiar, origem nacional.

\subsection{Trabalhos Relacionados}

\subsubsection{Fairness Auditing Tools}
\begin{itemize}
    \item \textbf{AIF360} (IBM): Metricas de fairness mas sem mapeamento direto para requisitos legais
    \item \textbf{Fairlearn} (Microsoft): Foco em metricas tecnicas, nao compliance regulatoria
    \item \textbf{Aequitas} (U. Chicago): Toolkit para bias auditing, mas sem coverage de EEOC/ECOA especificos
\end{itemize}

\textbf{Limitacao}: Ferramentas existentes focam em metricas ML (demographic parity, equalized odds) mas nao traduzem diretamente para requisitos legais.

\subsubsection{Legal-Tech para Compliance}
\begin{itemize}
    \item \textbf{Contratos inteligentes}: Blockchain para compliance em financas (nao aplicavel a ML)
    \item \textbf{GRC platforms} (Governance, Risk, Compliance): ServiceNow, SAP---focados em processos, nao algoritmos
\end{itemize}

\textbf{Gap}: Falta integracao entre ferramentas de fairness ML e requisitos regulatorios especificos.

\subsubsection{Trabalhos Academicos}

\textbf{Huq (2019)}: Analisa tensao entre EEOC regulations e algoritmos opacos. Argumenta necessidade de transparency para compliance.

\textbf{Reisman et al. (2018)}: ``Algorithmic Impact Assessments''---propoe framework qualitativo para auditoria, mas sem automacao.

\textbf{Selbst et al. (2019)}: ``Fairness and Abstraction in Sociotechnical Systems''---critica metricas ML por ignorarem contexto legal.

\textbf{Nossa contribuicao}: Ponte entre requisitos legais e testes automatizados executaveis.

\subsection{Desafios de Traducao Legal $\rightarrow$ Tecnico}

\begin{table}[h]
\centering
\caption{Desafios de codificacao de requisitos legais}
\begin{tabular}{p{3cm}p{4cm}}
\toprule
\textbf{Requisito Legal} & \textbf{Desafio Tecnico} \\
\midrule
``Discriminacao proibida'' & Definir threshold quantitativo \\
``Razoes especificas'' (ECOA) & Explicabilidade de ML \\
``Business necessity'' & Balancear accuracy vs. fairness \\
``Proxy variables'' & Detectar correlacoes indiretas \\
\bottomrule
\end{tabular}
\end{table}

\subsection{Precedentes Legais Relevantes}

\subsubsection{Griggs v. Duke Power Co. (1971)}
Estabeleceu \textbf{disparate impact doctrine}: Praticas neutras que causam impacto desproporcional sao discriminatorias, mesmo sem intencao.

Aplicacao: Modelos ML podem violar Title VII mesmo sem usar caracteristicas protegidas diretamente.

\subsubsection{Ricci v. DeStefano (2009)}
Tensao entre evitar disparate impact e evitar disparate treatment.

Aplicacao: Organizacoes nao podem simplesmente descartar resultados de modelos que mostram disparidades---devem demonstrar business necessity.

\subsection{Estado da Pratica Atual}

Organizacoes tipicamente verificam compliance via:
\begin{enumerate}
    \item \textbf{Revisao legal manual}: Advogados analisam documentacao de modelos (20-80h por modelo)
    \item \textbf{Auditorias estatisticas ad-hoc}: Estatisticos calculam impact ratios manualmente
    \item \textbf{Checklists qualitativos}: Sem validacao quantitativa
\end{enumerate}

Problemas: Alto custo, baixa reproducibilidade, cobertura incompleta, timing subotimo.

Nossa abordagem: Automacao end-to-end com cobertura completa de requisitos EEOC/ECOA.

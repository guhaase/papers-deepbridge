\section{Discussao}

\subsection{Principais Descobertas}

\subsubsection{Deteccao Superior a Revisao Manual}
Framework automatizado detectou \textbf{67\% mais violacoes} que revisao manual (5 vs. 3 em CS1, 4 vs. 2 em CS2). Violacoes nao-detectadas manualmente incluiam:

\begin{itemize}
    \item \textbf{Testes estatisticos}: Chi-squared e Fisher's exact frequentemente omitidos em revisoes manuais por custo computacional
    \item \textbf{Proxy variables}: Correlacoes sutis ($r \approx 0.7$) nao-obvias sem analise quantitativa
    \item \textbf{Violacoes procedurais}: ECOA adverse action requirements frequentemente negligenciados
\end{itemize}

\subsubsection{Trade-off Compliance-Accuracy Aceitavel}
Mitigacoes recomendadas resultaram em perda minima de acuracia:
\begin{itemize}
    \item Hiring AI: -2.6\% F1-Score
    \item Lending AI: -1.2\% AUC
\end{itemize}

Resultado alinhado com literatura (Hardt et al., 2016; Menon \& Williamson, 2018) sugerindo que post-processing e in-processing fairness interventions tipicamente causam perda $<$ 5\% em accuracy.

\subsubsection{Valor de Continuous Monitoring}
Deteccao de compliance drift em producao (CS1: 94\% $\rightarrow$ 81\%) demonstra necessidade de monitoring continuo vs. auditoria pontual pre-deployment.

\subsection{Implicacoes Praticas}

\subsubsection{Para Organizacoes}

\textbf{Reducao de Risco Legal}:
\begin{itemize}
    \item Deteccao precoce de violacoes evita deployment de sistemas nao-compliant
    \item Evidencia quantitativa de due diligence em caso de litigo
    \item Relatorios auditaveis para reguladores (EEOC, CFPB)
\end{itemize}

\textbf{Eficiencia Operacional}:
\begin{itemize}
    \item 87\% reducao em custo de auditoria (\$11,250 $\rightarrow$ \$1,350 em media)
    \item Liberacao de recursos especializados para tarefas estrategicas
    \item Integracao em CI/CD elimina gargalos de compliance
\end{itemize}

\subsubsection{Para Desenvolvedores de ML}

Framework permite shift-left de compliance:
\begin{itemize}
    \item Testes executados durante desenvolvimento, nao apenas pre-deployment
    \item Feedback rapido (15min vs. semanas esperando revisao legal)
    \item Recomendacoes tecnicas especificas (vs. orientacoes legais abstratas)
\end{itemize}

\subsubsection{Para Reguladores}

\textbf{Padronizacao de Metricas}:
Framework codifica interpretacoes especificas de regulacoes (ex: four-fifths rule, $p < 0.05$ thresholds), criando consistency entre auditorias.

\textbf{Transparencia}:
Relatorios automatizados fornecem evidencia quantitativa verificavel, reduzindo dependencia de alegacoes qualitativas.

\textbf{Escalabilidade}:
Permite auditoria de milhares de sistemas sem aumento proporcional em recursos regulatorios.

\subsection{Limitacoes}

\subsubsection{Limitacao 1: Interpretacao Legal}
Framework codifica \textbf{uma interpretacao} de requisitos regulatorios. Regulacoes sao frequentemente ambiguas e sujeitas a interpretacao judicial.

\textbf{Exemplo}: Four-fifths rule e ``rule of thumb'', nao threshold legal absoluto. Contextos especificos podem requerer standards diferentes.

\textbf{Mitigacao}: Fornecemos parametros configuravel para thresholds. Recomendamos consulta legal para casos limite.

\subsubsection{Limitacao 2: Cobertura Regulatoria}
Framework foca em EEOC Title VII e ECOA Regulation B. Nao cobre:
\begin{itemize}
    \item Fair Housing Act (parcialmente sobreposto com ECOA)
    \item ADA (Americans with Disabilities Act)
    \item GDPR (regulacoes europeias)
    \item Regulacoes estaduais especificas (ex: California CCPA)
\end{itemize}

\textbf{Trabalho futuro}: Extensao modular para regulacoes adicionais.

\subsubsection{Limitacao 3: Business Necessity}
EEOC permite adverse impact se houver ``business necessity'' comprovada. Esta determinacao e qualitativa e context-dependent.

Framework \textbf{detecta} violacoes mas nao pode automaticamente determinar se business necessity justifica disparidade.

\textbf{Recomendacao}: Usar framework para quantificar trade-offs, mas decisao final requer julgamento humano.

\subsubsection{Limitacao 4: Proxies Complexos}
Detector de proxy variables usa correlacao linear. Proxies nao-lineares ou interacoes complexas podem nao ser detectados.

\textbf{Exemplo}: Combinacao de ``zip\_code'' + ``education\_level'' pode ser proxy para raca mesmo se correlacoes individuais sao baixas.

\textbf{Mitigacao futura}: Incorporar tecnicas de causal inference para detectar proxies indiretos.

\subsection{Consideracoes Eticas}

\subsubsection{Compliance vs. Justice}
Compliance legal e \textbf{necessario mas nao suficiente} para justica. Sistemas podem passar em testes de compliance mas ainda causar danos:

\begin{itemize}
    \item Four-fifths rule permite disparidade de ate 20\%
    \item Nao cobre fairness intra-grupo (ex: interseccionalidade)
    \item Foco em grupos protegidos legalmente pode negligenciar outros grupos vulneraveis
\end{itemize}

\textbf{Posicionamento}: Framework e ferramenta de \textit{compliance}, nao substituto para analise etica abrangente.

\subsubsection{Automacao de Julgamento Legal}
Codificar requisitos legais em algoritmos pode:
\begin{itemize}
    \item[\cmark] Aumentar consistency e reproducibilidade
    \item[\xmark] Reduzir flexibilidade e context-sensitivity
    \item[\xmark] Criar illusao de objectividade quando decisoes sao fundamentalmente normativas
\end{itemize}

\textbf{Recomendacao}: Usar framework como \textit{decision support}, nao decision automation.

\subsubsection{Acesso e Equidade}
Framework open-source reduz barreiras para compliance, mas:
\begin{itemize}
    \item Pequenas empresas podem carecer de expertise para interpretar resultados
    \item Risco de ``checkbox compliance'' sem understanding substantivo
\end{itemize}

\textbf{Mitigacao}: Documentacao educativa, exemplos detalhados, parcerias com organizacoes de suporte legal.

\subsection{Trabalhos Futuros}

\subsubsection{Extensoes Tecnicas}
\begin{enumerate}
    \item \textbf{Regulacoes adicionais}: GDPR, ADA, Fair Housing Act, BIPA
    \item \textbf{Causal inference}: Detectar proxies via causal graphs vs. correlacoes
    \item \textbf{Individual fairness}: Complementar group fairness com similarity-based fairness
    \item \textbf{Interseccionalidade}: Testes para combinacoes de caracteristicas protegidas
    \item \textbf{Temporal analysis}: Detectar drift de compliance ao longo do tempo
\end{enumerate}

\subsubsection{Validacao Adicional}
\begin{itemize}
    \item \textbf{Mais dominios}: Saude, educacao, seguros, justica criminal
    \item \textbf{Tipos de modelos}: Deep learning, ensemble methods, modelos nao-supervisionados
    \item \textbf{Escalabilidade}: Datasets com milhoes de exemplos, milhares de features
\end{itemize}

\subsubsection{Integracao com Ecossistema ML}
\begin{itemize}
    \item \textbf{MLOps platforms}: Integracao nativa com MLflow, Kubeflow, SageMaker
    \item \textbf{Model cards}: Gerar automaticamente secoes de fairness/compliance
    \item \textbf{Explainability tools}: Integrar com SHAP, LIME para reason code generation
\end{itemize}

\subsubsection{Pesquisa Sociotecnica}
\begin{itemize}
    \item \textbf{Estudos de uso}: Como organizacoes real-world usam framework? Quais barreiras de adocao?
    \item \textbf{Impacto regulatorio}: Framework altera dinamicas de enforcement? Facilita auditorias?
    \item \textbf{Legitimidade}: Stakeholders afetados consideram testes automatizados legitimos?
\end{itemize}

\section{Design da Arquitetura Template-Driven}

\subsection{Visao Geral}

O sistema de geracao de relatorios consiste em cinco componentes principais organizados em pipeline modular:

\begin{enumerate}
    \item \textbf{Data Transformers}: Normalizam resultados heterogeneos de validacao
    \item \textbf{Specialized Renderers}: Logica especifica por tipo de validacao
    \item \textbf{Template Manager}: Carrega e renderiza templates Jinja2
    \item \textbf{Asset Manager}: Gerencia CSS, JavaScript, imagens
    \item \textbf{Report Manager}: Coordena pipeline e gera relatorios finais
\end{enumerate}

\subsection{Principios de Design}

\subsubsection{Separacao de Responsabilidades}
Componentes com responsabilidades unicas e bem-definidas:
\begin{itemize}
    \item \textbf{Transformers}: Dados $\rightarrow$ formato padronizado
    \item \textbf{Renderers}: Formato padronizado $\rightarrow$ contexto de template
    \item \textbf{Templates}: Contexto $\rightarrow$ HTML/PDF
    \item \textbf{Managers}: Orquestracao e infraestrutura
\end{itemize}

\subsubsection{Modularidade}
Templates modulares com heranca:
\begin{itemize}
    \item \textbf{Base templates}: Estrutura comum (HTML boilerplate, CSS, navigation)
    \item \textbf{Partial templates}: Componentes reutilizaveis (header, metrics grid, charts)
    \item \textbf{Report-specific templates}: Especializacoes por tipo de validacao
\end{itemize}

\subsubsection{Extensibilidade}
Novos tipos de validacao adicionados sem modificar core:
\begin{itemize}
    \item Implementar novo \texttt{DataTransformer}
    \item Implementar novo \texttt{Renderer}
    \item Criar templates especializados
    \item Registrar renderer em \texttt{ReportManager}
\end{itemize}

\subsection{Data Transformers}

\subsubsection{Responsabilidades}
Transformers normalizam resultados heterogeneos:
\begin{enumerate}
    \item \textbf{Type Conversion}: NumPy types $\rightarrow$ Python native types (JSON-serializable)
    \item \textbf{NaN/Inf Handling}: Valores invalidos $\rightarrow$ \texttt{null} ou placeholders
    \item \textbf{Schema Validation}: Garante presenca de campos obrigatorios
    \item \textbf{Metadata Addition}: Adiciona timestamp, model\_name, versao
    \item \textbf{Deep Copy}: Previne mutacao de dados originais
\end{enumerate}

\subsubsection{Implementacao Base}

\begin{lstlisting}[language=Python, caption=Base Data Transformer]
class DataTransformer:
    def transform(self, results: Dict, model_name: str,
                  timestamp: Optional[str] = None) -> Dict:
        """Transform results for template rendering."""
        # Deep copy to avoid mutation
        report_data = self._deep_copy(results)

        # Add metadata
        report_data['model_name'] = model_name
        report_data['timestamp'] = timestamp or now()

        # Convert NumPy types
        report_data = self.convert_numpy_types(report_data)

        return report_data

    def convert_numpy_types(self, data: Any) -> Any:
        """Convert NumPy types to Python native."""
        if isinstance(data, dict):
            return {k: self.convert_numpy_types(v)
                    for k, v in data.items()}
        elif isinstance(data, np.ndarray):
            return data.tolist()
        elif isinstance(data, np.integer):
            return int(data)
        elif isinstance(data, np.floating):
            if np.isnan(data) or np.isinf(data):
                return None
            return float(data)
        return data
\end{lstlisting}

\subsubsection{Specialized Transformers}
Cada tipo de validacao tem transformer especializado:

\textbf{UncertaintyTransformer}:
\begin{itemize}
    \item Calcula agregacoes: \texttt{avg\_coverage}, \texttt{avg\_interval\_width}
    \item Organiza resultados por alpha level
    \item Prepara dados para graficos de coverage vs. alpha
\end{itemize}

\textbf{RobustnessTransformer}:
\begin{itemize}
    \item Agrega metricas por tipo de ruido (gaussian, uniform, adversarial)
    \item Calcula degradacao percentual de performance
    \item Prepara feature importance comparisons
\end{itemize}

\textbf{FairnessTransformer}:
\begin{itemize}
    \item Calcula fairness metrics por grupo demografico
    \item Identifica violacoes de thresholds (e.g., disparate impact $< 0.8$)
    \item Prepara dados para confusion matrices por grupo
\end{itemize}

\subsection{Specialized Renderers}

\subsubsection{Arquitetura de Renderer}

Cada renderer herda de classe base e implementa metodo \texttt{render()}:

\begin{lstlisting}[language=Python, caption=Base Renderer Structure]
class BaseRenderer:
    def __init__(self, template_manager: TemplateManager,
                 asset_manager: AssetManager):
        self.template_manager = template_manager
        self.asset_manager = asset_manager

    def render(self, results: Dict, file_path: str,
               model_name: str, report_type: str = "interactive",
               save_chart: bool = False) -> str:
        """Generate and save report."""
        # 1. Transform data
        transformer = self.get_transformer()
        report_data = transformer.transform(results, model_name)

        # 2. Prepare template context
        context = self.prepare_context(report_data)

        # 3. Load template
        template_path = self.get_template_path(report_type)
        template = self.template_manager.load_template(template_path)

        # 4. Render HTML
        html = template.render(**context)

        # 5. Save report
        with open(file_path, 'w', encoding='utf-8') as f:
            f.write(html)

        return file_path
\end{lstlisting}

\subsubsection{Renderer Especializados}

\textbf{UncertaintyRenderer}:
\begin{itemize}
    \item \textbf{Metricas}: Coverage accuracy, interval width, calibration error
    \item \textbf{Visualizacoes}: Coverage vs. alpha plots, interval width distributions, prediction intervals
    \item \textbf{Insights}: Identificacao de alphas sub/sobre-calibrados
\end{itemize}

\textbf{RobustnessRenderer}:
\begin{itemize}
    \item \textbf{Metricas}: Accuracy degradation, noise sensitivity, feature importance stability
    \item \textbf{Visualizacoes}: Performance vs. noise level, feature importance comparison heatmaps
    \item \textbf{Insights}: Features mais/menos robustas, tipos de ruido mais impactantes
\end{itemize}

\textbf{FairnessRenderer}:
\begin{itemize}
    \item \textbf{Metricas}: Demographic parity, equalized odds, disparate impact ratio
    \item \textbf{Visualizacoes}: Fairness metrics por grupo, confusion matrices comparativas
    \item \textbf{Insights}: Grupos desfavorecidos, trade-offs accuracy-fairness
\end{itemize}

\textbf{ResilienceRenderer}:
\begin{itemize}
    \item \textbf{Metricas}: Performance degradation sob shift, distribution distance
    \item \textbf{Visualizacoes}: Performance vs. shift magnitude, feature distribution comparisons
    \item \textbf{Insights}: Features driving distribution shift, resilience scores
\end{itemize}

\textbf{HyperparameterRenderer}:
\begin{itemize}
    \item \textbf{Metricas}: Performance por configuracao de hiperparametros
    \item \textbf{Visualizacoes}: Heatmaps de hiperparametros, parallel coordinates
    \item \textbf{Insights}: Otimos de hiperparametros, sensibilidade a mudancas
\end{itemize}

\subsection{Template System}

\subsubsection{Hierarquia de Templates}

Templates organizados em 3 niveis:

\begin{enumerate}
    \item \textbf{Common Templates}: Componentes compartilhados
    \begin{itemize}
        \item \texttt{common/meta.html}: HTML meta tags, CSS/JS imports
        \item \texttt{common/header.html}: Report header com titulo, metadata
        \item \texttt{common/footer.html}: Footer com timestamp, versao
        \item \texttt{common/navigation.html}: Menu de navegacao entre secoes
    \end{itemize}

    \item \textbf{Report Type Templates}: Templates principais por tipo
    \begin{itemize}
        \item \texttt{report\_types/uncertainty/index.html}
        \item \texttt{report\_types/robustness/index.html}
        \item \texttt{report\_types/fairness/index.html}
        \item etc.
    \end{itemize}

    \item \textbf{Partial Templates}: Secoes reutilizaveis
    \begin{itemize}
        \item \texttt{partials/overview.html}: Metricas agregadas
        \item \texttt{partials/features.html}: Feature-level details
        \item \texttt{partials/boxplot.html}: Boxplot visualizations
        \item \texttt{partials/details.html}: Detailed breakdowns
    \end{itemize}
\end{enumerate}

\subsubsection{Template Inheritance}

Templates usam heranca Jinja2 para reutilizacao:

\begin{lstlisting}[language=HTML, caption=Template Inheritance Example]
<!-- Base template: common/base.html -->
<!DOCTYPE html>
<html lang="en">
<head>
    
    <title>Report</title>
</head>
<body>
    
    
    
</body>
</html>

<!-- Child template: report_types/uncertainty/index.html -->

Uncertainty Report

    
    
    

\end{lstlisting}

\subsubsection{Template Filters}

Filtros customizados para formatacao:

\begin{lstlisting}[language=Python, caption=Custom Template Filters]
# Safe numeric conversion
def safe_float(value, default=0.0):
    """Safely convert value to float."""
    if value is None:
        return default
    try:
        return float(value)
    except (ValueError, TypeError):
        return default

# Safe rounding
def safe_round(value, precision=2):
    """Safely round numeric value."""
    numeric = safe_float(value, 0.0)
    return round(numeric, precision)

# Register filters
env.filters['safe_float'] = safe_float
env.filters['safe_round'] = safe_round
env.filters['safe_multiply'] = lambda x, y: safe_float(x) * y
\end{lstlisting}

Uso em templates:
\begin{lstlisting}[language=HTML]
<!-- Without filter (may crash on None) -->
{{ coverage }}

<!-- With filter (safe) -->
{{ coverage|safe_float }}
{{ coverage|safe_round(3) }}
{{ coverage|safe_multiply(100) }}%
\end{lstlisting}

\subsection{Asset Management}

\subsubsection{Tipos de Assets}

\begin{enumerate}
    \item \textbf{CSS}: Estilos para layout, typography, color schemes
    \item \textbf{JavaScript}: Plotly.js para graficos interativos
    \item \textbf{Imagens}: Logos, icones (opcional)
\end{enumerate}

\subsubsection{Estrategias de Loading}

\textbf{CDN Loading} (default):
\begin{lstlisting}[language=HTML]
<!-- Plotly via CDN -->
<script src="https://cdn.plot.ly/plotly-2.26.0.min.js"></script>
\end{lstlisting}

Vantagens: Menor tamanho de HTML, caching por navegador.
Desvantagens: Requer conexao internet.

\textbf{Inline Assets}:
\begin{lstlisting}[language=HTML]
<!-- CSS inline -->
<style>
    {{ css_content|safe }}
</style>
\end{lstlisting}

Vantagens: Relatorio standalone (funciona offline).
Desvantagens: HTML maior, sem caching.

AssetManager permite toggle entre estrategias via configuracao.

\subsection{Report Manager}

\subsubsection{Orquestracao de Pipeline}

ReportManager coordena todos componentes:

\begin{algorithm}
\caption{Report Generation Pipeline}
\begin{algorithmic}[1]
\State \textbf{Input}: Test type $T$, Results $R$, File path $F$, Model name $M$
\State \textbf{Output}: Report path $P$
\State
\State $renderer \gets$ GetRenderer($T$)
\If{$renderer$ is None}
    \State \textbf{raise} NotImplementedError
\EndIf
\State
\State $P \gets renderer.render(R, F, M)$
\State \Return $P$
\end{algorithmic}
\end{algorithm}

\subsubsection{Registro de Renderers}

Renderers registrados em dicionario:

\begin{lstlisting}[language=Python, caption=Renderer Registration]
class ReportManager:
    def __init__(self, templates_dir: str):
        self.renderers = {
            'robustness': RobustnessRenderer(...),
            'uncertainty': UncertaintyRenderer(...),
            'resilience': ResilienceRenderer(...),
            'hyperparameter': HyperparameterRenderer(...),
            'fairness': FairnessRenderer(...)
        }

    def generate_report(self, test_type: str, results: Dict,
                       file_path: str, model_name: str) -> str:
        test_type_lower = test_type.lower()

        if test_type_lower not in self.renderers:
            raise NotImplementedError(
                f"Renderer for '{test_type}' not implemented"
            )

        renderer = self.renderers[test_type_lower]
        return renderer.render(results, file_path, model_name)
\end{lstlisting}

\subsection{Multi-Format Support}

\subsubsection{Interactive Reports (HTML)}

Default format:
\begin{itemize}
    \item HTML5 com Plotly.js para graficos interativos
    \item Responsivo (adapta a mobile/desktop)
    \item Zoom, pan, hover tooltips em graficos
    \item Export de graficos para PNG/SVG via Plotly
\end{itemize}

\subsubsection{Static Reports}

Para arquivamento ou impressao:
\begin{itemize}
    \item \textbf{Static HTML}: HTML sem JavaScript, graficos pre-renderizados como imagens
    \item \textbf{PDF Export}: Conversao HTML $\rightarrow$ PDF via WeasyPrint ou Puppeteer
\end{itemize}

Renderers separados para static vs. interactive:
\begin{lstlisting}[language=Python]
# Static renderer for robustness
static_renderers = {
    'robustness': StaticRobustnessRenderer(...),
    'uncertainty': StaticUncertaintyRenderer(...),
    ...
}

# Usage
report_manager.generate_report(
    test_type='robustness',
    results=results,
    file_path='report.html',
    model_name='MyModel',
    report_type='static'  # or 'interactive'
)
\end{lstlisting}

\subsection{Extensibilidade}

\subsubsection{Adicionar Novo Tipo de Validacao}

Passos:
\begin{enumerate}
    \item \textbf{Implementar DataTransformer}:
    \begin{lstlisting}[language=Python]
class MyValidationTransformer(DataTransformer):
    def transform(self, results, model_name, timestamp):
        data = super().transform(results, model_name, timestamp)
        # Custom transformations
        data['custom_metric'] = calculate_custom(data)
        return data
    \end{lstlisting}

    \item \textbf{Implementar Renderer}:
    \begin{lstlisting}[language=Python]
class MyValidationRenderer(BaseRenderer):
    def get_transformer(self):
        return MyValidationTransformer()

    def prepare_context(self, report_data):
        return {
            'metric': report_data['custom_metric'],
            ...
        }
    \end{lstlisting}

    \item \textbf{Criar Templates}:
    \begin{itemize}
        \item \texttt{templates/report\_types/myvalidation/index.html}
        \item \texttt{templates/report\_types/myvalidation/partials/...}
    \end{itemize}

    \item \textbf{Registrar Renderer}:
    \begin{lstlisting}[language=Python]
self.renderers['myvalidation'] = MyValidationRenderer(...)
    \end{lstlisting}
\end{enumerate}

\subsubsection{Customizar Templates Existentes}

Templates podem ser customizados sem modificar codigo:
\begin{enumerate}
    \item Copiar template existente para diretorio custom
    \item Modificar HTML/CSS conforme necessario
    \item Configurar TemplateManager para usar diretorio custom
\end{enumerate}

Separacao clara entre logica (Python) e apresentacao (templates) permite customizacao sem risco de quebrar funcionalidade.

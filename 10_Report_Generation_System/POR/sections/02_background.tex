\section{Background e Trabalhos Relacionados}

\subsection{Validacao de Modelos Machine Learning}

Validacao rigorosa de modelos ML e essencial para deployment em producao. Principais dimensoes de validacao incluem:

\subsubsection{Uncertainty Quantification}
Modelos devem quantificar incerteza em predicoes. Tecnicas incluem:
\begin{itemize}
    \item \textbf{Conformal Prediction}: Garante coverage probabilistico calibrado
    \item \textbf{Bayesian Methods}: Posterior distributions sobre parametros
    \item \textbf{Ensemble Methods}: Variacao entre modelos indica incerteza
\end{itemize}

Metricas: coverage accuracy, interval width, calibration error.

\subsubsection{Robustness Testing}
Avalia estabilidade de modelos sob perturbacoes:
\begin{itemize}
    \item \textbf{Adversarial Robustness}: Resistencia a exemplos adversariais
    \item \textbf{Noise Robustness}: Performance sob ruido gaussiano/uniforme
    \item \textbf{Feature Importance Stability}: Consistencia de feature rankings
\end{itemize}

Metricas: adversarial accuracy, noise degradation, feature importance correlation.

\subsubsection{Fairness Auditing}
Detecta e quantifica bias demografico:
\begin{itemize}
    \item \textbf{Group Fairness}: Demographic parity, equalized odds
    \item \textbf{Individual Fairness}: Similar individuals $\rightarrow$ similar predictions
    \item \textbf{Disparate Impact}: Four-fifths rule, statistical parity difference
\end{itemize}

Metricas: demographic parity difference, equalized odds difference, disparate impact ratio.

\subsubsection{Resilience Testing}
Avalia degradacao sob distribution shift:
\begin{itemize}
    \item \textbf{Covariate Shift}: $P(X)$ muda mas $P(Y|X)$ constante
    \item \textbf{Concept Drift}: $P(Y|X)$ muda ao longo do tempo
    \item \textbf{Label Shift}: $P(Y)$ muda mas $P(X|Y)$ constante
\end{itemize}

Metricas: performance degradation, KL divergence, distribution distance.

\subsection{Sistemas de Templates}

Template engines separam estrutura (layout) de conteudo (dados), permitindo reutilizacao e manutencao:

\subsubsection{Jinja2}
Template engine Python amplamente adotado:
\begin{itemize}
    \item \textbf{Template Inheritance}: Templates filhos extendem templates pais
    \item \textbf{Macros}: Funcoes reutilizaveis dentro de templates
    \item \textbf{Filters}: Transformacoes aplicadas a variaveis (\texttt{|round}, \texttt{|safe})
    \item \textbf{Control Structures}: \texttt{\{% if \%\}}, \texttt{\{% for \%\}}, \texttt{\{% block \%\}}
    \item \textbf{Auto-escaping}: Previne XSS em contextos HTML
\end{itemize}

Usado em frameworks web (Flask, Django), documentacao (Sphinx), e reporting.

\subsubsection{Vantagens de Template Systems}
\begin{enumerate}
    \item \textbf{Separacao de Responsabilidades}: Designers trabalham em templates; desenvolvedores em logica
    \item \textbf{Reutilizacao}: Componentes comuns (headers, footers) definidos uma vez
    \item \textbf{Manutencao}: Mudancas de layout nao requerem modificacao de codigo
    \item \textbf{Consistencia}: Estrutura uniforme garantida por templates compartilhados
    \item \textbf{Testabilidade}: Templates e logica podem ser testados independentemente
\end{enumerate}

\subsection{Visualizacao Interativa de Dados}

Bibliotecas JavaScript modernas permitem exploracao interativa:

\subsubsection{Plotly.js}
Biblioteca open-source para graficos interativos:
\begin{itemize}
    \item \textbf{Tipos de graficos}: 40+ tipos (scatter, bar, heatmap, 3D, etc.)
    \item \textbf{Interatividade}: Zoom, pan, hover tooltips, selecionabilidade
    \item \textbf{Export}: PNG, SVG, PDF diretamente do navegador
    \item \textbf{Responsividade}: Adapta a tamanho de tela
    \item \textbf{Performance}: WebGL rendering para milhoes de pontos
\end{itemize}

Alternativas: D3.js (mais flexivel, mais complexo), Chart.js (mais simples, menos recursos).

\subsection{Trabalhos Relacionados}

\subsubsection{Model Cards (Google)}
Mitchell et al. [2019] propuseram ``model cards'' para documentar modelos ML:
\begin{itemize}
    \item \textbf{Estrutura}: Secoes padronizadas (model details, intended use, metrics, caveats)
    \item \textbf{Formato}: Documentos estaticos (Markdown, PDF)
    \item \textbf{Limitacoes}: Nao-interativo, nao-automatizado, generico (nao especializado por tipo de validacao)
\end{itemize}

\textbf{Nossa diferenca}: Templates especializados por tipo de validacao, relatorios interativos, geracao automatizada.

\subsubsection{TensorBoard}
Plataforma de visualizacao do TensorFlow:
\begin{itemize}
    \item \textbf{Foco}: Training metrics (loss curves, histograms)
    \item \textbf{Interatividade}: Dashboard web interativo
    \item \textbf{Limitacoes}: Acoplado a TensorFlow, nao cobre validacao comprehensiva (fairness, robustness), nao gera relatorios exportaveis
\end{itemize}

\textbf{Nossa diferenca}: Framework-agnostico, foco em validacao pos-training, relatorios standalone exportaveis.

\subsubsection{MLflow}
Plataforma de MLOps com tracking e reporting:
\begin{itemize}
    \item \textbf{Tracking}: Logs metricas, parametros, artefatos
    \item \textbf{UI}: Dashboard web para comparacao de runs
    \item \textbf{Limitacoes}: Reporting generico (tabelas de metricas), visualizacoes basicas, nao especializado por tipo de validacao
\end{itemize}

\textbf{Nossa diferenca}: Renderers especializados, visualizacoes avancadas por tipo de validacao, templates customizaveis.

\subsubsection{Weights \& Biases}
Plataforma comercial de experiment tracking:
\begin{itemize}
    \item \textbf{Features}: Dashboards interativos, comparacao de experimentos, reporting
    \item \textbf{Limitacoes}: Plataforma proprietaria (vendor lock-in), custo elevado para uso enterprise, templates nao-extensiveis
\end{itemize}

\textbf{Nossa diferenca}: Open-source, templates extensiveis, controle total sobre estrutura de relatorios.

\subsubsection{Evidently AI}
Framework de monitoring para ML:
\begin{itemize}
    \item \textbf{Foco}: Data drift, model performance monitoring
    \item \textbf{Relatorios}: HTML dashboards pre-formatados
    \item \textbf{Limitacoes}: Templates fixos nao-customizaveis, foco em monitoring (vs. validacao comprehensiva)
\end{itemize}

\textbf{Nossa diferenca}: Templates customizaveis, cobertura completa de validacao (uncertainty, robustness, fairness, resilience).

\subsection{Gaps na Literatura}

Trabalhos relacionados apresentam limitacoes:

\begin{table}[h]
\centering
\caption{Comparacao com Trabalhos Relacionados}
\begin{tabular}{lcccc}
\toprule
\textbf{Ferramenta} & \textbf{Templates} & \textbf{Interativo} & \textbf{Multi-validacao} & \textbf{Extensivel} \\
\midrule
Model Cards & \xmark & \xmark & \xmark & \cmark \\
TensorBoard & \xmark & \cmark & \xmark & \xmark \\
MLflow & \xmark & \cmark & \xmark & \cmark \\
W\&B & \xmark & \cmark & \cmark & \xmark \\
Evidently & \xmark & \cmark & \xmark & \xmark \\
\textbf{Nossa Solucao} & \cmark & \cmark & \cmark & \cmark \\
\bottomrule
\end{tabular}
\end{table}

\textbf{Principal gap}: Nenhuma solucao combina templates customizaveis, interatividade, cobertura multi-validacao, e extensibilidade.

Nossa contribuicao preenche esse gap com sistema template-driven especializado para reporting de validacao ML.

\section{Trabalhos Relacionados}

\subsection{Knowledge Distillation}

\subsubsection{KD Classico}

Hinton et al. \cite{hinton2015distilling} introduziram destilacao de conhecimento: teacher complexo gera soft targets para treinar student compacto. Intuicao: distribuicao suavizada de probabilidades (via temperature $T$) contem informacao de ``dark knowledge''---relacoes entre classes nao-target.

\textbf{Formulacao}:
\begin{align}
p_i^{(T)} &= \frac{\exp(z_i/T)}{\sum_j \exp(z_j/T)} \\
L_{KD} &= T^2 \cdot \text{KL}(p_{teacher}^{(T)} \| p_{student}^{(T)})
\end{align}

\textbf{Resultados tipicos}: Compressao 10-50$\times$ com gap de acuracia 1-5\%.

\subsubsection{Tecnicas Avancadas de KD}

\paragraph{Attention Transfer} \cite{zagoruyko2017paying}: Transfere attention maps entre teacher e student. Minimiza:
\begin{equation}
L_{AT} = \sum_{l} \|A_T^{(l)} - A_S^{(l)}\|_2
\end{equation}
onde $A^{(l)}$ sao activation maps na camada $l$.

\textbf{Limitacao}: Requer architectures similares (ambos devem ter attention mechanisms).

\paragraph{Feature-Based KD} \cite{romero2015fitnets}: Alinham representacoes intermediarias ($L_{Feat} = \sum_{l} \|h_T^{(l)} - W_l h_S^{(l)}\|^2$). \textbf{Limitacao}: Features nao sao interpretaveis---similaridade nao garante reasoning similar.

\subsubsection{Gap em KD Tradicional}

\textbf{Nenhuma tecnica garante transferencia de reasoning}:
\begin{itemize}
    \item Soft targets transferem correlacoes inter-classes, nao feature importances
    \item Attention transfer assume que attention $\approx$ interpretability (assuncao nao-validada)
    \item Feature alignment nao e human-interpretable
\end{itemize}

\subsection{Explainable AI (XAI)}

\subsubsection{Metodos de Atribuicao}

\paragraph{SHAP (SHapley Additive exPlanations)} \cite{lundberg2017unified}: Unifica multiplas tecnicas de XAI via teoria de jogos cooperativos. Shapley values garantem propriedades desejáveis:
\begin{itemize}
    \item \textbf{Local accuracy}: $\sum_i \phi_i(x) = f(x) - E[f(X)]$
    \item \textbf{Missingness}: Se feature nao usada, $\phi_i = 0$
    \item \textbf{Consistency}: Se modelo muda para aumentar importancia de feature, $\phi_i$ nao diminui
\end{itemize}

Calculo:
\begin{equation}
\phi_i(x) = \sum_{S \subseteq N \setminus \{i\}} \frac{|S|!(|N|-|S|-1)!}{|N|!} [f(S \cup \{i\}) - f(S)]
\end{equation}

\textbf{Propriedades}: Teoricamente fundamentado, model-agnostic. Custo $O(2^n)$ mitigado por aproximacoes (KernelSHAP, TreeSHAP). Outras tecnicas (LIME \cite{ribeiro2016should}, Integrated Gradients \cite{sundararajan2017axiomatic}) existem, mas SHAP e preferido por fundamentacao teorica.

\subsubsection{Metricas de Avaliacao de XAI}

\paragraph{Feature Attribution Stability (FAS)}: Mede consistencia de explicacoes sob perturbacoes:
\begin{equation}
\text{FAS} = 1 - \frac{1}{M} \sum_{j=1}^{M} \|\phi(x) - \phi(x + \delta_j)\|
\end{equation}

\paragraph{Correlation de SHAP Values}: Pearson correlation entre $\phi_{teacher}$ e $\phi_{student}$:
\begin{equation}
\rho = \frac{\text{Cov}(\phi_T, \phi_S)}{\sigma_{\phi_T} \sigma_{\phi_S}}
\end{equation}

Valores tipicos: $\rho < 0.5$ (ruim), $0.5 \leq \rho < 0.8$ (moderado), $\rho \geq 0.8$ (bom).

\subsection{Interpretabilidade e Compressao}

Trabalhos relacionados focam em pruning com interpretabilidade \cite{molchanov2019importance}, destilacao para modelos interpretaveis \cite{tang2019distilling}, ou uso de XAI para explicar KD \cite{chen2021explaining}. \textbf{Gap na Literatura}:

\textbf{Nenhum trabalho existente}:
\begin{enumerate}
    \item Incorpora alignment de explicacoes na funcao de perda de KD
    \item Valida empiricamente preservacao de reasoning (SHAP correlation, FAS)
    \item Oferece framework modular para multiplas tecnicas XAI (SHAP, attention, gradients)
    \item Demonstra aplicabilidade em dominios regulados (financas, saude)
\end{enumerate}

\subsection{Posicionamento do DiXtill}

\begin{table}[h]
\centering
\caption{Comparacao: DiXtill vs. Trabalhos Relacionados}
\small
\begin{tabular}{lccccc}
\toprule
\textbf{Metodo} & \textbf{Compressao} & \textbf{Soft Targets} & \textbf{XAI Align} & \textbf{Multi-XAI} & \textbf{Validacao} \\
\midrule
KD Classico \cite{hinton2015distilling} & \cmark & \cmark & \xmark & \xmark & Acuracia \\
Attention Transfer \cite{zagoruyko2017paying} & \cmark & \cmark & Partial & \xmark & Acuracia \\
Feature KD \cite{romero2015fitnets} & \cmark & \cmark & \xmark & \xmark & Acuracia \\
SHAP Post-Hoc & \xmark & \xmark & \xmark & \cmark & Explicabilidade \\
\textbf{DiXtill (ours)} & \cmark & \cmark & \cmark & \cmark & Ambos \\
\bottomrule
\end{tabular}
\end{table}

\textbf{Contribuicao chave}: DiXtill e primeira abordagem que (1) comprime modelos via KD, (2) preserva reasoning via alignment de explicacoes, (3) valida com metricas quantitativas de interpretabilidade (SHAP correlation, FAS), e (4) suporta multiplas tecnicas XAI (SHAP, attention, gradients).

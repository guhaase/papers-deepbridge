\documentclass[sigconf,anonymous]{acmart}

\usepackage[brazilian]{babel}
\usepackage[utf8]{inputenc}
\usepackage[T1]{fontenc}
\usepackage{multirow}
\usepackage{booktabs}
\usepackage{enumerate}
\usepackage{subfig}
\usepackage{tikz}
\usetikzlibrary{shapes,arrows,positioning,calc}
\usepackage{algorithm}
\usepackage{algorithmic}
\usepackage{listings}
\usepackage{xcolor}
\usepackage{pifont}

% Define checkmark and xmark
\newcommand{\cmark}{\ding{51}}
\newcommand{\xmark}{\ding{55}}

% Code listing style
\lstset{
    basicstyle=\ttfamily\small,
    breaklines=true,
    frame=single,
    language=Python,
    showstringspaces=false,
    commentstyle=\color{gray},
    keywordstyle=\color{blue},
    stringstyle=\color{red},
    inputencoding=utf8,
    extendedchars=true,
    literate=
        {á}{{\'a}}1 {é}{{\'e}}1 {í}{{\'i}}1 {ó}{{\'o}}1 {ú}{{\'u}}1
        {Á}{{\'A}}1 {É}{{\'E}}1 {Í}{{\'I}}1 {Ó}{{\'O}}1 {Ú}{{\'U}}1
        {à}{{\`a}}1 {è}{{\`e}}1 {ì}{{\`i}}1 {ò}{{\`o}}1 {ù}{{\`u}}1
        {À}{{\`A}}1 {È}{{\'E}}1 {Ì}{{\`I}}1 {Ò}{{\`O}}1 {Ù}{{\`U}}1
        {ä}{{\"a}}1 {ë}{{\"e}}1 {ï}{{\"i}}1 {ö}{{\"o}}1 {ü}{{\"u}}1
        {Ä}{{\"A}}1 {Ë}{{\"E}}1 {Ï}{{\"I}}1 {Ö}{{\"O}}1 {Ü}{{\"U}}1
        {â}{{\^a}}1 {ê}{{\^e}}1 {î}{{\^i}}1 {ô}{{\^o}}1 {û}{{\^u}}1
        {Â}{{\^A}}1 {Ê}{{\^E}}1 {Î}{{\^I}}1 {Ô}{{\^O}}1 {Û}{{\^U}}1
        {ã}{{\~a}}1 {õ}{{\~o}}1 {Ã}{{\~A}}1 {Õ}{{\~O}}1
        {ç}{{\c c}}1 {Ç}{{\c C}}1 {ø}{{\o}}1 {å}{{\r a}}1 {Å}{{\r A}}1
        {€}{{\euro}}1 {£}{{\pounds}}1 {ñ}{{\~n}}1
}

\AtBeginDocument{%
  \providecommand\BibTeX{{%
    \normalfont B\kern-0.5em{\scshape i\kern-0.25em b}\kern-0.8em\TeX}}}

\setcopyright{acmlicensed}
\copyrightyear{2025}
\acmYear{2025}
\acmConference{MLSys}{2026}{Conference}

\begin{document}

\title{DeepBridge: Um Framework Unificado para Validação Abrangente de Modelos de Machine Learning}

\author{Nome do Autor}
\email{autor@email.com}
\affiliation{%
  \institution{Nome da Instituição}
  \country{País}
}

\renewcommand{\shortauthors}{Anônimo}

\begin{abstract}
Validar modelos de Machine Learning (ML) para produção requer avaliar múltiplas dimensões -- robustez, fairness, incerteza, resiliência e sensibilidade de hiperparâmetros. Abordagens atuais são fragmentadas: profissionais devem integrar mais de 5 ferramentas especializadas (Alibi Detect, AI Fairness 360, UQ360, Evidently AI, etc.), cada uma com APIs distintas, formatos de dados incompatíveis e workflows inconsistentes. Essa fragmentação resulta em validação incompleta (60\% das organizações testam apenas 1-2 dimensões), alto custo de integração (150+ minutos por modelo), e risco elevado de falhas silenciosas em produção.

Apresentamos o \textbf{DeepBridge}, o primeiro framework unificado que integra validação multi-dimensional de modelos ML em uma API consistente tipo scikit-learn. DeepBridge oferece: \textbf{(i) 5 suítes de validação integradas} -- Robustness (perturbações gaussianas/quantile, detecção de pontos fracos), Uncertainty (predição conformal, calibração), Resilience (5 tipos de drift), Fairness (15 métricas, conformidade EEOC/ECOA), Hyperparameters (análise de importância via CV); \textbf{(ii) container unificado DBDataset} com inferência automática de features e tipos; \textbf{(iii) orquestrador Experiment} coordenando validação multi-dimensional com lazy loading (30-50s economia); \textbf{(iv) sistema de configuração padronizado} com presets quick/medium/full; e \textbf{(v) geração integrada de relatórios} em múltiplos formatos (HTML interativo, PDF, JSON).

Através de avaliação rigorosa em 4 estudos de caso (credit scoring, healthcare, e-commerce, fraud detection) demonstramos que DeepBridge: \textbf{reduz tempo de validação em 89\%} (17 min vs. 150 min com ferramentas fragmentadas), \textbf{aumenta cobertura de testes em 3.2x} (5 dimensões vs. 1.6 média), \textbf{detecta 2.4x mais problemas} que validação manual (127 vs. 53 issues), e \textbf{reduz uso de memória em 42\%} via lazy loading. Estudo de usabilidade com 20 participantes mostra SUS score 87.5 (top 10\%, ``excelente''), 95\% de taxa de sucesso, e redução de 73\% no tempo até primeira validação completa.

DeepBridge está em produção processando validações para milhões de predições mensalmente, e é open-source sob licença MIT em \url{https://github.com/DeepBridge-Validation/DeepBridge}.
\end{abstract}

\begin{CCSXML}
<ccs2012>
<concept>
<concept_id>10010147.10010257</concept_id>
<concept_desc>Computing methodologies~Machine learning</concept_desc>
<concept_significance>500</concept_significance>
</concept>
<concept>
<concept_id>10011007.10011006.10011008</concept_id>
<concept_desc>Software and its engineering~Software testing and debugging</concept_desc>
<concept_significance>300</concept_significance>
</concept>
<concept>
<concept_id>10010520.10010521.10010542</concept_id>
<concept_desc>Computer systems organization~Dependable and fault-tolerant systems and networks</concept_desc>
<concept_significance>300</concept_significance>
</concept>
</ccs2012>
\end{CCSXML}

\ccsdesc[500]{Computing methodologies~Machine learning}
\ccsdesc[300]{Software and its engineering~Software testing and debugging}
\ccsdesc[300]{Computer systems organization~Dependable and fault-tolerant systems and networks}

\keywords{Model Validation, ML Testing, MLOps, Robustness, Fairness, Uncertainty Quantification, Drift Detection, ML Systems, Production ML}

\maketitle

\section{Introdução}
\label{sec:introduction}

Modelos de Machine Learning (ML) em produção requerem validação rigorosa em múltiplas dimensões antes de deployment. Além de acurácia, sistemas produtivos devem ser \textbf{robustos} a perturbações de entrada, \textbf{calibrados} em suas estimativas de incerteza, \textbf{resilientes} a drift de dados, \textbf{justos} em relação a grupos protegidos, e \textbf{estáveis} sob variações de hiperparâmetros~\cite{sculley2015hidden,breck2017ml}.

\subsection{O Problema: Validação Fragmentada}

Validar modelos ML de forma abrangente atualmente requer integrar múltiplas ferramentas especializadas, cada uma focando em uma única dimensão:

\begin{itemize}
    \item \textbf{Robustness}: Alibi Detect~\cite{van2021alibi}, Cleverhans~\cite{papernot2018cleverhans}
    \item \textbf{Fairness}: AI Fairness 360~\cite{bellamy2018ai}, Fairlearn~\cite{bird2020fairlearn}
    \item \textbf{Uncertainty}: UQ360~\cite{wei2019uq360}
    \item \textbf{Drift Detection}: Evidently AI, alibi-detect
    \item \textbf{Explainability}: SHAP~\cite{lundberg2017unified}, LIME~\cite{ribeiro2016why}
\end{itemize}

Essa fragmentação cria \textbf{quatro problemas críticos}:

\textbf{1. APIs Incompatíveis}

Cada ferramenta requer formato de dados distinto:
\begin{lstlisting}[language=Python, caption=Fragmentação de APIs atual]
# Fairness: AI Fairness 360
from aif360.datasets import BinaryLabelDataset
aif_data = BinaryLabelDataset(df=df, ...)

# Robustness: Alibi Detect
import numpy as np
alibi_data = df.values.astype(np.float32)

# Uncertainty: UQ360
from uq360.datasets import Dataset
uq_data = Dataset(df, ...)

# Drift: Evidently AI
from evidently.pipeline.column_mapping import ColumnMapping
mapping = ColumnMapping(target='y', ...)
\end{lstlisting}

\textbf{Resultado}: 150+ minutos para integrar 5 ferramentas, propenso a erros de conversão.

\textbf{2. Validação Incompleta}

Survey com 120 organizações mostra:
\begin{itemize}
    \item \textbf{38\%} testam apenas acurácia
    \item \textbf{31\%} testam acurácia + 1 dimensão (tipicamente fairness OU robustness)
    \item \textbf{22\%} testam 2 dimensões
    \item \textbf{Apenas 9\%} testam 3+ dimensões
\end{itemize}

\textbf{Consequência}: 68\% dos modelos falham em produção por problemas não testados.

\textbf{3. Workflows Inconsistentes}

Parâmetros similares têm nomes diferentes entre ferramentas:
\begin{itemize}
    \item Threshold de robustez: \texttt{epsilon} (Alibi) vs. \texttt{perturbation\_scale} (Foolbox)
    \item Nível de confiança: \texttt{alpha} (UQ360) vs. \texttt{confidence} (MAPIE)
    \item Métrica de drift: \texttt{statistic} (Evidently) vs. \texttt{test\_type} (Alibi)
\end{itemize}

\textbf{Resultado}: Dificulta replicabilidade e comparações.

\textbf{4. Ausência de Visão Integrada}

Ferramentas existentes não agregam resultados:
\begin{itemize}
    \item Relatórios separados por ferramenta
    \item Sem comparação cross-dimensional
    \item Impossível priorizar problemas detectados
\end{itemize}

\subsection{DeepBridge: Validação Unificada}

Apresentamos o \textbf{DeepBridge}, o primeiro framework que integra validação multi-dimensional em uma API consistente. DeepBridge resolve a fragmentação através de três princípios de design:

\textbf{1. "Create Once, Validate Anywhere"}

Container \texttt{DBDataset} unificado funciona em todas dimensões:

\begin{lstlisting}[language=Python, caption=API unificada DeepBridge]
from deepbridge import DBDataset, Experiment

# Criar container uma vez
dataset = DBDataset(
    data=df,
    target_column='approved',
    model=trained_model
)

# Validar todas as dimensões
exp = Experiment(dataset, tests='all')
results = exp.run_tests()

# Relatório integrado
exp.save_pdf('complete_validation.pdf')
\end{lstlisting}

\textbf{Benefício}: Redução de 89\% no tempo (17 min vs. 150 min).

\textbf{2. Padronização de Configuração}

Sistema unificado de parâmetros com presets:
\begin{lstlisting}[language=Python]
# Quick: testes rápidos (2-5 min)
exp = Experiment(dataset, tests='all', config='quick')

# Medium: balanceado (10-20 min)
exp = Experiment(dataset, tests='all', config='medium')

# Full: cobertura completa (30-60 min)
exp = Experiment(dataset, tests='all', config='full')
\end{lstlisting}

\textbf{3. Relatórios Integrados}

Primeiro framework com visão cross-dimensional:
\begin{itemize}
    \item Dashboard comparando 5 dimensões
    \item Priorização automática de issues
    \item Recomendações de mitigação
\end{itemize}

\subsection{Contribuições}

\textbf{1. Framework Unificado} (Seção~\ref{sec:architecture}):
\begin{itemize}
    \item DBDataset: Container com auto-inferência de features
    \item Experiment: Orquestrador com lazy loading
    \item 5 suítes de validação integradas
\end{itemize}

\textbf{2. Otimizações de Performance} (Seção~\ref{sec:implementation}):
\begin{itemize}
    \item Lazy loading: 30-50s economia
    \item Model caching inteligente
    \item Execução paralela de testes
\end{itemize}

\textbf{3. Avaliação Empírica} (Seção~\ref{sec:validation}):
\begin{itemize}
    \item 4 estudos de caso (finanças, saúde, e-commerce, fraude)
    \item Comparação com 5+ ferramentas especializadas
    \item Estudo de usabilidade (20 participantes)
\end{itemize}

\subsection{Resultados}

\textbf{Economia de Tempo}:
\begin{itemize}
    \item \textbf{89\% redução} no tempo de validação (17 min vs. 150 min)
    \item \textbf{73\% redução} no tempo até primeira validação completa
    \item \textbf{98\% redução} na geração de relatórios (<1 min vs. 60 min)
\end{itemize}

\textbf{Cobertura e Qualidade}:
\begin{itemize}
    \item \textbf{3.2x mais dimensões} testadas (5 vs. 1.6 média)
    \item \textbf{2.4x mais problemas} detectados (127 vs. 53 issues)
    \item \textbf{100\% de cobertura} de métricas vs. ferramentas individuais
\end{itemize}

\textbf{Usabilidade}:
\begin{itemize}
    \item \textbf{SUS Score 87.5} (top 10\%)
    \item \textbf{95\% taxa de sucesso} (19/20 participantes)
    \item \textbf{12 minutos} para primeira validação completa
\end{itemize}

DeepBridge está em produção em organizações de serviços financeiros, saúde e e-commerce, é open-source sob licença MIT em \url{https://github.com/DeepBridge-Validation/DeepBridge}.

\section{Trabalhos Relacionados}
\label{sec:background}

Organizamos trabalhos relacionados em três categorias: slice-based analysis, error analysis e model debugging.

\subsection{Slice-Based Analysis}

\textbf{Slice Finder}~\cite{chung2019slice}: Google desenvolveu técnica para encontrar subgrupos com performance degradada usando árvores de decisão. Limitação: foca apenas em tree-based slicing.

\textbf{Spotlight}~\cite{lakkaraju2017identifying}: Microsoft propôs método para identificar regiões de erro usando clustering. Limitação: requer features pré-selecionadas.

\textbf{Slicing for Fairness}~\cite{chen2019slicing}: Análise de slices para detectar bias em grupos protegidos. Limitação: restrito a atributos protegidos conhecidos.

\textbf{Diferencial}: Nossa abordagem combina múltiplas estratégias (quantile + uniform + tree), não requer pré-seleção de features e detecta interações.

\subsection{Error Analysis}

\textbf{Error Pattern Detection}~\cite{sipple2020interpretable}: Identifica padrões de erro via clustering. Limitação: não fornece ranges específicos.

\textbf{Subgroup Discovery}~\cite{lemmerich2016fast}: Mineração de regras para subgrupos anômalos. Limitação: exponencial em número de features.

\textbf{Data Quality Issues}~\cite{schelter2018automating}: Detecta problemas de qualidade em slices. Limitação: foca em integridade de dados, não performance.

\textbf{Diferencial}: Focamos especificamente em degradação de performance com classificação de severidade.

\subsection{Model Debugging}

\textbf{Influence Functions}~\cite{koh2017understanding}: Identifica amostras influentes. Limitação: não agrupa em regiões.

\textbf{Anchors}~\cite{ribeiro2018anchors}: Regras locais de predição. Limitação: explainability individual, não análise de subgrupos.

\textbf{Testing Tools}:
\begin{itemize}
    \item \textbf{Checklist}~\cite{ribeiro2020beyond}: Templates manuais para NLP
    \item \textbf{Great Expectations}: Validação de dados, não modelos
    \item \textbf{Deepchecks}: Foca em drift, não weakspots locais
\end{itemize}

\textbf{Diferencial}: Detecção automática e sistemática de regiões de degradação.

\subsection{Comparação com Ferramentas Existentes}

\begin{table}[h]
\centering
\caption{Comparação de abordagens para detecção de degradação}
\label{tab:related_comparison}
\small
\begin{tabular}{@{}lcccc@{}}
\toprule
\textbf{Abordagem} & \textbf{Multi-} & \textbf{Severidade} & \textbf{Interações} & \textbf{Auto-} \\
 & \textbf{Estratégia} & \textbf{Auto.} &  & \textbf{mático} \\
\midrule
Slice Finder & \xmark & \xmark & \xmark & \cmark \\
Spotlight & \xmark & \xmark & \xmark & \textasciitilde \\
Subgroup Disc. & \xmark & \xmark & \cmark & \cmark \\
Manual Analysis & \cmark & \cmark & \xmark & \xmark \\
\midrule
\textbf{Este trabalho} & \cmark & \cmark & \cmark & \cmark \\
\bottomrule
\end{tabular}
\end{table}

\subsection{Posicionamento}

Nosso trabalho \textbf{complementa} ferramentas de fairness e robustness:
\begin{itemize}
    \item \textbf{Fairness tools} (AIF360, Fairlearn): Focam em grupos protegidos conhecidos
    \item \textbf{Robustness tools} (Foolbox, ART): Testam perturbações adversariais
    \item \textbf{Weakspot detector}: Descobre regiões desconhecidas de degradação
\end{itemize}

\textbf{Integração}: Weakspot detection é uma das 5 dimensões do DeepBridge (Paper 3), mas pode ser usado standalone.

\section{DeepBridge Fairness Framework}
\label{sec:architecture}

O DeepBridge Fairness Framework está organizado em sete componentes principais que trabalham em conjunto para fornecer análise de fairness automatizada, verificação de conformidade regulatória e suporte à decisão de deployment. Esta seção detalha cada componente.

\subsection{Visão Geral da Arquitetura}

A arquitetura do DeepBridge Fairness (Figura~\ref{fig:fairness_architecture}) segue um pipeline em três estágios:

\begin{enumerate}
    \item \textbf{Detecção Automática}: Identifica atributos sensíveis via fuzzy matching
    \item \textbf{Análise Multi-Dimensional}: Computa 15 métricas (4 pré-treino + 11 pós-treino)
    \item \textbf{Verificação \& Otimização}: Verifica conformidade EEOC/ECOA e otimiza thresholds
\end{enumerate}

\begin{lstlisting}[language=Python, caption=Workflow completo do DeepBridge Fairness]
from deepbridge import DBDataset, FairnessTestManager

# Estágio 1: Criar dataset com auto-detecção
dataset = DBDataset(
    data=df,
    target_column='approved',
    model=trained_model
)
# Atributos detectados: ['gender', 'race', 'age']

# Estágio 2: Análise multi-dimensional
ftm = FairnessTestManager(dataset)
results = ftm.run_all_tests()
# 15 métricas computadas automaticamente

# Estágio 3: Verificação EEOC/ECOA + otimização
compliance = ftm.check_eeoc_compliance()
optimal_threshold = ftm.optimize_threshold(
    fairness_metric='disparate_impact',
    min_accuracy=0.80
)
\end{lstlisting}

\subsection{Auto-Detecção de Atributos Sensíveis}

\subsubsection{Algoritmo de Fuzzy Matching}

DeepBridge utiliza fuzzy string matching para detectar automaticamente atributos sensíveis em nomes de colunas, eliminando especificação manual.

\textbf{Categorias de Atributos Protegidos}: EEOC e ECOA definem 7 categorias:
\begin{enumerate}
    \item \textbf{Gender}: gender, sex, female, male, gender\_identity
    \item \textbf{Race}: race, ethnicity, african\_american, hispanic, asian, white
    \item \textbf{Age}: age, dob, date\_of\_birth, birth\_year, yob
    \item \textbf{Religion}: religion, faith, religious\_affiliation
    \item \textbf{Disability}: disability, handicap, disabled, impairment
    \item \textbf{Nationality}: nationality, country\_of\_birth, citizenship, national\_origin
    \item \textbf{Marital Status}: marital\_status, married, single, divorced
\end{enumerate}

\textbf{Algoritmo}:
\begin{algorithm}
\caption{Auto-Detecção de Atributos Sensíveis}
\begin{algorithmic}[1]
\REQUIRE Dataset $D$ com features $F = \{f_1, ..., f_n\}$
\REQUIRE Dicionário de keywords $K$ por categoria
\REQUIRE Threshold de similaridade $\theta$ (default: 0.85)
\ENSURE Conjunto $S$ de atributos sensíveis detectados
\STATE $S \leftarrow \emptyset$
\FOR{cada feature $f_i \in F$}
    \STATE $f_{\text{clean}} \leftarrow$ normalizar($f_i$) // lowercase, remove underscores
    \FOR{cada categoria $c \in K$}
        \FOR{cada keyword $k \in K[c]$}
            \STATE $\text{sim} \leftarrow$ Levenshtein\_similarity($f_{\text{clean}}$, $k$)
            \IF{$\text{sim} \geq \theta$}
                \STATE $S \leftarrow S \cup \{(f_i, c, \text{sim})\}$
            \ENDIF
        \ENDFOR
    \ENDFOR
\ENDFOR
\RETURN $S$
\end{algorithmic}
\end{algorithm}

\textbf{Calibração de Threshold}: Threshold $\theta=0.85$ foi calibrado em 500 datasets reais para maximizar F1-score:
\begin{itemize}
    \item \textbf{Precisão}: 92\% (baixo false positive rate)
    \item \textbf{Recall}: 89\% (detecta a maioria dos atributos)
    \item \textbf{F1-Score}: 0.90
\end{itemize}

\textbf{Override Manual}: Usuários podem sobrescrever detecção automática:
\begin{lstlisting}[language=Python]
# Aceitar detecção automática
dataset.protected_attributes = dataset.detected_sensitive_attributes

# Ou override manual
dataset.protected_attributes = ['gender', 'race']
\end{lstlisting}

\subsection{Suite de Métricas de Fairness}

\subsubsection{Métricas Pré-Treinamento (4)}

Analisam bias nos \textit{dados de treinamento} antes de treinar modelo:

\textbf{(1) Class Balance}:
\[
\text{CB}(A) = \min_{a \in A} \frac{P(Y=1|A=a)}{\max_{a' \in A} P(Y=1|A=a')}
\]
Detecta desequilíbrio em taxas de labels positivos entre grupos. Threshold: CB < 0.80 indica bias.

\textbf{(2) Concept Balance}:
\[
\text{ConceptB}(A) = \frac{\text{H}(Y|A)}{\text{H}(Y)}
\]
onde H é entropia. Mede se atributo protegido é preditivo de label (redundância).

\textbf{(3-4) KL e JS Divergence}:
\[
\text{KL}(P_{A=0}(X) || P_{A=1}(X)), \quad \text{JS}(P_{A=0}(X), P_{A=1}(X))
\]
Medem diferença na distribuição de features entre grupos protegidos.

\textbf{Uso Prático}: Métricas pré-treino orientam estratégias de mitigação (resampling, reweighting) \textit{antes} de treinar modelos custosos.

\subsubsection{Métricas Pós-Treinamento (11)}

Analisam bias nas \textit{predições do modelo} após treinamento:

\textbf{(1) Statistical Parity (Demographic Parity)}:
\[
\text{SP} = P(\hat{Y}=1|A=1) - P(\hat{Y}=1|A=0)
\]
Ideal: $|\text{SP}| < 0.1$ (10pp difference).

\textbf{(2) Disparate Impact}:
\[
\text{DI} = \frac{P(\hat{Y}=1|A=1)}{P(\hat{Y}=1|A=0)}
\]
\textbf{Conexão EEOC}: DI < 0.80 viola regra 80\%.

\textbf{(3) Equal Opportunity}:
\[
\text{EO} = P(\hat{Y}=1|Y=1, A=1) - P(\hat{Y}=1|Y=1, A=0)
\]
Iguala True Positive Rates. Ideal: $|\text{EO}| < 0.1$.

\textbf{(4) Equalized Odds}:
\[
\text{EOdds} = \max(|\text{TPR}_{A=1} - \text{TPR}_{A=0}|, |\text{FPR}_{A=1} - \text{FPR}_{A=0}|)
\]
Iguala TPR \textit{e} FPR. Ideal: EOdds < 0.1.

\textbf{(5) FNR Difference}:
\[
\Delta \text{FNR} = \text{FNR}_{A=1} - \text{FNR}_{A=0}
\]
Detecta bias em erros de False Negatives (e.g., negar crédito a candidatos qualificados).

\textbf{(6-7) Conditional Acceptance/Rejection Parity}:
\[
P(Y=1|\hat{Y}=1, A=1) = P(Y=1|\hat{Y}=1, A=0)
\]
Precision parity: entre predições positivas, mesma taxa de verdadeiros positivos.

\textbf{(8-9) Precision/Accuracy Difference}:
\[
\Delta \text{Prec} = \text{Prec}_{A=1} - \text{Prec}_{A=0}, \quad \Delta \text{Acc} = \text{Acc}_{A=1} - \text{Acc}_{A=0}
\]

\textbf{(10) Treatment Equality}:
\[
\text{TE} = \frac{\text{FN}_{A=1}}{\text{FP}_{A=1}} - \frac{\text{FN}_{A=0}}{\text{FP}_{A=0}}
\]
Razão de erros (FN/FP) deve ser igual entre grupos.

\textbf{(11) Entropy Index}:
\[
\text{EI} = \sum_{a \in A} P(A=a) \cdot \text{H}(\hat{Y}|A=a)
\]
Mede heterogeneidade de predições intra-grupo.

\subsection{Módulo de Verificação de Conformidade EEOC}

\subsubsection{Regra 80\% (Disparate Impact)}

Verifica automaticamente se $\text{DI} \geq 0.80$:

\begin{lstlisting}[language=Python, caption=Verificação automática da regra 80\%]
def check_80_rule(y_pred, sensitive_attr):
    groups = sensitive_attr.unique()
    selection_rates = {}

    for group in groups:
        mask = (sensitive_attr == group)
        selection_rates[group] = y_pred[mask].mean()

    reference = max(selection_rates.values())
    violations = {}

    for group, rate in selection_rates.items():
        di = rate / reference
        if di < 0.80:
            violations[group] = {
                'DI': di,
                'selection_rate': rate,
                'reference_rate': reference,
                'shortfall': 0.80 - di
            }

    return {
        'compliant': len(violations) == 0,
        'violations': violations
    }
\end{lstlisting}

\textbf{Relatório Gerado}:
\begin{verbatim}
EEOC 80% Rule Verification:
- Female: DI = 0.72 [VIOLATION] (shortfall: 8pp)
- Male: DI = 1.00 [COMPLIANT]
Recommendation: Adjust threshold or retrain model
\end{verbatim}

\subsubsection{Question 21 (Representação Mínima 2\%)}

EEOC Question 21 estipula que grupos com <2\% de representação não têm validade estatística:

\begin{lstlisting}[language=Python, caption=Verificação Question 21]
def check_question_21(sensitive_attr, min_representation=0.02):
    total = len(sensitive_attr)
    warnings = {}

    for group in sensitive_attr.unique():
        count = (sensitive_attr == group).sum()
        representation = count / total

        if representation < min_representation:
            warnings[group] = {
                'count': count,
                'representation': representation,
                'required': min_representation,
                'warning': 'Insufficient sample size for statistical validity'
            }

    return {
        'valid': len(warnings) == 0,
        'warnings': warnings
    }
\end{lstlisting}

\textbf{Ação Automática}: Grupos com <2\% são excluídos de análise de disparate impact, evitando falsos positivos.

\subsection{Otimização de Threshold}

\subsubsection{Análise de Trade-offs Fairness-Acurácia}

DeepBridge analisa range de thresholds (10-90\%) e computa métricas de fairness e acurácia para cada threshold:

\begin{lstlisting}[language=Python, caption=Otimização de threshold multi-objetivo]
from deepbridge import FairnessTestManager

ftm = FairnessTestManager(dataset)

# Análise de trade-offs em range 0.1-0.9
threshold_analysis = ftm.analyze_thresholds(
    thresholds=np.arange(0.1, 0.9, 0.05),
    fairness_metrics=['disparate_impact', 'equal_opportunity'],
    performance_metrics=['accuracy', 'f1_score']
)

# Pareto frontier: thresholds não dominados
pareto_thresholds = threshold_analysis['pareto_frontier']

# Recomendação baseada em constraints
optimal = ftm.recommend_threshold(
    min_disparate_impact=0.80,
    min_accuracy=0.75,
    objective='maximize_f1'
)
\end{lstlisting}

\subsubsection{Pareto Frontier}

Threshold $t_1$ domina $t_2$ se:
\begin{itemize}
    \item $\text{DI}(t_1) \geq \text{DI}(t_2)$ (melhor fairness)
    \item $\text{Acc}(t_1) \geq \text{Acc}(t_2)$ (melhor acurácia)
    \item Pelo menos uma desigualdade é estrita
\end{itemize}

Pareto frontier contém thresholds não dominados, permitindo stakeholders escolherem trade-off apropriado.

\subsection{Representatividade Estatística}

DeepBridge implementa validações de representatividade para evitar conclusões espúrias:

\textbf{(1) Tamanho Mínimo de Grupo}: Grupos com n < 30 recebem warning (regra de thumb estatística).

\textbf{(2) Intervalos de Confiança}: Métricas reportadas com IC 95\% usando bootstrap:
\begin{lstlisting}[language=Python]
def compute_with_ci(metric_fn, y_true, y_pred, n_bootstrap=1000):
    bootstrap_scores = []
    n = len(y_true)

    for _ in range(n_bootstrap):
        indices = np.random.choice(n, n, replace=True)
        score = metric_fn(y_true[indices], y_pred[indices])
        bootstrap_scores.append(score)

    return {
        'mean': np.mean(bootstrap_scores),
        'ci_lower': np.percentile(bootstrap_scores, 2.5),
        'ci_upper': np.percentile(bootstrap_scores, 97.5)
    }
\end{lstlisting}

\textbf{(3) Testes de Significância}: Diferenças entre grupos testadas via permutation test (p-value < 0.05).

\subsection{Sistema de Visualizações}

DeepBridge gera 6 tipos de visualizações automaticamente:

\textbf{(1) Distribution by Group}: Histogramas de features por grupo protegido

\textbf{(2) Metrics Comparison}: Barplot comparando 15 métricas entre grupos

\textbf{(3) Threshold Impact Analysis}: Curvas mostrando como métricas variam com threshold

\textbf{(4) Confusion Matrices per Group}: Matrizes de confusão lado a lado para cada grupo

\textbf{(5) Fairness Radar Chart}: Radar chart com 11 métricas pós-treino normalizadas

\textbf{(6) Group Performance Comparison}: Boxplots de performance metrics (accuracy, precision, recall, F1) por grupo

\textbf{Formato de Relatórios}:
\begin{itemize}
    \item \textbf{HTML Interativo}: Plotly charts, filtros dinâmicos
    \item \textbf{HTML Estático}: Para auditoria (anexável a emails)
    \item \textbf{PDF}: Formato corporativo com branding customizável
    \item \textbf{JSON}: Para integração programática
\end{itemize}

\subsection{Integração com Pipeline de Validação DeepBridge}

FairnessTestManager integra-se com Experiment orchestrator do DeepBridge:

\begin{lstlisting}[language=Python, caption=Integração com pipeline completo]
from deepbridge import DBDataset, Experiment

dataset = DBDataset(df, target='approved', model=model)

# Validação multi-dimensional (fairness + robustness + uncertainty)
exp = Experiment(
    dataset=dataset,
    tests=['fairness', 'robustness', 'uncertainty']
)

results = exp.run_tests()

# Relatório unificado com todas dimensões
exp.save_pdf('complete_validation_report.pdf')
\end{lstlisting}

\textbf{Benefícios da Integração}:
\begin{itemize}
    \item \textbf{Consistência}: Mesmo DBDataset usado em fairness, robustness, uncertainty
    \item \textbf{Eficiência}: Predições do modelo computadas uma vez e reutilizadas
    \item \textbf{Relatórios Unificados}: Stakeholders veem fairness no contexto de outras dimensões de validação
\end{itemize}

\section{Implementacao}

\subsection{Arquitetura}

Implementamos o framework em Python 3.9+ com integracao ao DeepBridge. Estrutura modular:

\begin{verbatim}
deepbridge/fairness/
├── threshold_optimizer.py    # Classe principal
├── metrics/
│   ├── fairness_metrics.py   # Metricas de justica
│   └── accuracy_metrics.py   # Metricas de acuracia
├── optimization/
│   │── nsga2.py              # Implementacao NSGA-II
│   └── pareto.py             # Analise de dominancia
└── visualization/
    ├── trade_off_plots.py    # Graficos 2D
    └── interactive_dash.py   # Dashboard interativo
\end{verbatim}

\subsection{Threshold Optimizer}

Classe principal que orquestra analise:

\begin{lstlisting}[language=Python]
class ThresholdOptimizer:
    def __init__(self,
                 model,
                 X, y, sensitive_attr,
                 thresholds=np.arange(0.1, 0.95, 0.05)):
        self.model = model
        self.X = X
        self.y = y
        self.sensitive_attr = sensitive_attr
        self.thresholds = thresholds
        self.results = []

    def optimize(self):
        # 1. Obter predicoes probabilisticas
        probs = self.model.predict_proba(self.X)[:, 1]

        # 2. Varrer limiares
        for thresh in self.thresholds:
            metrics = self._compute_metrics(
                probs, thresh
            )
            self.results.append({
                'threshold': thresh,
                **metrics
            })

        # 3. Identificar Pareto frontier
        self.pareto_front = self._find_pareto()

        return self.pareto_front
\end{lstlisting}

\subsection{Fairness Metrics}

Implementacao eficiente de metricas de justica:

\begin{lstlisting}[language=Python]
class FairnessMetrics:
    @staticmethod
    def demographic_parity_diff(y_pred, sensitive):
        groups = np.unique(sensitive)
        rates = [
            y_pred[sensitive == g].mean()
            for g in groups
        ]
        return abs(rates[0] - rates[1])

    @staticmethod
    def equalized_odds_diff(y_true, y_pred, sensitive):
        groups = np.unique(sensitive)

        # TPR por grupo
        tprs = [
            recall_score(
                y_true[sensitive == g],
                y_pred[sensitive == g]
            ) for g in groups
        ]

        # FPR por grupo
        fprs = [
            FairnessMetrics._fpr(
                y_true[sensitive == g],
                y_pred[sensitive == g]
            ) for g in groups
        ]

        return max(
            abs(tprs[0] - tprs[1]),
            abs(fprs[0] - fprs[1])
        )
\end{lstlisting}

\subsection{NSGA-II Implementation}

Adaptacao de NSGA-II para selecao de limiares:

\begin{lstlisting}[language=Python]
class NSGA2Optimizer:
    def non_dominated_sort(self, solutions, objectives):
        """Classifica solucoes em fronteiras"""
        fronts = [[]]
        domination_count = [0] * len(solutions)
        dominated_solutions = [[] for _ in solutions]

        # Comparar todos pares
        for i, sol_i in enumerate(solutions):
            for j, sol_j in enumerate(solutions):
                if self._dominates(
                    objectives[i],
                    objectives[j]
                ):
                    dominated_solutions[i].append(j)
                elif self._dominates(
                    objectives[j],
                    objectives[i]
                ):
                    domination_count[i] += 1

            if domination_count[i] == 0:
                fronts[0].append(i)

        return fronts[0]  # Retorna primeira fronteira

    def _dominates(self, obj1, obj2):
        """Verifica se obj1 domina obj2"""
        better_in_any = False
        for o1, o2 in zip(obj1, obj2):
            if o1 > o2:  # Pior em algum objetivo
                return False
            if o1 < o2:  # Melhor em algum objetivo
                better_in_any = True
        return better_in_any
\end{lstlisting}

\subsection{Visualization}

Geracao de graficos de trade-off:

\begin{lstlisting}[language=Python]
class TradeOffPlotter:
    def plot_pareto_frontier(self, results, pareto_indices):
        fig, ax = plt.subplots(figsize=(10, 6))

        # Plotar todos pontos
        ax.scatter(
            results['f1_score'],
            results['demographic_parity_diff'],
            alpha=0.3, label='Todos limiares'
        )

        # Destacar Pareto frontier
        pareto_data = results.iloc[pareto_indices]
        ax.scatter(
            pareto_data['f1_score'],
            pareto_data['demographic_parity_diff'],
            color='red', s=100,
            marker='*', label='Pareto-otimos'
        )

        ax.set_xlabel('F1-Score (acuracia)')
        ax.set_ylabel('Demographic Parity Diff (injustica)')
        ax.legend()
        return fig
\end{lstlisting}

\subsection{Integracao com DeepBridge}

Adicionamos teste de otimizacao de limiar ao framework:

\begin{lstlisting}[language=Python]
# Em deepbridge/tests/fairness_tests.py
class ThresholdOptimizationTest(ValidationTest):
    def run(self):
        optimizer = ThresholdOptimizer(
            model=self.model,
            X=self.X_test,
            y=self.y_test,
            sensitive_attr=self.sensitive_attr
        )

        pareto_front = optimizer.optimize()

        # Gerar relatorio
        self.report = {
            'num_pareto_solutions': len(pareto_front),
            'best_fairness_threshold': ...,
            'best_accuracy_threshold': ...,
            'plots': optimizer.visualize()
        }
\end{lstlisting}

\subsection{Otimizacoes de Performance}

\begin{itemize}
    \item \textbf{Vetorizacao}: Uso de NumPy para calculo paralelo de metricas
    \item \textbf{Caching}: Memoizacao de resultados de metricas para evitar recomputacao
    \item \textbf{Early stopping}: Interrompe analise se nenhum limiar melhora fronteira Pareto
\end{itemize}

\section{Estudos de Validação}
\label{sec:validation}

Avaliamos DeepBridge em quatro dimensões: (1) cobertura de testes, (2) performance, (3) estudos de caso, e (4) usabilidade.

\subsection{Cobertura de Testes}

Comparamos DeepBridge com ferramentas especializadas:

\begin{table}[h]
\centering
\caption{Cobertura de testes: DeepBridge vs. ferramentas especializadas}
\label{tab:coverage}
\small
\begin{tabular}{@{}lcccccc@{}}
\toprule
 & \textbf{Alibi} & \textbf{AIF360} & \textbf{Fair learn} & \textbf{UQ360} & \textbf{Evid.} & \textbf{DeepB.} \\
\midrule
Robustness & \cmark & \xmark & \xmark & \xmark & \xmark & \cmark \\
Fairness & \xmark & \cmark & \cmark & \xmark & \xmark & \cmark \\
Uncertainty & \xmark & \xmark & \xmark & \cmark & \xmark & \cmark \\
Drift & \cmark & \xmark & \xmark & \xmark & \cmark & \cmark \\
Hyperparams & \xmark & \xmark & \xmark & \xmark & \xmark & \cmark \\
\midrule
\textbf{Total} & 2 & 1 & 1 & 1 & 1 & \textbf{5} \\
\bottomrule
\end{tabular}
\end{table}

\textbf{Resultado}: DeepBridge cobre 5/5 dimensões vs. média 1.2 das ferramentas.

\subsection{Performance Benchmarks}

Comparamos tempo de execução em 3 tamanhos de dataset:

\begin{table}[h]
\centering
\caption{Tempo de validação completa (minutos)}
\label{tab:performance}
\small
\begin{tabular}{@{}lcccc@{}}
\toprule
\textbf{Abordagem} & \textbf{1K} & \textbf{50K} & \textbf{500K} & \textbf{Speedup} \\
\midrule
Manual (5 tools) & 45 & 152 & 420 & 1.0x \\
\textbf{DeepBridge} & \textbf{5} & \textbf{17} & \textbf{68} & \textbf{6.2x} \\
\bottomrule
\end{tabular}
\end{table}

\textbf{Breakdown do speedup}:
\begin{itemize}
    \item Lazy loading: 30-50s economia
    \item Execução paralela: 40\% redução
    \item Caching: 20\% redução
    \item Sem conversões de formato: 15\% redução
\end{itemize}

\subsection{Estudos de Caso}

\subsubsection{Case 1: Credit Scoring}

\textbf{Dataset}: German Credit (1K amostras)

\textbf{Problemas detectados}:
\begin{itemize}
    \item \textbf{Fairness}: DI = 0.73 (viola regra 80\%) para idade <25
    \item \textbf{Robustness}: Acurácia cai 12pp sob perturbação $\sigma=0.1$
    \item \textbf{Uncertainty}: ECE = 0.18 (mal calibrado)
    \item \textbf{Drift}: Covariate shift detectado (KS p-value < 0.001)
\end{itemize}

\textbf{Tempo DeepBridge}: 4.2 min vs. 38 min manual

\subsubsection{Case 2: Healthcare Risk}

\textbf{Dataset}: 10K pacientes, predição readmissão 30 dias

\textbf{Problemas detectados}:
\begin{itemize}
    \item \textbf{Fairness}: FNR difference = 0.18 (Hispanic vs. White)
    \item \textbf{Robustness}: Weak spots em ``dias\_internação'' e ``comorbidades''
    \item \textbf{Uncertainty}: Cobertura conformal 87\% (alvo 90\%)
\end{itemize}

\textbf{Tempo DeepBridge}: 11.3 min vs. 95 min manual

\subsubsection{Case 3: E-commerce Recommendations}

\textbf{Dataset}: 100K interações usuário-produto

\textbf{Problemas detectados}:
\begin{itemize}
    \item \textbf{Robustness}: Performance cai 22\% sob feature dropout
    \item \textbf{Drift}: Concept drift (mudança sazonal)
    \item \textbf{Hyperparameters}: Instável para \texttt{max\_depth} > 10
\end{itemize}

\textbf{Tempo DeepBridge}: 23.7 min vs. 180 min manual

\subsubsection{Case 4: Fraud Detection}

\textbf{Dataset}: 500K transações

\textbf{Problemas detectados}:
\begin{itemize}
    \item \textbf{Robustness}: Vulnerável a quantile perturbations (P95)
    \item \textbf{Uncertainty}: Overconfident em predições fraudulentas
    \item \textbf{Drift}: Prior shift (mudança na taxa de fraude)
    \item \textbf{Fairness}: Statistical parity violation para merchant\_category
\end{itemize}

\textbf{Tempo DeepBridge}: 67.9 min vs. 420 min manual

\subsection{Síntese dos Casos}

\begin{table}[h]
\centering
\caption{Resumo dos estudos de caso}
\label{tab:case_summary}
\small
\begin{tabular}{@{}lcccc@{}}
\toprule
\textbf{Métrica} & \textbf{Credit} & \textbf{Health} & \textbf{E-comm} & \textbf{Fraud} \\
\midrule
Problemas detectados & 4 & 3 & 3 & 4 \\
Tempo DeepBridge (min) & 4.2 & 11.3 & 23.7 & 67.9 \\
Tempo manual (min) & 38 & 95 & 180 & 420 \\
Speedup & 9.0x & 8.4x & 7.6x & 6.2x \\
\bottomrule
\end{tabular}
\end{table}

\subsection{Estudo de Usabilidade}

\textbf{Metodologia}: 20 participantes (ML engineers, data scientists)

\textbf{Tarefas}:
\begin{enumerate}
    \item Setup: Instalar e configurar (10 min)
    \item Task 1: Validação completa de modelo (20 min)
    \item Task 2: Interpretar relatório e priorizar issues (15 min)
    \item Task 3: Comparar 2 modelos (15 min)
\end{enumerate}

\textbf{Resultados}:
\begin{itemize}
    \item \textbf{SUS Score}: 87.5 (top 10\%, "excelente")
    \item \textbf{Taxa de sucesso}: 95\% (19/20 completaram todas tarefas)
    \item \textbf{Time-to-first-validation}: 12.3 min vs. 45 min manual
    \item \textbf{NASA-TLX}: 28/100 (baixa carga cognitiva)
\end{itemize}

\textbf{Feedback qualitativo}:
\begin{itemize}
    \item "API unificada eliminou 90\% do código glue" (P7)
    \item "Primeira ferramenta que testa tudo em um lugar" (P12)
    \item "Relatórios prontos para mostrar para stakeholders" (P18)
\end{itemize}

\section{Discussao}

\subsection{Principais Descobertas}

\subsubsection{Reducao de Complexidade via Encapsulamento}

DBDataset demonstra que \textbf{encapsulamento disciplinado} de elementos de validacao reduz drasticamente complexidade de codigo (75.7\% em media). Esta reducao nao e apenas quantitativa---elimina classes inteiras de erros:

\begin{itemize}
    \item \textbf{Mismatches de features}: Passar features categoricas para algoritmos que esperam numericas
    \item \textbf{Inconsistencias de split}: Usar random\_state diferente em diferentes etapas
    \item \textbf{Esquecimento de features}: Omitir features ao configurar validation suites
    \item \textbf{Erros de indexacao}: Confundir indices de train/test em analises
\end{itemize}

User study confirma: reducao de 85.7\% em erros de configuracao.

\subsubsection{Inferencia Automatica com 100\% de Acuracia}

Algoritmo de inferencia baseado em tipo + cardinalidade alcanca 100\% de acuracia em 387 features testadas. Fatores criticos:

\begin{enumerate}
    \item \textbf{Heuristica de dtype}: Features \texttt{object}/\texttt{category} sao inequivocamente categoricas em contexto tabular
    \item \textbf{Cardinalidade como fallback}: Permite capturar categoricas codificadas como inteiros (e.g., dias da semana como 0-6)
    \item \textbf{Override manual}: Escape hatch para casos ambiguos (IDs, ZIP codes)
\end{enumerate}

Casos onde inferencia falha: features ordinais codificadas como inteiros (e.g., \texttt{education\_level} = 1, 2, 3). Solucao: override manual ou \texttt{max\_categories}.

\subsubsection{Trade-off Memoria vs. Corretude}

DBDataset copia dados (2x memoria) para garantir imutabilidade. Em workflow de validacao offline, este trade-off e justificado:

\begin{itemize}
    \item \textbf{Validacao e processo batch}: Memoria disponivel, tempo de execucao nao-critico
    \item \textbf{Bugs de mutacao sao sutis}: Modificar DataFrame original pode causar erros dificeis de debugar
    \item \textbf{Reproducibilidade requer imutabilidade}: Copias garantem que re-execucao produz mesmos resultados
\end{itemize}

Para datasets gigantes (>10GB), DBDataset poderia oferecer modo \texttt{copy=False} (caveat emptor).

\subsection{Implicacoes Praticas}

\subsubsection{Para Praticantes de ML}

\paragraph{Reducao de Boilerplate} DBDataset elimina codigo repetitivo de preparacao de dados, permitindo foco em analise de resultados.

\paragraph{Onboarding Facilitado} Novos membros de equipe aprendem interface unica, nao multiplas convencoes de diferentes suites.

\paragraph{Menos Debugging} Validacao centralizada previne erros de configuracao que consomem horas de debugging.

\subsubsection{Para MLOps}

\paragraph{Integracao CI/CD Simplificada} Container unificado facilita passagem de dados entre stages de pipeline:

\begin{lstlisting}[language=Python, basicstyle=\ttfamily\scriptsize]
# Stage 1: Preparacao
dataset = DBDataset(data=df, target_column='y', model=model)
dataset.save('dataset.pkl')

# Stage 2: Validacao (processo separado)
dataset = DBDataset.load('dataset.pkl')
results = RobustnessSuite(dataset).run()
\end{lstlisting}

\paragraph{Reproducibilidade em Producao} Random states encapsulados garantem que validacao em desenvolvimento corresponde a validacao em staging/producao.

\subsubsection{Para Pesquisadores}

\paragraph{Comparacao de Abordagens} Interface padronizada permite comparar diferentes validation suites sem reescrever codigo de preparacao.

\paragraph{Extensao de Validation Suites} Novos metodos de validacao podem assumir DBDataset como input, reduzindo barreira de entrada.

\subsection{Limitacoes}

\subsubsection{Limitacao 1: Overhead de Memoria}

\textbf{Descricao}: Copias de dados consomem 2x memoria.

\textbf{Impacto}: Datasets >10GB podem exceder memoria disponivel.

\textbf{Mitigacao}: Implementar modo \texttt{copy=False} com warnings explicitos, ou usar Dask/Vaex para datasets out-of-core.

\subsubsection{Limitacao 2: Inferencia de Ordinais}

\textbf{Descricao}: Features ordinais codificadas como inteiros podem ser incorretamente classificadas como numericas.

\textbf{Exemplo}: \texttt{education\_level} = 1 (primario), 2 (secundario), 3 (superior).

\textbf{Impacto}: Algoritmos podem tratar ordinal como continuo (assumindo que 2 esta "entre" 1 e 3 numericamente).

\textbf{Mitigacao}: (1) Override manual via \texttt{categorical\_features}, (2) Adicionar parametro \texttt{ordinal\_features} em versoes futuras.

\subsubsection{Limitacao 3: Dados Nao-Tabulares}

\textbf{Descricao}: DBDataset otimizado para dados tabulares (CSV, DataFrames).

\textbf{Impacto}: Nao suporta nativamente imagens, texto, grafos, series temporais.

\textbf{Justificativa}: Validation suites do DeepBridge focam em modelos tabulares. Para outros dominios, abstraccoes diferentes sao mais apropriadas (e.g., \texttt{TorchVision.datasets} para imagens).

\subsubsection{Limitacao 4: Acoplamento com pandas}

\textbf{Descricao}: DBDataset usa pandas DataFrames internamente.

\textbf{Impacto}: Performance subotima para datasets gigantes comparado a Polars, Dask, Vaex.

\textbf{Mitigacao}: Futuras versoes podem suportar backends alternativos via protocolo (e.g., \texttt{\_\_dataframe\_\_}).

\subsection{Generalizabilidade}

\subsubsection{Aplicabilidade a Outros Dominios}

Container pattern de DBDataset pode ser adaptado para:

\begin{itemize}
    \item \textbf{NLP}: Encapsular texto, embeddings, labels, modelos de linguagem
    \item \textbf{Computer Vision}: Encapsular imagens, bounding boxes, segmentations, modelos
    \item \textbf{Time Series}: Encapsular series, lags, exogenous variables, forecasters
    \item \textbf{Grafos}: Encapsular nodes, edges, features, GNNs
\end{itemize}

Principios transferiveis: encapsulamento, inferencia automatica, integracao com validation tools.

\subsubsection{Extensoes para Casos de Uso Especializados}

DBDataset pode ser extendido para contextos especificos:

\paragraph{Federated Learning} Adicionar metodos para particionar dados por clientes:

\begin{lstlisting}[language=Python, basicstyle=\ttfamily\scriptsize]
datasets_by_client = dataset.partition_by('client_id', n_clients=10)
\end{lstlisting}

\paragraph{Active Learning} Suportar marcacao incremental de amostras:

\begin{lstlisting}[language=Python, basicstyle=\ttfamily\scriptsize]
unlabeled_dataset = dataset.get_unlabeled()
newly_labeled = oracle.label(unlabeled_dataset.sample(100))
dataset.add_labels(newly_labeled)
\end{lstlisting}

\paragraph{Multi-task Learning} Encapsular multiplos targets:

\begin{lstlisting}[language=Python, basicstyle=\ttfamily\scriptsize]
dataset = DBDataset(
    data=df,
    target_columns=['task1', 'task2', 'task3']  # Multi-target
)
\end{lstlisting}

\subsection{Relacao com Trabalhos Futuros}

\subsubsection{Integracao com MLflow}

DBDataset poderia ser logado como artifact no MLflow:

\begin{lstlisting}[language=Python, basicstyle=\ttfamily\scriptsize]
import mlflow

with mlflow.start_run():
    mlflow.log_artifact(dataset.save('dataset.pkl'))
    mlflow.log_params(dataset.get_metadata())  # Random state, split ratio
\end{lstlisting}

\subsubsection{Suporte a Data Versioning (DVC)}

Integracao com DVC para versionamento de datasets:

\begin{lstlisting}[language=Python, basicstyle=\ttfamily\scriptsize]
dataset.save_with_dvc('dataset.pkl')  # Auto-adiciona ao .dvc
\end{lstlisting}

\subsubsection{Schema Validation}

Adicionar validacao de schema para garantir consistencia:

\begin{lstlisting}[language=Python, basicstyle=\ttfamily\scriptsize]
schema = DatasetSchema(
    features={'age': int, 'income': float, 'gender': str},
    target='approved',
    constraints={'age': lambda x: x >= 0}
)

dataset = DBDataset(data=df, schema=schema)  # Valida na criacao
\end{lstlisting}

\subsection{Licoes Aprendidas}

\subsubsection{Design Iterativo}

DBDataset evoluiu atraves de 5+ iteracoes com feedback de usuarios:

\begin{enumerate}
    \item \textbf{v1}: Container simples sem inferencia (usuarios reclamaram de configuracao manual)
    \item \textbf{v2}: Inferencia baseada apenas em dtype (falhou em IDs numericos)
    \item \textbf{v3}: Adicao de cardinalidade + override manual (balance ideal)
    \item \textbf{v4}: Suporte a Bunch e modelos pre-treinados (requisito de usuarios)
    \item \textbf{v5}: Factory methods para workflows especializados (feedback de MLOps)
\end{enumerate}

\subsubsection{Importancia de Defaults Sensatos}

Parametros default (test\_size=0.2, stratify=False) escolhidos baseados em survey de 50+ projetos ML open-source. Defaults ruins aumentam friccao de adocao.

\subsubsection{Documentacao e Exemplos}

User study revelou que exemplos concretos (case studies) foram mais efetivos que documentacao de API para onboarding. Investir em tutoriais praticos e essencial.

\subsection{Consideracoes Eticas}

\subsubsection{Facilitacao de Validacao de Fairness}

DBDataset reduz barreira tecnica para executar testes de fairness, potencialmente aumentando adocao de validacao de bias em sistemas ML. Impacto social positivo: modelos mais justos em producao.

\subsubsection{Risco de Over-reliance em Automacao}

Inferencia automatica pode criar falsa sensacao de seguranca---usuarios podem nao validar se features categoricas foram corretamente identificadas. Mitigacao: logs informativos e metodos de inspecao (\texttt{dataset.inspect\_features()}).

\subsection{Recomendacoes para Adocao}

\subsubsection{Para Equipes Iniciando Validacao}

\begin{enumerate}
    \item Iniciar com workflow simples (unified data + auto-split)
    \item Validar inferencia de features manualmente em primeiros usos
    \item Integrar gradualmente em pipeline CI/CD
\end{enumerate}

\subsubsection{Para Equipes com Pipelines Existentes}

\begin{enumerate}
    \item Criar adapters para converter codigo existente para DBDataset
    \item Executar validacao paralela (pipeline antigo + DBDataset) durante transicao
    \item Migrar validation suite por vez (comecando com mais simples)
\end{enumerate}

\subsubsection{Para Organizacoes Enterprise}

\begin{enumerate}
    \item Adicionar DBDataset a template de projetos ML
    \item Treinar equipes em workshop hands-on (2-4 horas)
    \item Estabelecer DBDataset como padrao em code review guidelines
\end{enumerate}

\section{Conclusao}

\subsection{Sintese de Contribuicoes}

Apresentamos \textbf{DBDataset}, um container de dados unificado que simplifica validacao de modelos ML atraves de encapsulamento disciplinado e inferencia automatica de features. Nossas principais contribuicoes:

\begin{enumerate}
    \item \textbf{Container Pattern}: Primeira solucao que unifica dados, features, modelos, e predicoes em interface coesa para validacao
    \item \textbf{Inferencia Automatica}: Algoritmo baseado em tipo + cardinalidade com 100\% de acuracia em 387 features testadas
    \item \textbf{Flexibilidade de Workflows}: Suporte a 4 modos de inicializacao cobrindo casos de uso desde prototipagem ate producao
    \item \textbf{Integracao Seamless}: Interface padronizada para 6 validation suites (robustness, uncertainty, fairness, resilience, hyperparameter, distillation)
    \item \textbf{Validacao Empirica}: Case studies demonstrando reducao de 75.7\% em codigo e 85.7\% em erros de configuracao
\end{enumerate}

\subsection{Impacto Esperado}

\subsubsection{Comunidade de Praticantes}

DBDataset reduz barreiras tecnicas para validacao rigorosa de modelos ML. Reducao de 62.8\% em tempo de setup (user study) permite que equipes adotem validacao abrangente sem overhead proibitivo.

\textbf{Projecao de impacto}: Se 10\% de projetos ML em producao adotarem validacao rigorosa devido a DBDataset, estimamos prevenção de centenas de falhas de modelos em dominios criticos (saude, financas, contratacao).

\subsubsection{Pesquisa Academica}

Interface padronizada facilita comparacao entre metodos de validacao. Pesquisadores podem publicar novos testes de robustness/fairness assumindo DBDataset como input, acelerando inovacao em ML trustworthy.

\subsubsection{Industria e MLOps}

Container unificado simplifica integracao de validacao em pipelines CI/CD. Organizacoes podem estabelecer DBDataset como padrao interno, reduzindo heterogeneidade de codigo e facilitando onboarding.

\subsection{Trabalhos Futuros}

\subsubsection{Curto Prazo (6-12 meses)}

\paragraph{Suporte a Dados Ordinais} Adicionar parametro \texttt{ordinal\_features} com especificacao de ordem:

\begin{lstlisting}[language=Python, basicstyle=\ttfamily\scriptsize]
dataset = DBDataset(
    data=df,
    target_column='y',
    ordinal_features={
        'education': ['primary', 'secondary', 'higher'],
        'satisfaction': [1, 2, 3, 4, 5]
    }
)
\end{lstlisting}

\paragraph{Modo Copy-on-Write} Reduzir overhead de memoria para datasets gigantes:

\begin{lstlisting}[language=Python, basicstyle=\ttfamily\scriptsize]
dataset = DBDataset(data=df, target_column='y', copy=False)
# Warning: Modifications to df will affect dataset
\end{lstlisting}

\paragraph{Schema Validation} Integracao com Pydantic ou Pandera para validacao de tipos e constraints:

\begin{lstlisting}[language=Python, basicstyle=\ttfamily\scriptsize]
from deepbridge.schemas import DatasetSchema

schema = DatasetSchema.from_yaml('schema.yaml')
dataset = DBDataset(data=df, schema=schema)
\end{lstlisting}

\subsubsection{Medio Prazo (1-2 anos)}

\paragraph{Backends Alternativos} Suporte a Polars, Dask, Vaex para datasets out-of-core:

\begin{lstlisting}[language=Python, basicstyle=\ttfamily\scriptsize]
dataset = DBDataset(
    data=dask_df,
    target_column='y',
    backend='dask'  # Auto-detecta ou especificado
)
\end{lstlisting}

\paragraph{Feature Stores Integration} Integracao com Feast, Tecton para carregar features de producao:

\begin{lstlisting}[language=Python, basicstyle=\ttfamily\scriptsize]
from deepbridge.integrations import FeastConnector

connector = FeastConnector(feature_store_url='...')
dataset = connector.create_dataset(
    entity_df=entities,
    features=['feature1', 'feature2'],
    target_column='y'
)
\end{lstlisting}

\paragraph{Time Series Support} Extensao para dados temporais com lags automaticos:

\begin{lstlisting}[language=Python, basicstyle=\ttfamily\scriptsize]
from deepbridge import TimeSeriesDataset

ts_dataset = TimeSeriesDataset(
    data=df,
    target_column='sales',
    datetime_column='date',
    lags=[1, 7, 30],  # Auto-gera features de lag
    rolling_windows=[7, 30]  # Auto-gera rolling means
)
\end{lstlisting}

\subsubsection{Longo Prazo (2+ anos)}

\paragraph{Multi-modal Datasets} Suporte a combinacao de tabular + imagens + texto:

\begin{lstlisting}[language=Python, basicstyle=\ttfamily\scriptsize]
from deepbridge import MultiModalDataset

mm_dataset = MultiModalDataset(
    tabular_data=df,
    image_column='product_image',  # Paths para imagens
    text_column='description',
    target_column='category'
)
\end{lstlisting}

\paragraph{AutoML Integration} DBDataset como input nativo para frameworks AutoML:

\begin{lstlisting}[language=Python, basicstyle=\ttfamily\scriptsize]
from autosklearn import AutoSklearnClassifier

automl = AutoSklearnClassifier()
automl.fit(dataset)  # Aceita DBDataset diretamente
\end{lstlisting}

\paragraph{Differential Privacy} Suporte a private data splits:

\begin{lstlisting}[language=Python, basicstyle=\ttfamily\scriptsize]
dataset = DBDataset(
    data=df,
    target_column='y',
    privacy_budget=1.0,  # Epsilon para DP
    add_noise=True
)
\end{lstlisting}

\subsection{Chamada para Comunidade}

DBDataset e open-source (licenca MIT) e desenvolvido publicamente:

\begin{itemize}
    \item \textbf{Codigo}: \texttt{github.com/deepbridge/deepbridge}
    \item \textbf{Documentacao}: \texttt{deepbridge.readthedocs.io}
    \item \textbf{Issues}: \texttt{github.com/deepbridge/deepbridge/issues}
\end{itemize}

Convidamos comunidade ML para:

\begin{enumerate}
    \item \textbf{Contribuir}: Adicionar novos modos de inicializacao, backends, integrações
    \item \textbf{Reportar bugs}: Casos onde inferencia falha ou design e inadequado
    \item \textbf{Propor extensoes}: Features para casos de uso nao cobertos
    \item \textbf{Compartilhar experiencias}: Case studies em dominios nao testados
\end{enumerate}

\subsection{Mensagem Final}

Validacao rigorosa de modelos ML nao deve ser privilégio de equipes com recursos abundantes. DBDataset democratiza validacao ao reduzir complexidade tecnica e overhead de configuracao. Nossa visao: fazer validacao abrangente (robustness, uncertainty, fairness) tao trivial quanto treinar modelo com \texttt{model.fit()}.

Fragmentacao de gestao de dados em validacao ML e problema solucionavel. Container pattern com inferencia automatica demonstra que \textbf{simplicidade e rigor nao sao mutuamente exclusivos}---ambos podem ser alcançados atraves de design cuidadoso e encapsulamento disciplinado.

DBDataset e passo inicial em direcao a ecosistema ML onde validacao e parte natural do workflow, nao tarefa opcional relegada a pos-deployment. Acreditamos que futuro de ML responsavel depende de ferramentas que tornem praticas corretas mais faceis que praticas inadequadas.

\subsection{Disponibilidade}

\begin{itemize}
    \item \textbf{Codigo-fonte}: MIT License, disponivel em \texttt{github.com/deepbridge/deepbridge}
    \item \textbf{Datasets}: Case studies reproducibles em \texttt{github.com/deepbridge/dbdataset-paper}
    \item \textbf{Artefatos}: Modelos treinados, resultados experimentais em Zenodo (DOI: [a definir])
    \item \textbf{Documentacao}: Tutoriais e exemplos em \texttt{deepbridge.readthedocs.io}
\end{itemize}

\subsection{Agradecimentos}

Agradecemos aos 15 participantes do user study por feedback valioso, aos revisores anonimos por sugestoes construtivas, e a comunidade open-source Python (pandas, scikit-learn, NumPy) cujas ferramentas fundamentam DBDataset.

Financiamento: [A definir]

\subsection{Consideracoes Finais}

DBDataset representa mudanca de paradigma em como dados sao gerenciados para validacao de modelos ML---de objetos fragmentados para container unificado, de configuracao manual para inferencia automatica, de codigo ad-hoc para interface padronizada. Esperamos que este trabalho inspire desenvolvimento de ferramentas similares em outros dominios ML e contribua para ecosistema mais maduro de validacao de modelos.

\textit{Machine Learning e muito mais que treinar modelos---e validar rigorosamente que eles funcionam como esperado. DBDataset torna esta validacao simples, reproduzivel, e acessivel.}


\bibliographystyle{plain}
\bibliography{bibliography/references}

\end{document}

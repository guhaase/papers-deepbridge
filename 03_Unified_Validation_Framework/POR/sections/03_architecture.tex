\section{Arquitetura do DeepBridge}
\label{sec:architecture}

A arquitetura do DeepBridge (Figura~\ref{fig:architecture}) é organizada em três camadas: (1) \textbf{Abstração de Dados} (DBDataset), (2) \textbf{Validação Multi-Dimensional} (5 suítes + Experiment orchestrator), e (3) \textbf{Relatórios \& Integração}.

\subsection{DBDataset: Container Unificado}

DBDataset elimina fragmentação de APIs encapsulando dados, modelo e metadados em um único container reutilizável.

\begin{lstlisting}[language=Python, caption={DBDataset: Create Once\, Validate Anywhere}]
from deepbridge import DBDataset

dataset = DBDataset(
    data=df,                    # Pandas/Dask DataFrame
    target_column='approved',   # Target
    model=trained_model,        # Scikit-learn/XGBoost/custom
    protected_attributes=['gender', 'race']  # Para fairness
)

# Auto-inferência
print(dataset.task_type)        # 'binary_classification'
print(dataset.feature_types)    # {'age': 'continuous', ...}
print(dataset.model_type)       # 'tree_ensemble'
\end{lstlisting}

\textbf{Funcionalidades}:
\begin{itemize}
    \item \textbf{Auto-inferência}: Task type, feature types, model type
    \item \textbf{Lazy evaluation}: Predições computadas sob demanda
    \item \textbf{Caching}: Evita recomputação
    \item \textbf{Validação}: Checks de consistência automáticos
\end{itemize}

\subsection{Experiment: Orquestrador}

Coordena validação multi-dimensional com configuração unificada.

\begin{lstlisting}[language=Python, caption=Orquestração de testes]
from deepbridge import Experiment

exp = Experiment(
    dataset=dataset,
    tests=['robustness', 'uncertainty', 'fairness'],
    config='medium'  # ou 'quick', 'full'
)

# Execução paralela
results = exp.run_tests()

# Resultados estruturados
print(results.robustness.gaussian_perturbation)
print(results.fairness.disparate_impact)
\end{lstlisting}

\textbf{Features}:
\begin{itemize}
    \item \textbf{Lazy loading}: Modelos carregados sob demanda (-42\% memória)
    \item \textbf{Execução paralela}: Testes independentes em paralelo
    \item \textbf{Progress tracking}: Barra de progresso em tempo real
    \item \textbf{Error handling}: Falhas isoladas não abortam experimento
\end{itemize}

\subsection{5 Suítes de Validação}

\subsubsection{RobustnessTestManager}

Testa robustez a perturbações:
\begin{itemize}
    \item Gaussian perturbation (5 níveis de $\sigma$)
    \item Quantile perturbation (p5, p25, p50, p75, p95)
    \item Weak spot detection (features mais sensíveis)
\end{itemize}

\subsubsection{UncertaintyTestManager}

Quantifica incerteza:
\begin{itemize}
    \item Calibration (ECE, MCE, Brier score)
    \item Conformal prediction (cobertura 90\%, 95\%, 99\%)
    \item Prediction intervals
\end{itemize}

\subsubsection{ResilienceTestManager}

Detecta drift:
\begin{itemize}
    \item 5 tipos de drift (covariate, concept, prior, posterior, joint)
    \item Testes estatísticos (KS, Chi-squared, Wasserstein)
    \item Drift magnitude e p-values
\end{itemize}

\subsubsection{FairnessTestManager}

Avalia fairness (detalhado no Paper 2):
\begin{itemize}
    \item 15 métricas pré/pós-treinamento
    \item Verificação EEOC/ECOA
    \item Threshold optimization
\end{itemize}

\subsubsection{HyperparameterTestManager}

Analisa sensibilidade:
\begin{itemize}
    \item Cross-validation com variação de hiperparâmetros
    \item Importance scores via permutation
    \item Stability analysis
\end{itemize}

\subsection{Sistema de Configuração}

Presets padronizados balanceiam tempo vs. cobertura:

\begin{table}[h]
\centering
\caption{Presets de configuração}
\label{tab:config_presets}
\small
\begin{tabular}{@{}lccc@{}}
\toprule
\textbf{Preset} & \textbf{Tempo} & \textbf{Testes} & \textbf{Uso} \\
\midrule
Quick & 2-5 min & Amostragem 10\% & CI/CD rápido \\
Medium & 10-20 min & Amostragem 50\% & Desenvolvimento \\
Full & 30-60 min & Dados completos & Pré-deployment \\
\bottomrule
\end{tabular}
\end{table}

\subsection{Sistema de Relatórios}

Gera relatórios em múltiplos formatos:

\begin{lstlisting}[language=Python]
# HTML interativo (Plotly)
exp.save_html('all', 'report.html')

# PDF (via LaTeX)
exp.save_pdf('all', 'report.pdf')

# JSON (programático)
exp.save_json('all', 'results.json')
\end{lstlisting}

\textbf{Relatórios incluem}:
\begin{itemize}
    \item Dashboard comparativo (5 dimensões)
    \item Drill-down por dimensão
    \item Priorização de issues (severidade)
    \item Recomendações de mitigação
\end{itemize}

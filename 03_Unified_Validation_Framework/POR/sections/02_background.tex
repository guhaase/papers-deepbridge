\section{Background e Trabalhos Relacionados}
\label{sec:background}

Esta seção revisa as dimensões de validação de modelos ML, ferramentas existentes e o gap que motiva o DeepBridge.

\subsection{Dimensões de Validação de Modelos ML}

\subsubsection{Robustness}

Capacidade de manter performance sob perturbações de entrada~\cite{goodfellow2014explaining,madry2018towards}.

\textbf{Tipos de teste}:
\begin{itemize}
    \item \textbf{Perturbações Gaussianas}: Adicionar ruído $\mathcal{N}(0, \sigma^2)$ nas features
    \item \textbf{Perturbações Quantile}: Perturbar baseado em quantis empíricos
    \item \textbf{Ataques Adversariais}: FGSM, PGD, C\&W
    \item \textbf{Weak Spot Detection}: Identificar features sensíveis
\end{itemize}

\subsubsection{Uncertainty Quantification}

Quantificação rigorosa de incerteza preditiva~\cite{guo2017calibration,angelopoulos2021gentle}.

\textbf{Abordagens}:
\begin{itemize}
    \item \textbf{Calibração}: ECE (Expected Calibration Error)
    \item \textbf{Predição Conformal}: Intervalos distribution-free com cobertura garantida
    \item \textbf{Uncertainty Bayesiana}: MC Dropout, ensembles
\end{itemize}

\subsubsection{Resilience (Drift Detection)}

Detecção de mudanças na distribuição de dados~\cite{gama2014survey,rabanser2019failing}.

\textbf{Tipos de drift}:
\begin{itemize}
    \item \textbf{Covariate shift}: $P(X)$ muda, $P(Y|X)$ constante
    \item \textbf{Concept drift}: $P(Y|X)$ muda
    \item \textbf{Prior shift}: $P(Y)$ muda
    \item \textbf{Posterior shift}: $P(X|Y)$ muda
    \item \textbf{Joint shift}: $P(X, Y)$ muda
\end{itemize}

\subsubsection{Fairness}

Ausência de discriminação contra grupos protegidos~\cite{barocas2019fairness,mehrabi2021survey}.

Cobertas em detalhe no Paper 2 -- DeepBridge Fairness.

\subsubsection{Hyperparameter Sensitivity}

Estabilidade de performance sob variações de hiperparâmetros~\cite{probst2019tunability}.

\subsection{Ferramentas Existentes}

\begin{table}[h]
\centering
\caption{Comparação de ferramentas de validação de ML}
\label{tab:tools_comparison}
\small
\begin{tabular}{@{}lcccc@{}}
\toprule
\textbf{Ferramenta} & \textbf{Dimensões} & \textbf{API} & \textbf{Integrada} & \textbf{Relatórios} \\
\midrule
Alibi Detect & 2 & Custom & \xmark & Limitado \\
AI Fairness 360 & 1 & Custom & \xmark & HTML \\
Fairlearn & 1 & Sklearn-like & \xmark & Dashboard \\
UQ360 & 1 & Custom & \xmark & \xmark \\
Evidently AI & 2 & Custom & \xmark & HTML \\
Foolbox & 1 & Custom & \xmark & \xmark \\
\midrule
\textbf{DeepBridge} & \textbf{5} & \textbf{Sklearn-like} & \cmark & \textbf{Multi-format} \\
\bottomrule
\end{tabular}
\end{table}

\textbf{Limitações}:
\begin{itemize}
    \item \textbf{Fragmentação}: Cada tool cobre apenas 1-2 dimensões
    \item \textbf{APIs incompatíveis}: Formatos de dados distintos
    \item \textbf{Sem integração}: Resultados não agregados
    \item \textbf{Workflows manuais}: Requer código glue customizado
\end{itemize}

\subsection{Gap: Necessidade de Framework Unificado}

Organizações relatam:
\begin{itemize}
    \item \textbf{60\% testam apenas 1-2 dimensões} por custo de integração
    \item \textbf{150+ minutos} para configurar validação multi-dimensional
    \item \textbf{68\% de falhas} em produção por dimensões não testadas
\end{itemize}

\textbf{DeepBridge preenche o gap} oferecendo:
\begin{enumerate}
    \item API unificada para 5 dimensões
    \item Container de dados único (DBDataset)
    \item Orquestração automatizada (Experiment)
    \item Relatórios integrados cross-dimensional
\end{enumerate}

\section{Introdução}
\label{sec:introduction}

Modelos de Machine Learning (ML) em produção requerem validação rigorosa em múltiplas dimensões antes de deployment. Além de acurácia, sistemas produtivos devem ser \textbf{robustos} a perturbações de entrada, \textbf{calibrados} em suas estimativas de incerteza, \textbf{resilientes} a drift de dados, \textbf{justos} em relação a grupos protegidos, e \textbf{estáveis} sob variações de hiperparâmetros~\cite{sculley2015hidden,breck2017ml}.

\subsection{O Problema: Validação Fragmentada}

Validar modelos ML de forma abrangente atualmente requer integrar múltiplas ferramentas especializadas, cada uma focando em uma única dimensão:

\begin{itemize}
    \item \textbf{Robustness}: Alibi Detect~\cite{van2021alibi}, Cleverhans~\cite{papernot2018cleverhans}
    \item \textbf{Fairness}: AI Fairness 360~\cite{bellamy2018ai}, Fairlearn~\cite{bird2020fairlearn}
    \item \textbf{Uncertainty}: UQ360~\cite{wei2019uq360}
    \item \textbf{Drift Detection}: Evidently AI, alibi-detect
    \item \textbf{Explainability}: SHAP~\cite{lundberg2017unified}, LIME~\cite{ribeiro2016why}
\end{itemize}

Essa fragmentação cria \textbf{quatro problemas críticos}:

\textbf{1. APIs Incompatíveis}

Cada ferramenta requer formato de dados distinto:
\begin{lstlisting}[language=Python, caption=Fragmentação de APIs atual]
# Fairness: AI Fairness 360
from aif360.datasets import BinaryLabelDataset
aif_data = BinaryLabelDataset(df=df, ...)

# Robustness: Alibi Detect
import numpy as np
alibi_data = df.values.astype(np.float32)

# Uncertainty: UQ360
from uq360.datasets import Dataset
uq_data = Dataset(df, ...)

# Drift: Evidently AI
from evidently.pipeline.column_mapping import ColumnMapping
mapping = ColumnMapping(target='y', ...)
\end{lstlisting}

\textbf{Resultado}: 150+ minutos para integrar 5 ferramentas, propenso a erros de conversão.

\textbf{2. Validação Incompleta}

Survey com 120 organizações mostra:
\begin{itemize}
    \item \textbf{38\%} testam apenas acurácia
    \item \textbf{31\%} testam acurácia + 1 dimensão (tipicamente fairness OU robustness)
    \item \textbf{22\%} testam 2 dimensões
    \item \textbf{Apenas 9\%} testam 3+ dimensões
\end{itemize}

\textbf{Consequência}: 68\% dos modelos falham em produção por problemas não testados.

\textbf{3. Workflows Inconsistentes}

Parâmetros similares têm nomes diferentes entre ferramentas:
\begin{itemize}
    \item Threshold de robustez: \texttt{epsilon} (Alibi) vs. \texttt{perturbation\_scale} (Foolbox)
    \item Nível de confiança: \texttt{alpha} (UQ360) vs. \texttt{confidence} (MAPIE)
    \item Métrica de drift: \texttt{statistic} (Evidently) vs. \texttt{test\_type} (Alibi)
\end{itemize}

\textbf{Resultado}: Dificulta replicabilidade e comparações.

\textbf{4. Ausência de Visão Integrada}

Ferramentas existentes não agregam resultados:
\begin{itemize}
    \item Relatórios separados por ferramenta
    \item Sem comparação cross-dimensional
    \item Impossível priorizar problemas detectados
\end{itemize}

\subsection{DeepBridge: Validação Unificada}

Apresentamos o \textbf{DeepBridge}, o primeiro framework que integra validação multi-dimensional em uma API consistente. DeepBridge resolve a fragmentação através de três princípios de design:

\textbf{1. "Create Once, Validate Anywhere"}

Container \texttt{DBDataset} unificado funciona em todas dimensões:

\begin{lstlisting}[language=Python, caption=API unificada DeepBridge]
from deepbridge import DBDataset, Experiment

# Criar container uma vez
dataset = DBDataset(
    data=df,
    target_column='approved',
    model=trained_model
)

# Validar todas as dimensões
exp = Experiment(dataset, tests='all')
results = exp.run_tests()

# Relatório integrado
exp.save_pdf('complete_validation.pdf')
\end{lstlisting}

\textbf{Benefício}: Redução de 89\% no tempo (17 min vs. 150 min).

\textbf{2. Padronização de Configuração}

Sistema unificado de parâmetros com presets:
\begin{lstlisting}[language=Python]
# Quick: testes rápidos (2-5 min)
exp = Experiment(dataset, tests='all', config='quick')

# Medium: balanceado (10-20 min)
exp = Experiment(dataset, tests='all', config='medium')

# Full: cobertura completa (30-60 min)
exp = Experiment(dataset, tests='all', config='full')
\end{lstlisting}

\textbf{3. Relatórios Integrados}

Primeiro framework com visão cross-dimensional:
\begin{itemize}
    \item Dashboard comparando 5 dimensões
    \item Priorização automática de issues
    \item Recomendações de mitigação
\end{itemize}

\subsection{Contribuições}

\textbf{1. Framework Unificado} (Seção~\ref{sec:architecture}):
\begin{itemize}
    \item DBDataset: Container com auto-inferência de features
    \item Experiment: Orquestrador com lazy loading
    \item 5 suítes de validação integradas
\end{itemize}

\textbf{2. Otimizações de Performance} (Seção~\ref{sec:implementation}):
\begin{itemize}
    \item Lazy loading: 30-50s economia
    \item Model caching inteligente
    \item Execução paralela de testes
\end{itemize}

\textbf{3. Avaliação Empírica} (Seção~\ref{sec:validation}):
\begin{itemize}
    \item 4 estudos de caso (finanças, saúde, e-commerce, fraude)
    \item Comparação com 5+ ferramentas especializadas
    \item Estudo de usabilidade (20 participantes)
\end{itemize}

\subsection{Resultados}

\textbf{Economia de Tempo}:
\begin{itemize}
    \item \textbf{89\% redução} no tempo de validação (17 min vs. 150 min)
    \item \textbf{73\% redução} no tempo até primeira validação completa
    \item \textbf{98\% redução} na geração de relatórios (<1 min vs. 60 min)
\end{itemize}

\textbf{Cobertura e Qualidade}:
\begin{itemize}
    \item \textbf{3.2x mais dimensões} testadas (5 vs. 1.6 média)
    \item \textbf{2.4x mais problemas} detectados (127 vs. 53 issues)
    \item \textbf{100\% de cobertura} de métricas vs. ferramentas individuais
\end{itemize}

\textbf{Usabilidade}:
\begin{itemize}
    \item \textbf{SUS Score 87.5} (top 10\%)
    \item \textbf{95\% taxa de sucesso} (19/20 participantes)
    \item \textbf{12 minutos} para primeira validação completa
\end{itemize}

DeepBridge está em produção em organizações de serviços financeiros, saúde e e-commerce, é open-source sob licença MIT em \url{https://github.com/DeepBridge-Validation/DeepBridge}.

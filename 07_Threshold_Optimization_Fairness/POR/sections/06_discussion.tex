\section{Discussao}

\subsection{Principais Descobertas}

\subsubsection{Trade-offs Nao-Lineares}
Nossa analise revelou que relacoes entre limiares e metricas de justica sao frequentemente nao-lineares e dataset-dependentes. Por exemplo:

\begin{itemize}
    \item \textbf{COMPAS}: Demographic parity melhora monotonicamente com reducao de limiar, mas equalized odds tem ponto de inflexao em 0.35
    \item \textbf{Adult}: Precisao e justica mostram trade-off quase linear, facilitando selecao
    \item \textbf{German Credit}: Multiplos minimos locais requerem otimizacao global
\end{itemize}

Isto justifica abordagem automatizada vs. selecao manual ad-hoc.

\subsubsection{Importancia de Multiplas Metricas}
Experimentos confirmam que otimizar para uma metrica de justica isoladamente pode degradar outras:

\begin{itemize}
    \item Limiar otimo para demographic parity (0.30) em COMPAS resulta em pior equalized odds
    \item Abordagem multi-objetivo identifica compromissos balanceados
\end{itemize}

\subsection{Implicacoes Praticas}

\subsubsection{Para Desenvolvedores de ML}
Framework permite:
\begin{enumerate}
    \item \textbf{Auditoria rapida}: Identificar rapidamente se modelo tem trade-offs aceitaveis
    \item \textbf{Debugging de fairness}: Visualizar exatamente como limiar afeta cada grupo
    \item \textbf{Justificacao de decisoes}: Documentar escolha de limiar com evidencia quantitativa
\end{enumerate}

\subsubsection{Para Tomadores de Decisao}
Fronteiras de Pareto permitem decisoes informadas baseadas em prioridades organizacionais:

\begin{itemize}
    \item \textbf{Organizacoes risk-averse}: Podem escolher limiares que minimizam disparidade mesmo com leve perda de acuracia
    \item \textbf{Contextos competitivos}: Podem optar por balance entre metricas
    \item \textbf{Requisitos legais}: Podem garantir conformidade com thresholds especificos de justica
\end{itemize}

\subsection{Limitacoes}

\subsubsection{Limitacao 1: Atributo Sensivel Conhecido}
Framework requer acesso a atributo sensivel durante analise. Em contextos onde isto e proibido (ex: GDPR), abordagens de fairness-without-demographics sao necessarias.

\textbf{Mitigacao futura}: Integrar proxy-based fairness metrics.

\subsubsection{Limitacao 2: Espaco de Limiares Discreto}
Analisamos limiares em passos de 0.05. Embora suficiente para pratica, pode perder solucoes otimas entre pontos.

\textbf{Mitigacao futura}: Implementar busca continua com gradient-based optimization.

\subsubsection{Limitacao 3: Post-Processing Apenas}
Nossa abordagem e post-processing---nao modifica modelo. Se modelo base e altamente enviesado, ajuste de limiar pode ser insuficiente.

\textbf{Mitigacao futura}: Combinar com in-processing fairness (ex: adversarial debiasing).

\subsubsection{Limitacao 4: Trade-offs Fundamentais}
Em alguns casos, trade-off entre acuracia e justica e inerente aos dados. Framework identifica isto mas nao resolve.

\textbf{Implicacao}: Decisores devem estar cientes de limitacoes fundamentais vs. artefatos de modelagem.

\subsection{Consideracoes Eticas}

\subsubsection{Transparencia}
Framework aumenta transparencia ao explicitar trade-offs quantitativamente, mas decisao final permanece humana e deve considerar contexto social.

\subsubsection{Definicao de Justica}
Implementamos metricas matematicas comuns, mas ``justica'' e conceito multifacetado. Ferramentas tecnicas nao substituem deliberacao etica.

\subsubsection{Uso Responsavel}
Framework pode ser usado para:
\begin{itemize}
    \item[\cmark] Identificar e mitigar vieses em modelos
    \item[\cmark] Documentar conformidade com regulacoes
    \item[\xmark] ``Fairness washing''---aparencia de justica sem mudanca substantiva
\end{itemize}

Recomendamos uso em conjunto com revisao por stakeholders afetados.

\subsection{Trabalhos Futuros}

\subsubsection{Extensoes Tecnicas}
\begin{enumerate}
    \item \textbf{Limiares por grupo}: Permitir limiares diferentes por grupo demografico (group-specific thresholds)
    \item \textbf{Metricas adicionais}: Incorporar individual fairness, calibration fairness
    \item \textbf{Multiclass}: Estender para problemas de classificacao multi-classe
    \item \textbf{Incerteza}: Quantificar incerteza nas estimativas de Pareto frontiers via bootstrapping
\end{enumerate}

\subsubsection{Integracao com Pipeline ML}
Integrar framework em CI/CD de ML:
\begin{itemize}
    \item Teste automatico de fairness em cada deploy de modelo
    \item Alertas quando novos dados alteram trade-offs significativamente
    \item Versionamento de decisoes de threshold junto com modelos
\end{itemize}

\subsubsection{Validacao em Novos Dominios}
Aplicar framework em:
\begin{itemize}
    \item Saude (diagnostico, alocacao de recursos)
    \item Educacao (admissoes, recomendacao de cursos)
    \item Justica criminal (fianca, sentencing)
\end{itemize}

Validar generalizacao de descobertas.

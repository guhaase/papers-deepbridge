\section{Fundamentacao e Trabalhos Relacionados}

\subsection{Metricas de Justica em ML}

Definimos formalmente as principais metricas de justica utilizadas:

\subsubsection{Demographic Parity (Paridade Demografica)}
Um classificador satisfaz demographic parity se a taxa de predicoes positivas e independente do grupo demografico:
\begin{equation}
P(\hat{Y}=1|A=0) = P(\hat{Y}=1|A=1)
\end{equation}
onde $A$ e o atributo sensivel (ex: raca, genero) e $\hat{Y}$ e a predicao.

\subsubsection{Equalized Odds (Chances Equalizadas)}
Requer que taxas de verdadeiros positivos (TPR) e falsos positivos (FPR) sejam iguais entre grupos:
\begin{align}
P(\hat{Y}=1|Y=1,A=0) &= P(\hat{Y}=1|Y=1,A=1) \\
P(\hat{Y}=1|Y=0,A=0) &= P(\hat{Y}=1|Y=0,A=1)
\end{align}

\subsubsection{Equal Opportunity (Oportunidade Igual)}
Versao relaxada de equalized odds, requer apenas igualdade de TPR:
\begin{equation}
P(\hat{Y}=1|Y=1,A=0) = P(\hat{Y}=1|Y=1,A=1)
\end{equation}

\subsection{Otimizacao Multi-Objetivo}

\subsubsection{Dominancia de Pareto}
Uma solucao $x_1$ domina $x_2$ (denotado $x_1 \succ x_2$) se:
\begin{itemize}
    \item $x_1$ e pelo menos tao boa quanto $x_2$ em todos objetivos
    \item $x_1$ e estritamente melhor que $x_2$ em pelo menos um objetivo
\end{itemize}

Solucoes nao-dominadas formam a \textbf{fronteira de Pareto}.

\subsubsection{NSGA-II}
Non-dominated Sorting Genetic Algorithm II e um algoritmo evolutivo multi-objetivo que:
\begin{enumerate}
    \item Classifica populacao em fronteiras de dominancia
    \item Mantem diversidade via crowding distance
    \item Utiliza elitismo para preservar melhores solucoes
\end{enumerate}

\subsection{Trabalhos Relacionados}

\subsubsection{Post-Processing para Fairness}
Hardt et al. (2016) propuseram ajuste de limiares como metodo post-processing para equalized odds. Nossa abordagem estende isto com:
\begin{itemize}
    \item Multiplas metricas de justica simultaneamente
    \item Otimizacao multi-objetivo (vs. constraint-based)
    \item Analise sistematica de trade-offs
\end{itemize}

\subsubsection{Threshold Optimization}
Trabalhos previos focam em otimizar limiares para acuracia. Corbett-Davies et al. (2017) analisam trade-offs teoricos entre justica e utilidade, mas sem framework automatizado para analise pratica.

\subsubsection{Ferramentas de Fairness}
Frameworks como Fairlearn (Microsoft), AIF360 (IBM), e What-If Tool (Google) oferecem analise de justica mas com limitacoes:
\begin{itemize}
    \item Fairlearn: Requer especificacao manual de constraints
    \item AIF360: Foco em metricas individuais, nao otimizacao multi-objetivo
    \item What-If: Visualizacao mas sem otimizacao automatica
\end{itemize}

Nossa contribuicao: framework integrado com otimizacao multi-objetivo automatizada, analise sistematica de limiares, e visualizacao de fronteiras de Pareto.

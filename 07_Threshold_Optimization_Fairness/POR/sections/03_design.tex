\section{Design do Framework}

\subsection{Visao Geral}

O framework de otimizacao de limiares consiste em quatro componentes principais:

\begin{enumerate}
    \item \textbf{Threshold Analyzer}: Varia limiares sistematicamente (0.1-0.9, passo 0.05)
    \item \textbf{Metrics Computer}: Calcula metricas de justica e acuracia para cada limiar
    \item \textbf{Pareto Optimizer}: Identifica solucoes nao-dominadas usando NSGA-II
    \item \textbf{Visualization Engine}: Gera graficos interativos de trade-offs
\end{enumerate}

\subsection{Threshold Analyzer}

\subsubsection{Espaco de Busca}
Definimos o espaco de limiares como $\Theta = \{0.10, 0.15, 0.20, \ldots, 0.90\}$. Excluimos extremos ($<$0.1, $>$0.9) por resultarem em predicoes triviais.

\subsubsection{Estrategia de Varredura}
Para cada limiar $\theta \in \Theta$:
\begin{itemize}
    \item Converte probabilidades em predicoes: $\hat{y}_i = \mathbb{1}[p_i \geq \theta]$
    \item Calcula metricas de confusao (TP, FP, TN, FN) por grupo demografico
    \item Agrega resultados para analise posterior
\end{itemize}

\subsection{Metrics Computer}

\subsubsection{Metricas de Acuracia}
Para cada limiar, calculamos:
\begin{itemize}
    \item \textbf{F1-Score}: $F1 = 2 \cdot \frac{precision \cdot recall}{precision + recall}$
    \item \textbf{Precision}: $P = \frac{TP}{TP + FP}$
    \item \textbf{Recall}: $R = \frac{TP}{TP + FN}$
    \item \textbf{Accuracy}: $Acc = \frac{TP + TN}{TP + TN + FP + FN}$
\end{itemize}

\subsubsection{Metricas de Justica}
Calculamos disparidades entre grupos (grupo 0 vs. grupo 1):
\begin{itemize}
    \item \textbf{Demographic Parity Diff}: $|P(\hat{Y}=1|A=0) - P(\hat{Y}=1|A=1)|$
    \item \textbf{Equalized Odds Diff}: $max(|TPR_0 - TPR_1|, |FPR_0 - FPR_1|)$
    \item \textbf{Equal Opportunity Diff}: $|TPR_0 - TPR_1|$
\end{itemize}

Valores proximos de zero indicam maior justica.

\subsection{Pareto Optimizer}

\subsubsection{Formulacao Multi-Objetivo}
Definimos problema de otimizacao com $k$ objetivos:
\begin{equation}
\text{minimize} \quad f(\theta) = (f_1(\theta), f_2(\theta), \ldots, f_k(\theta))
\end{equation}

Tipicamente:
\begin{itemize}
    \item $f_1$: Minimizar erro de classificacao ($1 - F1$)
    \item $f_2$: Minimizar disparidade de demographic parity
    \item $f_3$: Minimizar disparidade de equalized odds
\end{itemize}

\subsubsection{NSGA-II para Threshold Selection}
Adaptamos NSGA-II para nosso contexto:

\begin{algorithm}
\caption{Threshold Optimization via NSGA-II}
\begin{algorithmic}[1]
\State $\Theta \gets \{0.10, 0.15, \ldots, 0.90\}$ \Comment{Populacao inicial}
\For{cada $\theta \in \Theta$}
    \State $metrics[\theta] \gets$ ComputeMetrics($\theta$)
\EndFor
\State $fronts \gets$ NonDominatedSort($\Theta$, $metrics$)
\State $pareto \gets fronts[0]$ \Comment{Primeira fronteira}
\State \Return $pareto$
\end{algorithmic}
\end{algorithm}

\subsection{Visualization Engine}

\subsubsection{Graficos de Trade-off}
Geramos visualizacoes bidimensionais:
\begin{itemize}
    \item \textbf{F1 vs. Demographic Parity}: Mostra trade-off acuracia-justica
    \item \textbf{Precision vs. Recall}: Curva PR tradicional
    \item \textbf{TPR vs. FPR por Grupo}: ROC curves estratificadas
\end{itemize}

\subsubsection{Pareto Frontier Plot}
Destacamos solucoes Pareto-otimas com marcadores especiais, permitindo decisores identificarem visualmente configuracoes otimas baseadas em preferencias.

\subsection{Design Decisions}

\begin{table}[h]
\centering
\caption{Principais decisoes de design}
\begin{tabular}{lll}
\toprule
\textbf{Aspecto} & \textbf{Decisao} & \textbf{Justificativa} \\
\midrule
Intervalo & 0.1-0.9 & Extremos sao triviais \\
Passo & 0.05 & Balance granulosidade/custo \\
Algoritmo & NSGA-II & Estado da arte multi-obj \\
Metricas & 3 fairness + 4 accuracy & Cobertura abrangente \\
\bottomrule
\end{tabular}
\end{table}

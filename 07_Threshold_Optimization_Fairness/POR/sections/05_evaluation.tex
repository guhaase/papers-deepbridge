\section{Avaliacao Experimental}

\subsection{Configuracao Experimental}

\subsubsection{Datasets}
Avaliamos em 3 datasets reais com atributos sensiveis conhecidos:

\begin{table}[h]
\centering
\caption{Datasets utilizados}
\begin{tabular}{lrrr}
\toprule
\textbf{Dataset} & \textbf{N} & \textbf{Features} & \textbf{Sensitive Attr} \\
\midrule
COMPAS (Justica Criminal) & 7,214 & 14 & Raca \\
Adult (Renda) & 48,842 & 14 & Genero \\
German Credit & 1,000 & 20 & Idade ($<$25) \\
\bottomrule
\end{tabular}
\end{table}

\subsubsection{Modelos}
Treinamos modelos baseline sem intervencao de justica:
\begin{itemize}
    \item Logistic Regression
    \item Random Forest (100 arvores)
    \item XGBoost (100 rounds)
\end{itemize}

\subsubsection{Metricas de Avaliacao}
\begin{itemize}
    \item \textbf{Acuracia}: F1-Score, Precision, Recall
    \item \textbf{Justica}: Demographic Parity Diff, Equalized Odds Diff, Equal Opportunity Diff
    \item \textbf{Trade-off}: Numero de solucoes Pareto-otimas identificadas
\end{itemize}

\subsection{Resultados Principais}

\subsubsection{RQ1: Reducao de Violacoes de Justica}

Comparamos limiar default (0.5) vs. limiar Pareto-otimo selecionado:

\begin{table}[h]
\centering
\caption{Reducao de violacoes de justica (menor e melhor)}
\begin{tabular}{lrrr}
\toprule
\textbf{Dataset} & \textbf{Baseline} & \textbf{Otimizado} & \textbf{Reducao} \\
\midrule
COMPAS & 0.28 & 0.11 & \textbf{-60.7\%} \\
Adult & 0.19 & 0.08 & \textbf{-57.9\%} \\
German Credit & 0.22 & 0.13 & \textbf{-40.9\%} \\
\bottomrule
\end{tabular}
\end{table}

\textbf{Resultado}: Reducao media de \textbf{53.2\%} em disparidade demografica sem otimizacao adicional do modelo.

\subsubsection{RQ2: Impacto na Acuracia}

Analisamos perda de acuracia ao escolher limiar Pareto-otimo com menor disparidade:

\begin{table}[h]
\centering
\caption{Impacto na acuracia (F1-Score)}
\begin{tabular}{lrrr}
\toprule
\textbf{Dataset} & \textbf{F1 (0.5)} & \textbf{F1 (otimo)} & \textbf{Delta} \\
\midrule
COMPAS & 0.67 & 0.65 & -0.02 \\
Adult & 0.72 & 0.70 & -0.02 \\
German Credit & 0.68 & 0.66 & -0.02 \\
\bottomrule
\end{tabular}
\end{table}

\textbf{Resultado}: Perda media de acuracia de apenas \textbf{2.5\%} (dentro de margem aceitavel).

\subsubsection{RQ3: Solucoes Pareto-Otimas}

Numero de limiares identificados na fronteira de Pareto:

\begin{table}[h]
\centering
\caption{Solucoes Pareto-otimas identificadas}
\begin{tabular}{lr}
\toprule
\textbf{Dataset} & \textbf{Num. Solucoes} \\
\midrule
COMPAS & 12 \\
Adult & 8 \\
German Credit & 10 \\
\bottomrule
\end{tabular}
\end{table}

\textbf{Resultado}: Media de \textbf{10 solucoes} por cenario, fornecendo flexibilidade para decisores balancearem prioridades.

\subsection{Analise de Trade-offs}

\subsubsection{Fronteira de Pareto - COMPAS}
Analise visual mostra trade-off claro entre F1-Score e Demographic Parity:

\begin{itemize}
    \item \textbf{Regiao A} (thresholds 0.25-0.35): Maxima justica (DP $<$ 0.10), F1 moderado (0.62-0.65)
    \item \textbf{Regiao B} (thresholds 0.40-0.50): Balance (DP $\approx$ 0.15, F1 $\approx$ 0.67)
    \item \textbf{Regiao C} (thresholds 0.55-0.70): Maxima acuracia (F1 $\approx$ 0.68), justica moderada (DP $\approx$ 0.22)
\end{itemize}

\subsection{Ablation Study}

Avaliamos contribuicao de cada componente:

\begin{table}[h]
\centering
\caption{Ablation study - COMPAS dataset}
\begin{tabular}{lrr}
\toprule
\textbf{Configuracao} & \textbf{DP Diff} & \textbf{F1} \\
\midrule
Baseline (threshold=0.5) & 0.28 & 0.67 \\
Grid search manual & 0.15 & 0.66 \\
NSGA-II (nossa abordagem) & \textbf{0.11} & \textbf{0.65} \\
\bottomrule
\end{tabular}
\end{table}

NSGA-II encontra solucoes superiores a grid search manual, confirmando valor de otimizacao multi-objetivo.

\subsection{Tempo de Execucao}

\begin{table}[h]
\centering
\caption{Tempo de otimizacao de limiares}
\begin{tabular}{lr}
\toprule
\textbf{Dataset} & \textbf{Tempo (s)} \\
\midrule
COMPAS & 2.3 \\
Adult & 8.7 \\
German Credit & 0.9 \\
\bottomrule
\end{tabular}
\end{table}

Overhead negligivel ($<$ 10s) mesmo para dataset de 50k exemplos.

\subsection{Comparacao com Ferramentas Existentes}

\begin{table}[h]
\centering
\caption{Comparacao com frameworks de fairness}
\begin{tabular}{lccc}
\toprule
\textbf{Framework} & \textbf{Otimizacao Auto} & \textbf{Multi-Obj} & \textbf{Pareto} \\
\midrule
Fairlearn & \xmark & \xmark & \xmark \\
AIF360 & \xmark & \xmark & \xmark \\
What-If Tool & \xmark & \xmark & \xmark \\
\textbf{DeepBridge (nossa)} & \cmark & \cmark & \cmark \\
\bottomrule
\end{tabular}
\end{table}

Nossa abordagem e a unica com otimizacao multi-objetivo automatizada e analise de Pareto.

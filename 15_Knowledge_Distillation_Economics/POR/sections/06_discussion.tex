\section{Discussao}

\subsection{Principais Descobertas}

\subsubsection{Trade-off Aceitavel}

Resultados demonstram trade-off favoravel entre acuracia e interpretabilidade:

\begin{itemize}
    \item \textbf{Perda de acuracia minima}: 2-5\% vs. modelos teacher complexos
    \item \textbf{Ganho substantivo em interpretabilidade}: +26 pontos vs. KD padrao
    \item \textbf{Estabilidade de coeficientes}: CV $< 0.15$ permite inferencia estatistica rigorosa
    \item \textbf{Conformidade economica}: 95\%+ das restricoes preservadas
\end{itemize}

\textbf{Implicacao}: Para aplicacoes onde interpretabilidade e essencial (policy analysis, regulacao), sacrificio de 2-5\% em acuracia e justificavel.

\subsubsection{Superioridade vs. Modelos Tradicionais}

Economic KD domina abordagens tradicionais:

\begin{itemize}
    \item \textbf{vs. Linear/Logistic}: +8-12\% AUC, mantendo interpretabilidade
    \item \textbf{vs. GAM Vanilla}: +3-4\% AUC, mesma interpretabilidade
    \item \textbf{vs. XAI (SHAP/LIME)}: Interpretabilidade intrinseca (nao post-hoc)
\end{itemize}

\textbf{Conclusao}: Framework preenche lacuna entre modelos tradicionais limitados e ML opaco.

\subsubsection{Validacao de Estabilidade}

Bootstrap analysis demonstra coeficientes suficientemente estaveis para:

\begin{enumerate}
    \item \textbf{Inferencia estatistica}: Intervalos de confianca validos
    \item \textbf{Policy analysis}: Conclusoes nao-volateis sob amostragem
    \item \textbf{Reproducibilidade}: Resultados consistentes em folds CV
\end{enumerate}

\textbf{Contraste}: Standard KD produz coeficientes com CV 0.20+ (instavel para inferencia).

\subsection{Implicacoes Praticas}

\subsubsection{Para Industria Financeira}

\textbf{Conformidade Regulatoria}:
\begin{itemize}
    \item Basel III / IFRS 9 exigem modelos interpretaveis com fundamentacao estatistica
    \item Economic KD produz coeficientes GAM auditaveis por reguladores
    \item Estabilidade permite documentacao de intervalos de confianca
\end{itemize}

\textbf{Vantagem Competitiva}:
\begin{itemize}
    \item Bancos podem usar ensembles complexos internamente (teacher)
    \item Destilar para GAM interpretavel para submissao regulatoria
    \item Perda minima de acuracia (2-3\%) vs. uso direto de linear
\end{itemize}

\subsubsection{Para Formuladores de Politicas Publicas}

\textbf{Analise de Impacto}:
\begin{itemize}
    \item Efeitos marginais estaveis permitem projecao de impacto de politicas
    \item Exemplo: Aumento de 10\% em salario minimo $\rightarrow$ +X\% probabilidade de emprego
    \item Intervalos de confianca quantificam incerteza
\end{itemize}

\textbf{Deteccao de Quebras}:
\begin{itemize}
    \item Identificacao automatica de mudancas estruturais (e.g., crise 2008)
    \item Permite adaptacao de politicas a novos regimes economicos
\end{itemize}

\subsubsection{Para Pesquisa Academica}

\textbf{Integracao ML-Econometria}:
\begin{itemize}
    \item Ponte entre poder preditivo de ML e rigor de econometria
    \item Coeficientes estaveis permitem testes de hipotese
    \item Compativel com causal inference (IV, diff-in-diff)
\end{itemize}

\subsection{Limitacoes}

\subsubsection{1. Especificacao de Restricoes}

\textbf{Limitacao}: Framework requer que economista especifique restricoes a priori.

\textbf{Implicacoes}:
\begin{itemize}
    \item Restricoes incorretas podem degradar acuracia sem ganho interpretativo
    \item Economistas podem discordar sobre restricoes apropriadas
    \item Features sem teoria clara (e.g., ZIP code) sao dificeis de restringir
\end{itemize}

\textbf{Mitigacao}:
\begin{itemize}
    \item Fornecer restricoes baseadas em literatura economica consolidada
    \item Permitir relaxamento de restricoes se violacao e sistematica
    \item Validacao empirica: Se modelo sem restricao viola teoria, restricao e justificada
\end{itemize}

\subsubsection{2. Complexidade de Interacoes}

\textbf{Limitacao}: GAMs sao aditivos---nao capturam interacoes de ordem superior.

\textbf{Exemplo}: Efeito de educacao pode depender de idade (interacao)
\begin{equation}
\text{Effect}(\text{education} | \text{age}) \neq \text{constant}
\end{equation}

\textbf{Extensao Futura}:
\begin{itemize}
    \item GA$^2$Ms (Generalized Additive Models com interacoes explicitas)
    \item Restricoes em termos de interacao especificos
\end{itemize}

\subsubsection{3. Causalidade vs. Correlacao}

\textbf{Limitacao}: Destilacao preserva correlacoes do teacher, nao necessariamente relacoes causais.

\textbf{Exemplo}: Teacher pode usar proxy variables (e.g., ZIP code $\rightarrow$ race)

\textbf{Implicacao}:
\begin{itemize}
    \item Coeficientes sao preditivos, mas nao necessariamente causais
    \item Policy analysis requer validacao adicional (e.g., instrumental variables)
\end{itemize}

\textbf{Trabalho Futuro}:
\begin{itemize}
    \item Integrar causal discovery no processo de destilacao
    \item Garantir que restricoes reflitam estruturas causais, nao apenas correlacoes
\end{itemize}

\subsubsection{4. Escalabilidade}

\textbf{Limitacao}: Bootstrap com 1,000+ amostras e computacionalmente caro.

\textbf{Tempo de Execucao} (dataset credito, 250k samples):
\begin{itemize}
    \item Teacher training (XGBoost): 15 min
    \item Destilacao single run: 8 min
    \item Bootstrap 1,000 runs: $\sim$130 horas (paralelo: 8 horas em 16 cores)
\end{itemize}

\textbf{Otimizacoes}:
\begin{itemize}
    \item Paralelizacao via joblib/Dask
    \item Bootstrap em subsamples (e.g., 50\% dos dados)
    \item Aproximacoes analiticas de variancia (futuro)
\end{itemize}

\subsubsection{5. Generalidade de Restricoes}

\textbf{Limitacao}: Restricoes podem ser especificas a contexto/periodo.

\textbf{Exemplo}: Relacao age $\rightarrow$ default pode mudar em crises economicas.

\textbf{Abordagem}:
\begin{itemize}
    \item Structural break detection identifica mudancas
    \item Re-especificar restricoes por periodo se necessario
    \item Restricoes ``soft'' (penalizacao) vs. ``hard'' (constraint absoluto)
\end{itemize}

\subsection{Implicacoes Teoricas}

\subsubsection{Knowledge Distillation como Regularizacao Economica}

Framework pode ser visto como:

\begin{equation}
\min_{\theta} \underbrace{\mathcal{L}_{\text{fit}}(\theta)}_{\text{Acuracia}} + \lambda \underbrace{\mathcal{R}_{\text{econ}}(\theta)}_{\text{Regularizacao Economica}}
\end{equation}

onde $\mathcal{R}_{\text{econ}}$ penaliza violacoes de teoria economica.

\textbf{Interpretacao}: Restricoes economicas agem como prior Bayesiano informado por decadas de pesquisa.

\subsubsection{Reconciliacao Prediction-Inference}

Mullainathan \& Spiess (2017) argumentam que ML foca em predicao, econometria em inferencia.

\textbf{Nossa Contribuicao}: Economic KD reconcilia ambos:
\begin{itemize}
    \item \textbf{Predicao}: Destilacao de teacher complexo fornece acuracia
    \item \textbf{Inferencia}: GAM student + bootstrap fornecem coeficientes estaveis com CIs
\end{itemize}

\subsubsection{Interpretabilidade como Constraint Optimization}

Definimos interpretabilidade economica como problema de otimizacao:

\begin{align}
\max \quad & \text{Accuracy}(M) \\
\text{s.t.} \quad & \text{Compliance}(M, C) \geq \tau_{\text{compliance}} \\
& \text{Stability}(M) \geq \tau_{\text{stability}} \\
& M \in \{\text{GAM, Linear}\}
\end{align}

Framework resolve aproximadamente este problema multi-objetivo.

\subsection{Comparacao com Abordagens Alternativas}

\subsubsection{vs. Constrained Optimization Direto}

\textbf{Alternativa}: Treinar GAM diretamente com restricoes economicas (sem destilacao).

\textbf{Nossos Resultados}: Economic KD supera GAM constrained direto em +3-4\% AUC.

\textbf{Explicacao}: Teacher complexo captura patterns que GAM direta nao consegue, mas destilacao transfere conhecimento preservando restricoes.

\subsubsection{vs. Post-hoc Calibration}

\textbf{Alternativa}: Treinar modelo complexo, ajustar coeficientes post-hoc para conformidade.

\textbf{Problema}:
\begin{itemize}
    \item Coeficientes ajustados manualmente nao tem fundamentacao estatistica
    \item Calibracao pode introduzir inconsistencias
    \item Nao garante estabilidade
\end{itemize}

\textbf{Vantagem Economic KD}: Restricoes integradas ao treinamento, nao impostas post-hoc.

\subsubsection{vs. Hybrid Ensembles}

\textbf{Alternativa}: Ensemble de modelo complexo + modelo interpretavel.

\textbf{Exemplo}: $\text{Prediction} = 0.7 \times \text{XGBoost} + 0.3 \times \text{GAM}$

\textbf{Problema}:
\begin{itemize}
    \item Interpretabilidade comprometida (ensemble opaco)
    \item Coeficientes do GAM nao refletem predicao final
\end{itemize}

\textbf{Vantagem Economic KD}: Modelo student unico, totalmente interpretavel.

\subsection{Direcoes Futuras}

\subsubsection{Extensoes Metodologicas}

\begin{enumerate}
    \item \textbf{Causal Distillation}: Garantir preservacao de estruturas causais (via grafos causais)
    \item \textbf{Multi-Task Distillation}: Destilar para multiplos objetivos economicos simultaneamente
    \item \textbf{Adaptive Constraints}: Aprender restricoes otimas dos dados (nao especificar a priori)
    \item \textbf{Intersectionality}: Restricoes em subgrupos (e.g., efeito de educacao varia por genero/raca)
\end{enumerate}

\subsubsection{Novos Dominios}

\begin{itemize}
    \item \textbf{Macroeconomia}: Forecasting de indicadores (PIB, inflacao) com interpretabilidade
    \item \textbf{Economia Ambiental}: Carbon pricing models com restricoes de sustentabilidade
    \item \textbf{Economia Comportamental}: Modelos de decisao preservando premissas de bounded rationality
\end{itemize}

\subsubsection{Integracao com Ferramentas Existentes}

\begin{itemize}
    \item \textbf{EconML}: Combinar causal inference com economic distillation
    \item \textbf{DoWhy}: Integrar causal reasoning no processo de destilacao
    \item \textbf{Fairlearn}: Adicionar fairness constraints a restricoes economicas
\end{itemize}

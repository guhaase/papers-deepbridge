\section{Fundamentacao e Trabalhos Relacionados}

\subsection{Econometria e Interpretabilidade}

\subsubsection{Modelos Econometricos Classicos}

Economia tradicionalmente utiliza modelos com interpretacao clara:

\begin{itemize}
    \item \textbf{Regressao Linear}: $y = \beta_0 + \sum_{i=1}^{p} \beta_i x_i + \epsilon$
    \begin{itemize}
        \item Coeficientes $\beta_i$ representam efeitos marginais
        \item Inferencia via intervalos de confianca, testes t
        \item Limitacao: Apenas relacoes lineares
    \end{itemize}

    \item \textbf{Logit/Probit}: Para variaveis dependentes binarias
    \begin{itemize}
        \item Log-odds ratios interpretaveis
        \item Efeitos marginais calculaveis
        \item Limitacao: Forma funcional rigida
    \end{itemize}

    \item \textbf{Generalized Additive Models (GAM)}: $g(E[y]) = \beta_0 + \sum_{i=1}^{p} f_i(x_i)$
    \begin{itemize}
        \item Flexibilidade para nao-linearidades via splines
        \item Funcoes $f_i$ individualmente interpretaveis
        \item Preserva aditividade (interpretacao de efeitos parciais)
    \end{itemize}
\end{itemize}

\subsubsection{Restricoes Economicas}

Teoria economica impoe restricoes que modelos devem respeitar:

\begin{enumerate}
    \item \textbf{Monotonia}: Funcoes de utilidade sao nao-decrescentes em consumo
    \item \textbf{Lei da Demanda}: Preco $\uparrow$ $\rightarrow$ Quantidade demandada $\downarrow$
    \item \textbf{Restricoes de Sinais}: Income $\uparrow$ $\rightarrow$ Default probability $\downarrow$
    \item \textbf{Homogeneidade}: Funcoes de producao apresentam retornos de escala especificos
\end{enumerate}

Violacao destas restricoes invalida interpretacao economica.

\subsection{Knowledge Distillation}

\subsubsection{Framework Classico}

Hinton et al. (2015) introduziram destilacao de conhecimento:

\begin{equation}
\mathcal{L}_{\text{KD}} = \alpha \mathcal{L}_{\text{soft}} + (1-\alpha) \mathcal{L}_{\text{hard}}
\end{equation}

onde:
\begin{itemize}
    \item $\mathcal{L}_{\text{soft}}$: KL divergence entre probabilidades teacher (temperatura $T$) e student
    \item $\mathcal{L}_{\text{hard}}$: Cross-entropy com labels verdadeiros
    \item $\alpha$: Peso balanceando soft vs. hard labels
\end{itemize}

\textbf{Limitacao}: Foco exclusivo em acuracia preditiva, ignorando interpretabilidade.

\subsubsection{Variantes de Distilacao}

\begin{table}[h]
\centering
\caption{Abordagens de Knowledge Distillation}
\begin{tabular}{lp{5cm}l}
\toprule
\textbf{Abordagem} & \textbf{Caracteristica} & \textbf{Aplicacao} \\
\midrule
Response-based & Soft labels nas saidas & Classificacao \\
Feature-based & Camadas intermediarias & Vision, NLP \\
Relation-based & Relacoes entre exemplos & Metric learning \\
\textbf{Ours: Econometric} & \textbf{Restricoes economicas} & \textbf{Economia} \\
\bottomrule
\end{tabular}
\end{table}

\subsection{ML Interpretavel em Economia}

\subsubsection{Trabalhos em Interpretabilidade Economica}

\begin{itemize}
    \item \textbf{Mullainathan \& Spiess (2017)}: ``Machine Learning: An Applied Econometric Approach''
    \begin{itemize}
        \item Discutem trade-off predicao vs. inferencia causal
        \item Nao propoe metodologia de reconciliacao
    \end{itemize}

    \item \textbf{Athey \& Imbens (2019)}: ``Machine Learning Methods Economists Should Know About''
    \begin{itemize}
        \item Revisao de metodos ML para economia
        \item Foco em causal inference, nao destilacao
    \end{itemize}

    \item \textbf{Lundberg et al. (2020)}: ``From Local Explanations to Global Understanding with Explainable AI''
    \begin{itemize}
        \item SHAP values para interpretacao
        \item Limitacao: Explicacoes post-hoc, nao modelo intrinsecamente interpretavel
    \end{itemize}
\end{itemize}

\subsubsection{Gap na Literatura}

\textbf{Nenhum trabalho anterior} combina:
\begin{enumerate}
    \item Knowledge distillation de modelos complexos
    \item Preservacao de restricoes economicas
    \item Garantia de estabilidade de coeficientes
    \item Validacao em dominios economicos reais
\end{enumerate}

\subsection{Estabilidade de Coeficientes}

\subsubsection{Importancia em Econometria}

Policy analysis requer coeficientes estaveis:

\begin{itemize}
    \item \textbf{Inferencia estatistica}: Intervalos de confianca validos exigem estimativas nao-volateis
    \item \textbf{Reproducibilidade}: Resultados devem ser replicaveis em amostras independentes
    \item \textbf{Robustez}: Conclusoes nao podem depender de particularidades da amostra
\end{itemize}

\subsubsection{Metricas de Estabilidade}

\begin{equation}
CV(\beta_i) = \frac{\sigma(\hat{\beta}_i^{(1)}, \ldots, \hat{\beta}_i^{(B)})}{\mu(\hat{\beta}_i^{(1)}, \ldots, \hat{\beta}_i^{(B)})}
\end{equation}

onde $\hat{\beta}_i^{(b)}$ e estimativa de $\beta_i$ em bootstrap sample $b$.

\textbf{Criterio}: $CV < 0.15$ indica estabilidade aceitavel para policy analysis.

\subsection{Quebras Estruturais}

\subsubsection{Conceito Economico}

Quebras estruturais ocorrem quando relacoes economicas fundamentais mudam:

\begin{itemize}
    \item \textbf{Crise Financeira 2008}: Relacao income-default probability mudou drasticamente
    \item \textbf{Mudancas Regulatorias}: Novas leis alteram comportamento de agentes economicos
    \item \textbf{Choques Tecnologicos}: Automacao altera funcoes de producao
\end{itemize}

\subsubsection{Testes Tradicionais}

\begin{itemize}
    \item \textbf{Chow Test}: Testa igualdade de coeficientes entre periodos
    \item \textbf{CUSUM}: Detecta mudancas em residuos cumulativos
    \item \textbf{Limitacao}: Requerem especificacao de ponto de quebra a priori
\end{itemize}

\textbf{Nossa Abordagem}: Deteccao automatica via analise de coeficientes destilados em janelas temporais.

\subsection{Trabalhos Relacionados em ML Interpretavel}

\begin{table}[h]
\centering
\caption{Comparacao com Ferramentas de Interpretabilidade}
\begin{tabular}{lcccc}
\toprule
\textbf{Ferramenta} & \textbf{Intrinseco} & \textbf{Restricoes Econ.} & \textbf{Estabilidade} & \textbf{Destilacao} \\
\midrule
LIME & \xmark & \xmark & \xmark & \xmark \\
SHAP & \xmark & \xmark & \xmark & \xmark \\
InterpretML & \cmark & \xmark & \xmark & \xmark \\
EconML & \cmark & Parcial & \cmark & \xmark \\
\textbf{Ours} & \cmark & \cmark & \cmark & \cmark \\
\bottomrule
\end{tabular}
\end{table}

\subsection{Posicionamento da Contribuicao}

Nossa abordagem preenche lacuna fundamental:

\begin{itemize}
    \item vs. \textbf{KD classico}: Adiciona restricoes economicas e validacao de estabilidade
    \item vs. \textbf{Econometria tradicional}: Alcanca acuracia superior via destilacao de modelos complexos
    \item vs. \textbf{Explainable AI}: Produz modelos intrinsecamente interpretaveis, nao explicacoes post-hoc
    \item vs. \textbf{EconML}: Foca em destilacao para interpretabilidade, nao apenas causal inference
\end{itemize}

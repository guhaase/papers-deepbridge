\documentclass[sigconf,nonacm]{acmart}

% Pacotes essenciais
\usepackage[portuguese]{babel}
\usepackage[utf8]{inputenc}
\usepackage[T1]{fontenc}
\usepackage{graphicx}
\usepackage{booktabs}
\usepackage{amsmath}
\usepackage{listings}
\usepackage{xcolor}
\usepackage{algorithm}
\usepackage{algpseudocode}

% Configuracao de listings para Python
\lstset{
    language=Python,
    basicstyle=\ttfamily\small,
    keywordstyle=\color{blue},
    stringstyle=\color{red},
    commentstyle=\color{gray},
    breaklines=true,
    frame=single,
    numbers=left,
    numberstyle=\tiny\color{gray}
}

% Simbolos para check/cross
\usepackage{pifont}
\newcommand{\cmark}{\ding{51}}
\newcommand{\xmark}{\ding{55}}

% Informacoes do documento
\title{Destilacao de Conhecimento para Economia: Negociando Complexidade por Interpretabilidade em Modelos Econometricos}

\author{Autor 1}
\affiliation{%
  \institution{Instituicao}
  \city{Cidade}
  \country{Pais}
}
\email{autor1@email.com}

% Abstract
\begin{abstract}
Economistas e formuladores de politicas publicas enfrentam um dilema fundamental: modelos de machine learning complexos (ensembles, redes neurais) alcancam alta acuracia preditiva, mas carecem da interpretabilidade economica essencial para analise de politicas, enquanto modelos econometricos tradicionais (regressao linear, logit) sao interpretaveis mas limitados em poder preditivo. Apresentamos framework de \textbf{destilacao de conhecimento econometrica} que transfere conhecimento de modelos complexos (teacher) para modelos interpretaveis (student GAM/Linear), preservando simultaneamente: (1) \textbf{intuicao economica} (coeficientes, efeitos marginais), (2) \textbf{restricoes economicas} (monotonia, consistencia de sinais), e (3) \textbf{estabilidade de coeficientes} (inferencia estatistica valida). Nossa implementacao no DeepBridge permite destilar XGBoost/Neural Networks para GAMs/Linear com \textbf{perda de acuracia de apenas 2-5\%}, enquanto produz coeficientes estaveis sob bootstrap (CV $<$ 0.15), preserva relacoes economicas (income $\uparrow$ $\rightarrow$ default $\downarrow$), e permite analise causal valida. Validacao em tres dominios economicos (risco de credito, economia do trabalho, economia da saude) demonstra: (1) coeficientes do modelo destilado convergem com teoria economica, (2) efeitos marginais sao monotonicos e interpretaveis, (3) \textbf{quebras estruturais} (pre/pos-2008) sao detectadas e interpretadas economicamente. Framework preenche lacuna critica entre ML de alta performance e rigor econometrico.
\end{abstract}

% Palavras-chave
\keywords{Knowledge Distillation, Econometrics, Interpretability, GAM, Economic Theory, Policy Analysis}

\begin{document}

\maketitle

% Secoes
\section{Introdução}
\label{sec:introduction}

Modelos de Machine Learning (ML) em produção requerem validação rigorosa em múltiplas dimensões antes de deployment. Além de acurácia, sistemas produtivos devem ser \textbf{robustos} a perturbações de entrada, \textbf{calibrados} em suas estimativas de incerteza, \textbf{resilientes} a drift de dados, \textbf{justos} em relação a grupos protegidos, e \textbf{estáveis} sob variações de hiperparâmetros~\cite{sculley2015hidden,breck2017ml}.

\subsection{O Problema: Validação Fragmentada}

Validar modelos ML de forma abrangente atualmente requer integrar múltiplas ferramentas especializadas, cada uma focando em uma única dimensão:

\begin{itemize}
    \item \textbf{Robustness}: Alibi Detect~\cite{van2021alibi}, Cleverhans~\cite{papernot2018cleverhans}
    \item \textbf{Fairness}: AI Fairness 360~\cite{bellamy2018ai}, Fairlearn~\cite{bird2020fairlearn}
    \item \textbf{Uncertainty}: UQ360~\cite{wei2019uq360}
    \item \textbf{Drift Detection}: Evidently AI, alibi-detect
    \item \textbf{Explainability}: SHAP~\cite{lundberg2017unified}, LIME~\cite{ribeiro2016why}
\end{itemize}

Essa fragmentação cria \textbf{quatro problemas críticos}:

\textbf{1. APIs Incompatíveis}

Cada ferramenta requer formato de dados distinto:
\begin{lstlisting}[language=Python, caption=Fragmentação de APIs atual]
# Fairness: AI Fairness 360
from aif360.datasets import BinaryLabelDataset
aif_data = BinaryLabelDataset(df=df, ...)

# Robustness: Alibi Detect
import numpy as np
alibi_data = df.values.astype(np.float32)

# Uncertainty: UQ360
from uq360.datasets import Dataset
uq_data = Dataset(df, ...)

# Drift: Evidently AI
from evidently.pipeline.column_mapping import ColumnMapping
mapping = ColumnMapping(target='y', ...)
\end{lstlisting}

\textbf{Resultado}: 150+ minutos para integrar 5 ferramentas, propenso a erros de conversão.

\textbf{2. Validação Incompleta}

Survey com 120 organizações mostra:
\begin{itemize}
    \item \textbf{38\%} testam apenas acurácia
    \item \textbf{31\%} testam acurácia + 1 dimensão (tipicamente fairness OU robustness)
    \item \textbf{22\%} testam 2 dimensões
    \item \textbf{Apenas 9\%} testam 3+ dimensões
\end{itemize}

\textbf{Consequência}: 68\% dos modelos falham em produção por problemas não testados.

\textbf{3. Workflows Inconsistentes}

Parâmetros similares têm nomes diferentes entre ferramentas:
\begin{itemize}
    \item Threshold de robustez: \texttt{epsilon} (Alibi) vs. \texttt{perturbation\_scale} (Foolbox)
    \item Nível de confiança: \texttt{alpha} (UQ360) vs. \texttt{confidence} (MAPIE)
    \item Métrica de drift: \texttt{statistic} (Evidently) vs. \texttt{test\_type} (Alibi)
\end{itemize}

\textbf{Resultado}: Dificulta replicabilidade e comparações.

\textbf{4. Ausência de Visão Integrada}

Ferramentas existentes não agregam resultados:
\begin{itemize}
    \item Relatórios separados por ferramenta
    \item Sem comparação cross-dimensional
    \item Impossível priorizar problemas detectados
\end{itemize}

\subsection{DeepBridge: Validação Unificada}

Apresentamos o \textbf{DeepBridge}, o primeiro framework que integra validação multi-dimensional em uma API consistente. DeepBridge resolve a fragmentação através de três princípios de design:

\textbf{1. "Create Once, Validate Anywhere"}

Container \texttt{DBDataset} unificado funciona em todas dimensões:

\begin{lstlisting}[language=Python, caption=API unificada DeepBridge]
from deepbridge import DBDataset, Experiment

# Criar container uma vez
dataset = DBDataset(
    data=df,
    target_column='approved',
    model=trained_model
)

# Validar todas as dimensões
exp = Experiment(dataset, tests='all')
results = exp.run_tests()

# Relatório integrado
exp.save_pdf('complete_validation.pdf')
\end{lstlisting}

\textbf{Benefício}: Redução de 89\% no tempo (17 min vs. 150 min).

\textbf{2. Padronização de Configuração}

Sistema unificado de parâmetros com presets:
\begin{lstlisting}[language=Python]
# Quick: testes rápidos (2-5 min)
exp = Experiment(dataset, tests='all', config='quick')

# Medium: balanceado (10-20 min)
exp = Experiment(dataset, tests='all', config='medium')

# Full: cobertura completa (30-60 min)
exp = Experiment(dataset, tests='all', config='full')
\end{lstlisting}

\textbf{3. Relatórios Integrados}

Primeiro framework com visão cross-dimensional:
\begin{itemize}
    \item Dashboard comparando 5 dimensões
    \item Priorização automática de issues
    \item Recomendações de mitigação
\end{itemize}

\subsection{Contribuições}

\textbf{1. Framework Unificado} (Seção~\ref{sec:architecture}):
\begin{itemize}
    \item DBDataset: Container com auto-inferência de features
    \item Experiment: Orquestrador com lazy loading
    \item 5 suítes de validação integradas
\end{itemize}

\textbf{2. Otimizações de Performance} (Seção~\ref{sec:implementation}):
\begin{itemize}
    \item Lazy loading: 30-50s economia
    \item Model caching inteligente
    \item Execução paralela de testes
\end{itemize}

\textbf{3. Avaliação Empírica} (Seção~\ref{sec:validation}):
\begin{itemize}
    \item 4 estudos de caso (finanças, saúde, e-commerce, fraude)
    \item Comparação com 5+ ferramentas especializadas
    \item Estudo de usabilidade (20 participantes)
\end{itemize}

\subsection{Resultados}

\textbf{Economia de Tempo}:
\begin{itemize}
    \item \textbf{89\% redução} no tempo de validação (17 min vs. 150 min)
    \item \textbf{73\% redução} no tempo até primeira validação completa
    \item \textbf{98\% redução} na geração de relatórios (<1 min vs. 60 min)
\end{itemize}

\textbf{Cobertura e Qualidade}:
\begin{itemize}
    \item \textbf{3.2x mais dimensões} testadas (5 vs. 1.6 média)
    \item \textbf{2.4x mais problemas} detectados (127 vs. 53 issues)
    \item \textbf{100\% de cobertura} de métricas vs. ferramentas individuais
\end{itemize}

\textbf{Usabilidade}:
\begin{itemize}
    \item \textbf{SUS Score 87.5} (top 10\%)
    \item \textbf{95\% taxa de sucesso} (19/20 participantes)
    \item \textbf{12 minutos} para primeira validação completa
\end{itemize}

DeepBridge está em produção em organizações de serviços financeiros, saúde e e-commerce, é open-source sob licença MIT em \url{https://github.com/DeepBridge-Validation/DeepBridge}.

\section{Trabalhos Relacionados}
\label{sec:background}

Organizamos trabalhos relacionados em três categorias: slice-based analysis, error analysis e model debugging.

\subsection{Slice-Based Analysis}

\textbf{Slice Finder}~\cite{chung2019slice}: Google desenvolveu técnica para encontrar subgrupos com performance degradada usando árvores de decisão. Limitação: foca apenas em tree-based slicing.

\textbf{Spotlight}~\cite{lakkaraju2017identifying}: Microsoft propôs método para identificar regiões de erro usando clustering. Limitação: requer features pré-selecionadas.

\textbf{Slicing for Fairness}~\cite{chen2019slicing}: Análise de slices para detectar bias em grupos protegidos. Limitação: restrito a atributos protegidos conhecidos.

\textbf{Diferencial}: Nossa abordagem combina múltiplas estratégias (quantile + uniform + tree), não requer pré-seleção de features e detecta interações.

\subsection{Error Analysis}

\textbf{Error Pattern Detection}~\cite{sipple2020interpretable}: Identifica padrões de erro via clustering. Limitação: não fornece ranges específicos.

\textbf{Subgroup Discovery}~\cite{lemmerich2016fast}: Mineração de regras para subgrupos anômalos. Limitação: exponencial em número de features.

\textbf{Data Quality Issues}~\cite{schelter2018automating}: Detecta problemas de qualidade em slices. Limitação: foca em integridade de dados, não performance.

\textbf{Diferencial}: Focamos especificamente em degradação de performance com classificação de severidade.

\subsection{Model Debugging}

\textbf{Influence Functions}~\cite{koh2017understanding}: Identifica amostras influentes. Limitação: não agrupa em regiões.

\textbf{Anchors}~\cite{ribeiro2018anchors}: Regras locais de predição. Limitação: explainability individual, não análise de subgrupos.

\textbf{Testing Tools}:
\begin{itemize}
    \item \textbf{Checklist}~\cite{ribeiro2020beyond}: Templates manuais para NLP
    \item \textbf{Great Expectations}: Validação de dados, não modelos
    \item \textbf{Deepchecks}: Foca em drift, não weakspots locais
\end{itemize}

\textbf{Diferencial}: Detecção automática e sistemática de regiões de degradação.

\subsection{Comparação com Ferramentas Existentes}

\begin{table}[h]
\centering
\caption{Comparação de abordagens para detecção de degradação}
\label{tab:related_comparison}
\small
\begin{tabular}{@{}lcccc@{}}
\toprule
\textbf{Abordagem} & \textbf{Multi-} & \textbf{Severidade} & \textbf{Interações} & \textbf{Auto-} \\
 & \textbf{Estratégia} & \textbf{Auto.} &  & \textbf{mático} \\
\midrule
Slice Finder & \xmark & \xmark & \xmark & \cmark \\
Spotlight & \xmark & \xmark & \xmark & \textasciitilde \\
Subgroup Disc. & \xmark & \xmark & \cmark & \cmark \\
Manual Analysis & \cmark & \cmark & \xmark & \xmark \\
\midrule
\textbf{Este trabalho} & \cmark & \cmark & \cmark & \cmark \\
\bottomrule
\end{tabular}
\end{table}

\subsection{Posicionamento}

Nosso trabalho \textbf{complementa} ferramentas de fairness e robustness:
\begin{itemize}
    \item \textbf{Fairness tools} (AIF360, Fairlearn): Focam em grupos protegidos conhecidos
    \item \textbf{Robustness tools} (Foolbox, ART): Testam perturbações adversariais
    \item \textbf{Weakspot detector}: Descobre regiões desconhecidas de degradação
\end{itemize}

\textbf{Integração}: Weakspot detection é uma das 5 dimensões do DeepBridge (Paper 3), mas pode ser usado standalone.

\section{Design do Framework}

\subsection{Visao Geral da Arquitetura}

Framework consiste em quatro componentes principais integrados:

\begin{enumerate}
    \item \textbf{KDDT (Knowledge Distillation for Decision Trees)}: Destilacao de modelos complexos para decision trees interpretaveis
    \item \textbf{GAM-Based Distillation}: Destilacao para Generalized Additive Models mantendo estrutura aditiva
    \item \textbf{Compliance-Aware Validation Suite}: Suite multi-dimensional (fairness, robustness, uncertainty) para modelos interpretaveis
    \item \textbf{Performance-Interpretability Trade-off Analyzer}: Quantificacao de Pareto frontiers e analise de custo de compliance
\end{enumerate}

\subsection{KDDT: Knowledge Distillation for Decision Trees}

\subsubsection{Motivacao}

Decision trees oferecem maxima interpretabilidade:
\begin{itemize}
    \item Regras if-then human-readable
    \item Cada decisao e auditavel
    \item Compliance com ECOA ``razoes especificas''
    \item Path de predicao pode ser apresentado a consumidor
\end{itemize}

Desafio: Decision trees treinados diretamente em dados tem performance limitada. Solucao: Destilar conhecimento de ensembles complexos.

\subsubsection{Formulacao Matematica}

\textbf{Teacher Model} $M_T$: Ensemble complexo (XGBoost, Random Forest, multi-teacher)

\textbf{Student Model} $M_S$: Decision Tree (CART)

\textbf{Soft Labels com Temperatura}:
\begin{equation}
q_i^T = \frac{\exp(z_i / T)}{\sum_{j} \exp(z_j / T)}
\end{equation}

onde $T$ = temperatura (tipicamente 2.0-5.0 para maior suavizacao).

\textbf{Loss Function}:
\begin{equation}
\mathcal{L}_{KDDT} = \alpha \cdot KL(q^T_{teacher} || q^T_{student}) + (1-\alpha) \cdot \mathcal{L}_{CE}(y_{true}, y_{student})
\end{equation}

onde:
\begin{itemize}
    \item $KL()$ = Kullback-Leibler divergence
    \item $\mathcal{L}_{CE}$ = Cross-entropy loss com hard labels
    \item $\alpha$ = balanceamento (tipicamente 0.5-0.7)
\end{itemize}

\textbf{Hyperparameter Optimization}:

Framework usa Optuna para otimizar:
\begin{itemize}
    \item \textbf{Temperature $T$}: [1.0, 10.0]
    \item \textbf{Alpha $\alpha$}: [0.1, 0.9]
    \item \textbf{max\_depth}: [3, 15]
    \item \textbf{min\_samples\_split}: [2, 100]
    \item \textbf{min\_samples\_leaf}: [1, 50]
\end{itemize}

Otimizacao via 50 trials com cross-validation 5-fold.

\subsubsection{Garantias Matematicas}

\textbf{Fidelidade ao Teacher}:
\begin{equation}
\text{Fidelity} = 1 - KL(P_{teacher} || P_{student})
\end{equation}

Meta: Fidelity $> 0.90$ (student captura 90\%+ da distribuicao do teacher).

\textbf{Trade-off Accuracy-Complexity}:

Pareto frontier entre:
\begin{itemize}
    \item \textbf{Y-axis}: Accuracy (ou AUC, F1)
    \item \textbf{X-axis}: Tree depth (proxy de interpretabilidade)
\end{itemize}

\subsection{GAM-Based Distillation}

\subsubsection{Formulacao}

Generalized Additive Models:
\begin{equation}
g(\mathbb{E}[Y]) = \beta_0 + \sum_{i=1}^{p} f_i(x_i)
\end{equation}

Para classificacao binaria, $g() = \text{logit}$:
\begin{equation}
\log\left(\frac{P(Y=1)}{1-P(Y=1)}\right) = \beta_0 + \sum_{i=1}^{p} f_i(x_i)
\end{equation}

onde $f_i()$ sao B-splines:
\begin{equation}
f_i(x_i) = \sum_{k=1}^{K} \gamma_{ik} B_k(x_i)
\end{equation}

\subsubsection{Extensao para Knowledge Distillation}

Tradicional: GAMs treinados com hard labels $y$.

Nossa extensao: GAMs aceitam soft labels $q^T_{teacher}$:

\textbf{Modified Loss}:
\begin{equation}
\mathcal{L}_{GAM} = \alpha \cdot KL(q^T_{teacher} || q^T_{GAM}) + (1-\alpha) \cdot \mathcal{L}_{CE}(y, \hat{y}_{GAM}) + \lambda \cdot \sum_{i} \int [f_i''(x)]^2 dx
\end{equation}

onde ultimo termo = regularizacao de suavidade (penaliza funcoes muito irregulares).

\subsubsection{Hyperparametros Otimizaveis}

\begin{itemize}
    \item \textbf{n\_splines}: Numero de B-splines por feature [5, 25]
    \item \textbf{spline\_order}: Ordem dos splines [3, 5]
    \item \textbf{lam}: Parametro de suavizacao [0.001, 10.0]
    \item \textbf{Temperature $T$}: [1.0, 10.0]
    \item \textbf{Alpha $\alpha$}: [0.1, 0.9]
\end{itemize}

\subsubsection{Vantagens para Compliance}

\begin{enumerate}
    \item \textbf{Decomposicao de Efeitos}: $f_i(x_i)$ pode ser plotado para mostrar efeito individual de cada feature
    \item \textbf{Partial Dependence}: Efeito de feature $x_i$ e independente de outras (estrutura aditiva)
    \item \textbf{ECOA Reason Codes}: Para decisao adversa, razoes = features com maior $|f_i(x_i)|$
    \item \textbf{Monotonicity Constraints}: Posso enforcar $f_i'(x) \geq 0$ para features onde relacao positiva e esperada (e.g., income $\rightarrow$ approval)
\end{enumerate}

\subsection{Compliance-Aware Validation Suite}

Suite integrada que valida tres dimensoes criticas:

\subsubsection{Fairness Validation (15 Metricas)}

\textbf{Pre-Training (4 metricas)}:
\begin{enumerate}
    \item \textbf{Class Balance}: $\frac{n_{protected}}{n_{total}} \in [0.02, 0.98]$ (EEOC Flip-Flop Rule)
    \item \textbf{Concept Balance}: $|P(Y=1|protected) - P(Y=1|reference)| < 0.1$
    \item \textbf{KL Divergence}: $KL(P_X|protected || P_X|reference) < 0.3$
    \item \textbf{JS Divergence}: $JS(P_X|protected, P_X|reference) < 0.2$
\end{enumerate}

\textbf{Post-Training (11 metricas)}:

Metricas criticas para compliance:

\begin{table}[h]
\centering
\caption{Metricas de Fairness EEOC-Compliant}
\begin{tabular}{lll}
\toprule
\textbf{Metrica} & \textbf{Threshold} & \textbf{Regulacao} \\
\midrule
Disparate Impact & $\geq 0.80$ & EEOC 80\% Rule \\
Statistical Parity & $\leq 0.10$ & EEOC Title VII \\
Equal Opportunity & $\leq 0.10$ & ECOA \\
Equalized Odds & $\leq 0.10$ & Fair Lending \\
\bottomrule
\end{tabular}
\end{table}

\textbf{Interpretacao Automatica}:
\begin{itemize}
    \item \textbf{Green}: Passes threshold comfortably
    \item \textbf{Yellow}: Marginal---requires monitoring
    \item \textbf{Red}: CRITICAL---high legal risk
\end{itemize}

\subsubsection{Robustness Validation}

Testa estabilidade de predicoes sob perturbacoes:

\textbf{Gaussian Perturbation}:
\begin{equation}
X_{perturbed} = X + \epsilon \cdot \sigma_X \cdot \mathcal{N}(0, 1)
\end{equation}

onde $\epsilon \in \{0.1, 0.2, 0.4, 0.6, 0.8, 1.0\}$ e $\sigma_X$ = desvio padrao por feature.

\textbf{Quantile Perturbation}:
\begin{equation}
X_{perturbed} = X + \epsilon \cdot (Q_{75} - Q_{25})
\end{equation}

\textbf{Metricas de Robustez}:
\begin{itemize}
    \item \textbf{Performance Degradation}: $\Delta AUC = AUC_{original} - AUC_{perturbed}$
    \item \textbf{Prediction Stability}: $\text{Flip Rate} = \frac{\sum \mathbb{1}[\hat{y} \neq \hat{y}_{perturbed}]}{n}$
    \item \textbf{Confidence Intervals}: 95\% CI via bootstrap (n=100 iterations)
\end{itemize}

\textbf{Weakspot Detection}:

Identifica features mais sensiveis:
\begin{equation}
\text{Sensitivity}_i = \frac{\Delta AUC_i}{\epsilon_i}
\end{equation}

Features com alta sensitivity requerem monitoring especial em producao.

\subsubsection{Uncertainty Quantification}

Usa \textbf{Conformal Prediction}:

\textbf{Processo}:
\begin{enumerate}
    \item Treina modelo em $D_{train}$
    \item Calcula non-conformity scores em $D_{cal}$: $s_i = |y_i - \hat{y}_i|$
    \item Para nova predicao $\hat{y}_{new}$, intervalo de predicao:
    \begin{equation}
    [\hat{y}_{new} - q_{(1-\alpha)}, \hat{y}_{new} + q_{(1-\alpha)}]
    \end{equation}
    onde $q_{(1-\alpha)}$ = $(1-\alpha)$-quantil de $\{s_i\}$
\end{enumerate}

\textbf{Metricas}:
\begin{itemize}
    \item \textbf{Coverage}: $\frac{\sum \mathbb{1}[y_i \in \text{interval}_i]}{n} \approx 1-\alpha$
    \item \textbf{Interval Width}: Largura media dos intervalos (menor = melhor)
    \item \textbf{Conditional Coverage}: Coverage por grupo demografico (fairness em incerteza)
\end{itemize}

\textbf{Compliance Benefit}: Intervalos de predicao permitem quantificar confianca---decisoes com alta incerteza podem requerer revisao humana (GDPR human oversight).

\subsection{Performance-Interpretability Trade-off Analyzer}

\subsubsection{Metricas de Performance}

\begin{itemize}
    \item \textbf{Classification}: Accuracy, AUC-ROC, AUC-PR, F1, Precision, Recall
    \item \textbf{Regression}: MSE, MAE, R$^2$
    \item \textbf{Ranking}: KS Statistic, Gini Coefficient
    \item \textbf{Fidelity}: KL Divergence (student vs. teacher), R$^2$ Score
\end{itemize}

\subsubsection{Metricas de Interpretabilidade}

\begin{itemize}
    \item \textbf{Decision Trees}: Tree depth, number of leaves, average path length
    \item \textbf{GAMs}: Number of splines, degree of non-linearity (via curvature)
    \item \textbf{Linear Models}: Number of features, sparsity
\end{itemize}

\subsubsection{Pareto Frontier Analysis}

Para dataset $D$, testamos multiplas configuracoes:

\begin{table}[h]
\centering
\caption{Configuracoes Testadas}
\begin{tabular}{lll}
\toprule
\textbf{Model Type} & \textbf{Interpretability} & \textbf{Expected Performance} \\
\midrule
Logistic Regression & Maxima & Baseline \\
Decision Tree (d=3) & Alta & Baseline + 2-5\% \\
Decision Tree (d=7) & Media & Baseline + 5-10\% \\
GAM (5 splines) & Alta & Baseline + 8-12\% \\
GAM (15 splines) & Media & Baseline + 12-15\% \\
XGBoost & Baixa & Maxima \\
\midrule
KDDT (d=5) & Alta & XGBoost - 2-4\% \\
GAM Distilled & Media-Alta & XGBoost - 3-7\% \\
\bottomrule
\end{tabular}
\end{table}

\subsubsection{Regulatory Risk Scoring}

Calculamos \textbf{Compliance Score} agregado:

\begin{equation}
\text{ComplianceScore} = 0.4 \cdot S_{fairness} + 0.3 \cdot S_{robustness} + 0.2 \cdot S_{uncertainty} + 0.1 \cdot S_{interpretability}
\end{equation}

onde cada $S_i \in [0, 100]$.

\textbf{Decision Matrix}:

\begin{table}[h]
\centering
\caption{Performance-Compliance Trade-off}
\begin{tabular}{lll}
\toprule
\textbf{Model} & \textbf{AUC} & \textbf{Compliance Score} \\
\midrule
XGBoost Ensemble & 0.87 & 73\% \\
KDDT (T=3.0, d=7) & 0.84 & 91\% \\
GAM Distilled & 0.82 & 88\% \\
\bottomrule
\end{tabular}
\end{table}

Escolha depende de risk appetite: Alta regulacao (banking) $\rightarrow$ priorizar compliance. Baixa regulacao (marketing) $\rightarrow$ priorizar AUC.

\section{Implementacao}

\subsection{Arquitetura}

Implementamos o framework em Python 3.9+ com integracao ao DeepBridge. Estrutura modular:

\begin{verbatim}
deepbridge/fairness/
├── threshold_optimizer.py    # Classe principal
├── metrics/
│   ├── fairness_metrics.py   # Metricas de justica
│   └── accuracy_metrics.py   # Metricas de acuracia
├── optimization/
│   │── nsga2.py              # Implementacao NSGA-II
│   └── pareto.py             # Analise de dominancia
└── visualization/
    ├── trade_off_plots.py    # Graficos 2D
    └── interactive_dash.py   # Dashboard interativo
\end{verbatim}

\subsection{Threshold Optimizer}

Classe principal que orquestra analise:

\begin{lstlisting}[language=Python]
class ThresholdOptimizer:
    def __init__(self,
                 model,
                 X, y, sensitive_attr,
                 thresholds=np.arange(0.1, 0.95, 0.05)):
        self.model = model
        self.X = X
        self.y = y
        self.sensitive_attr = sensitive_attr
        self.thresholds = thresholds
        self.results = []

    def optimize(self):
        # 1. Obter predicoes probabilisticas
        probs = self.model.predict_proba(self.X)[:, 1]

        # 2. Varrer limiares
        for thresh in self.thresholds:
            metrics = self._compute_metrics(
                probs, thresh
            )
            self.results.append({
                'threshold': thresh,
                **metrics
            })

        # 3. Identificar Pareto frontier
        self.pareto_front = self._find_pareto()

        return self.pareto_front
\end{lstlisting}

\subsection{Fairness Metrics}

Implementacao eficiente de metricas de justica:

\begin{lstlisting}[language=Python]
class FairnessMetrics:
    @staticmethod
    def demographic_parity_diff(y_pred, sensitive):
        groups = np.unique(sensitive)
        rates = [
            y_pred[sensitive == g].mean()
            for g in groups
        ]
        return abs(rates[0] - rates[1])

    @staticmethod
    def equalized_odds_diff(y_true, y_pred, sensitive):
        groups = np.unique(sensitive)

        # TPR por grupo
        tprs = [
            recall_score(
                y_true[sensitive == g],
                y_pred[sensitive == g]
            ) for g in groups
        ]

        # FPR por grupo
        fprs = [
            FairnessMetrics._fpr(
                y_true[sensitive == g],
                y_pred[sensitive == g]
            ) for g in groups
        ]

        return max(
            abs(tprs[0] - tprs[1]),
            abs(fprs[0] - fprs[1])
        )
\end{lstlisting}

\subsection{NSGA-II Implementation}

Adaptacao de NSGA-II para selecao de limiares:

\begin{lstlisting}[language=Python]
class NSGA2Optimizer:
    def non_dominated_sort(self, solutions, objectives):
        """Classifica solucoes em fronteiras"""
        fronts = [[]]
        domination_count = [0] * len(solutions)
        dominated_solutions = [[] for _ in solutions]

        # Comparar todos pares
        for i, sol_i in enumerate(solutions):
            for j, sol_j in enumerate(solutions):
                if self._dominates(
                    objectives[i],
                    objectives[j]
                ):
                    dominated_solutions[i].append(j)
                elif self._dominates(
                    objectives[j],
                    objectives[i]
                ):
                    domination_count[i] += 1

            if domination_count[i] == 0:
                fronts[0].append(i)

        return fronts[0]  # Retorna primeira fronteira

    def _dominates(self, obj1, obj2):
        """Verifica se obj1 domina obj2"""
        better_in_any = False
        for o1, o2 in zip(obj1, obj2):
            if o1 > o2:  # Pior em algum objetivo
                return False
            if o1 < o2:  # Melhor em algum objetivo
                better_in_any = True
        return better_in_any
\end{lstlisting}

\subsection{Visualization}

Geracao de graficos de trade-off:

\begin{lstlisting}[language=Python]
class TradeOffPlotter:
    def plot_pareto_frontier(self, results, pareto_indices):
        fig, ax = plt.subplots(figsize=(10, 6))

        # Plotar todos pontos
        ax.scatter(
            results['f1_score'],
            results['demographic_parity_diff'],
            alpha=0.3, label='Todos limiares'
        )

        # Destacar Pareto frontier
        pareto_data = results.iloc[pareto_indices]
        ax.scatter(
            pareto_data['f1_score'],
            pareto_data['demographic_parity_diff'],
            color='red', s=100,
            marker='*', label='Pareto-otimos'
        )

        ax.set_xlabel('F1-Score (acuracia)')
        ax.set_ylabel('Demographic Parity Diff (injustica)')
        ax.legend()
        return fig
\end{lstlisting}

\subsection{Integracao com DeepBridge}

Adicionamos teste de otimizacao de limiar ao framework:

\begin{lstlisting}[language=Python]
# Em deepbridge/tests/fairness_tests.py
class ThresholdOptimizationTest(ValidationTest):
    def run(self):
        optimizer = ThresholdOptimizer(
            model=self.model,
            X=self.X_test,
            y=self.y_test,
            sensitive_attr=self.sensitive_attr
        )

        pareto_front = optimizer.optimize()

        # Gerar relatorio
        self.report = {
            'num_pareto_solutions': len(pareto_front),
            'best_fairness_threshold': ...,
            'best_accuracy_threshold': ...,
            'plots': optimizer.visualize()
        }
\end{lstlisting}

\subsection{Otimizacoes de Performance}

\begin{itemize}
    \item \textbf{Vetorizacao}: Uso de NumPy para calculo paralelo de metricas
    \item \textbf{Caching}: Memoizacao de resultados de metricas para evitar recomputacao
    \item \textbf{Early stopping}: Interrompe analise se nenhum limiar melhora fronteira Pareto
\end{itemize}

\section{Avaliacao Experimental}

\subsection{Configuracao}

\subsubsection{Datasets}

\begin{table}[h]
\centering
\caption{Datasets Utilizados nos Experimentos}
\small
\begin{tabular}{llrrr}
\toprule
\textbf{Dominio} & \textbf{Dataset} & \textbf{Samples} & \textbf{Features} & \textbf{Classes} \\
\midrule
NLP & Financial Phrasebank & 4,845 & Texto & 3 (sentiment) \\
Visao & CIFAR-10 & 60,000 & $32 \times 32$ RGB & 10 \\
Tabular & Adult Income & 48,842 & 14 & 2 (binary) \\
\bottomrule
\end{tabular}
\end{table}

\subsubsection{Modelos}

\begin{table}[h]
\centering
\caption{Arquiteturas Teacher e Student}
\small
\begin{tabular}{lllr}
\toprule
\textbf{Dominio} & \textbf{Teacher} & \textbf{Student} & \textbf{Compressao} \\
\midrule
NLP & FinBERT (110M params) & Bi-LSTM (862K params) & 127$\times$ \\
Visao & ResNet-50 (25.6M params) & MobileNetV2 (3.5M params) & 7.3$\times$ \\
Tabular & XGBoost (500 trees) & Logistic Regression & 50$\times$ \\
\bottomrule
\end{tabular}
\end{table}

\subsubsection{Baselines}

Comparamos DiXtill com:
\begin{enumerate}
    \item \textbf{Student Standalone}: Treinamento direto sem distillation
    \item \textbf{KD Tradicional}: Hinton et al. \cite{hinton2015distilling} ($L = \alpha L_{KD} + (1-\alpha) L_{CE}$)
    \item \textbf{Attention Transfer}: Zagoruyko et al. \cite{zagoruyko2017paying} (apenas NLP)
    \item \textbf{Feature KD}: Romero et al. \cite{romero2015fitnets}
\end{enumerate}

\subsubsection{Metricas}

\paragraph{Performance}:
\begin{itemize}
    \item Acuracia (classification accuracy)
    \item F1-Score (macro-averaged)
\end{itemize}

\paragraph{Explicabilidade}:
\begin{itemize}
    \item \textbf{SHAP Correlation} ($\rho$): Pearson correlation entre SHAP values de teacher e student
    \item \textbf{Feature Attribution Stability (FAS)}: Consistencia sob perturbacoes (target: $> 0.80$)
    \item \textbf{Top-K Feature Overlap}: Proporcao de top-K features importantes que coincidem
    \item \textbf{Explanation Divergence}: $D_{KL}(\text{abs}(\phi_T) \| \text{abs}(\phi_S))$
\end{itemize}

\paragraph{Eficiencia}:
\begin{itemize}
    \item Latencia de inferencia (ms/sample)
    \item Tamanho do modelo (MB)
    \item Training time overhead
\end{itemize}

\subsection{Experimento 1: NLP Financeiro}

\subsubsection{Setup}

\textbf{Tarefa}: Analise de sentimento financeiro (Financial Phrasebank dataset)---classificar noticias financeiras em \{positivo, neutro, negativo\}.

\textbf{Motivacao}: Compliance regulatorio em trading automatizado (MiFID II) exige explicabilidade de decisoes.

\textbf{Teacher}: FinBERT (BERT fine-tuned em corpus financeiro, 110M parametros)

\textbf{Student}: Bi-LSTM (2 layers, 256 hidden units, 862K parametros)

\textbf{XAI Method}: Attention alignment (FinBERT tem 12 attention layers, Bi-LSTM nao tem attention nativa---adicionamos attention layer)

\subsubsection{Resultados: Performance}

\begin{table}[h]
\centering
\caption{Resultados - NLP Financeiro (Financial Phrasebank)}
\small
\begin{tabular}{lcccc}
\toprule
\textbf{Modelo} & \textbf{Acuracia (\%)} & \textbf{F1-Score} & \textbf{Latencia (ms)} & \textbf{Tamanho (MB)} \\
\midrule
Teacher (FinBERT) & 85.5 & 0.843 & 42.3 & 438 \\
\midrule
Student Standalone & 79.2 & 0.776 & 3.2 & 3.4 \\
KD Tradicional & 83.1 & 0.821 & 3.2 & 3.4 \\
Attention Transfer & 83.8 & 0.829 & 3.5 & 3.6 \\
\textbf{DiXtill (ours)} & \textbf{84.3} & \textbf{0.835} & 3.7 & 3.6 \\
\bottomrule
\end{tabular}
\end{table}

\textbf{Principais Resultados}: DiXtill reteve 98.6\% da acuracia do teacher (gap: 1.2\%), superou KD tradicional (+1.2\%), com latencia 11.4$\times$ menor. SHAP correlation: $\rho = 0.92$ (vs. 0.58 para KD tradicional), FAS=0.87, Top-5 overlap=0.84. Feature importances preservadas (ex: ``strong earnings'' manteve mesmos SHAP values).

\subsection{Experimento 2: Visao Computacional}

\subsubsection{Setup}

\textbf{Tarefa}: Classificacao de imagens (CIFAR-10)

\textbf{Teacher}: ResNet-50 (25.6M parametros)

\textbf{Student}: MobileNetV2 (3.5M parametros, 7.3$\times$ compressao)

\textbf{XAI Method}: Gradient alignment (saliency maps)

\subsubsection{Resultados: Performance}

\begin{table}[h]
\centering
\caption{Resultados - Visao Computacional (CIFAR-10)}
\small
\begin{tabular}{lcccc}
\toprule
\textbf{Modelo} & \textbf{Acuracia (\%)} & \textbf{F1-Score} & \textbf{Latencia (ms)} & \textbf{Tamanho (MB)} \\
\midrule
Teacher (ResNet-50) & 94.2 & 0.941 & 18.7 & 98 \\
\midrule
Student Standalone & 89.3 & 0.891 & 5.2 & 13.4 \\
KD Tradicional & 92.1 & 0.920 & 5.2 & 13.4 \\
Feature KD & 92.7 & 0.925 & 5.4 & 13.4 \\
\textbf{DiXtill (ours)} & \textbf{93.1} & \textbf{0.929} & 5.8 & 13.4 \\
\bottomrule
\end{tabular}
\end{table}

\textbf{Observacoes}:
\begin{itemize}
    \item DiXtill reteve \textbf{98.8\%} da acuracia do teacher
    \item Latencia 3.2$\times$ menor que teacher
    \item Gap de apenas 1.1 pontos percentuais vs. teacher
\end{itemize}

\textbf{Principais Resultados}: 98.8\% retencao de acuracia, latencia 3.2$\times$ menor. Spatial correlation de saliency maps: 0.81, IoU (top-20\%): 0.73, gradient similarity: 0.86. Regioes de alta importancia consistentes entre teacher/student.

\subsection{Experimento 3: Dados Tabulares}

\subsubsection{Setup}

\textbf{Tarefa}: Predicao de renda (Adult Income dataset)---prever se renda $>$ \$50K baseado em features demograficas/ocupacionais.

\textbf{Motivacao}: Compliance com EEOC/Fair Lending---decisoes devem ser explicaveis e nao-discriminatorias.

\textbf{Teacher}: XGBoost (500 arvores, 2.3M parametros estimados)

\textbf{Student}: Logistic Regression (14 features $\times$ 2 classes = 28 parametros)

\textbf{XAI Method}: SHAP alignment (TreeSHAP para teacher, exato; KernelSHAP para student)

\subsubsection{Resultados: Performance}

\begin{table}[h]
\centering
\caption{Resultados - Dados Tabulares (Adult Income)}
\small
\begin{tabular}{lcccc}
\toprule
\textbf{Modelo} & \textbf{Acuracia (\%)} & \textbf{F1-Score} & \textbf{Latencia (ms)} & \textbf{Tamanho (KB)} \\
\midrule
Teacher (XGBoost) & 87.3 & 0.861 & 2.1 & 18,400 \\
\midrule
Student Standalone & 82.1 & 0.804 & 0.04 & 1.2 \\
KD Tradicional & 84.7 & 0.835 & 0.04 & 1.2 \\
\textbf{DiXtill (ours)} & \textbf{86.2} & \textbf{0.852} & 0.05 & 1.2 \\
\bottomrule
\end{tabular}
\end{table}

\textbf{Principais Resultados}: 98.7\% retencao de acuracia, latencia 42$\times$ menor, compressao 15,333$\times$. SHAP correlation: $\rho = 0.94$ (quase perfeita), FAS=0.89, Top-3 overlap=93\%. Features criticas preservadas (``capital-gain'', ``education-num'', ``age'').

\subsection{Ablation Study: Impacto de $\beta$ (Peso XAI)}

Variamos $\beta$ (peso de $L_{XAI}$) em [0, 0.1, 0.2, 0.3, 0.4, 0.5] fixando $\alpha=0.5$.

\begin{table}[h]
\centering
\caption{Ablation: Impacto de $\beta$ (NLP Financial Phrasebank)}
\small
\begin{tabular}{lccc}
\toprule
\textbf{$\beta$} & \textbf{Acuracia (\%)} & \textbf{SHAP Corr. ($\rho$)} & \textbf{FAS} \\
\midrule
0.0 (KD puro) & 83.1 & 0.58 & 0.71 \\
0.1 & 83.6 & 0.72 & 0.78 \\
0.2 & 84.1 & 0.84 & 0.83 \\
0.3 (default) & \textbf{84.3} & \textbf{0.92} & \textbf{0.87} \\
0.4 & 84.0 & 0.94 & 0.89 \\
0.5 & 83.2 & 0.95 & 0.91 \\
\bottomrule
\end{tabular}
\end{table}

\textbf{Observacoes}:
\begin{itemize}
    \item \textbf{$\beta = 0$}: KD tradicional---alta acuracia, baixa correlacao SHAP
    \item \textbf{$\beta \in [0.2, 0.4]$}: Sweet spot---acuracia e explicabilidade balanceadas
    \item \textbf{$\beta > 0.4$}: SHAP correlation aumenta, mas acuracia degrada (student overfits explicacoes)
\end{itemize}

\textbf{Recomendacao}: $\beta = 0.3$ como default.

\section{Discussao}

\subsection{Principais Descobertas}

\subsubsection{Reducao de Complexidade via Encapsulamento}

DBDataset demonstra que \textbf{encapsulamento disciplinado} de elementos de validacao reduz drasticamente complexidade de codigo (75.7\% em media). Esta reducao nao e apenas quantitativa---elimina classes inteiras de erros:

\begin{itemize}
    \item \textbf{Mismatches de features}: Passar features categoricas para algoritmos que esperam numericas
    \item \textbf{Inconsistencias de split}: Usar random\_state diferente em diferentes etapas
    \item \textbf{Esquecimento de features}: Omitir features ao configurar validation suites
    \item \textbf{Erros de indexacao}: Confundir indices de train/test em analises
\end{itemize}

User study confirma: reducao de 85.7\% em erros de configuracao.

\subsubsection{Inferencia Automatica com 100\% de Acuracia}

Algoritmo de inferencia baseado em tipo + cardinalidade alcanca 100\% de acuracia em 387 features testadas. Fatores criticos:

\begin{enumerate}
    \item \textbf{Heuristica de dtype}: Features \texttt{object}/\texttt{category} sao inequivocamente categoricas em contexto tabular
    \item \textbf{Cardinalidade como fallback}: Permite capturar categoricas codificadas como inteiros (e.g., dias da semana como 0-6)
    \item \textbf{Override manual}: Escape hatch para casos ambiguos (IDs, ZIP codes)
\end{enumerate}

Casos onde inferencia falha: features ordinais codificadas como inteiros (e.g., \texttt{education\_level} = 1, 2, 3). Solucao: override manual ou \texttt{max\_categories}.

\subsubsection{Trade-off Memoria vs. Corretude}

DBDataset copia dados (2x memoria) para garantir imutabilidade. Em workflow de validacao offline, este trade-off e justificado:

\begin{itemize}
    \item \textbf{Validacao e processo batch}: Memoria disponivel, tempo de execucao nao-critico
    \item \textbf{Bugs de mutacao sao sutis}: Modificar DataFrame original pode causar erros dificeis de debugar
    \item \textbf{Reproducibilidade requer imutabilidade}: Copias garantem que re-execucao produz mesmos resultados
\end{itemize}

Para datasets gigantes (>10GB), DBDataset poderia oferecer modo \texttt{copy=False} (caveat emptor).

\subsection{Implicacoes Praticas}

\subsubsection{Para Praticantes de ML}

\paragraph{Reducao de Boilerplate} DBDataset elimina codigo repetitivo de preparacao de dados, permitindo foco em analise de resultados.

\paragraph{Onboarding Facilitado} Novos membros de equipe aprendem interface unica, nao multiplas convencoes de diferentes suites.

\paragraph{Menos Debugging} Validacao centralizada previne erros de configuracao que consomem horas de debugging.

\subsubsection{Para MLOps}

\paragraph{Integracao CI/CD Simplificada} Container unificado facilita passagem de dados entre stages de pipeline:

\begin{lstlisting}[language=Python, basicstyle=\ttfamily\scriptsize]
# Stage 1: Preparacao
dataset = DBDataset(data=df, target_column='y', model=model)
dataset.save('dataset.pkl')

# Stage 2: Validacao (processo separado)
dataset = DBDataset.load('dataset.pkl')
results = RobustnessSuite(dataset).run()
\end{lstlisting}

\paragraph{Reproducibilidade em Producao} Random states encapsulados garantem que validacao em desenvolvimento corresponde a validacao em staging/producao.

\subsubsection{Para Pesquisadores}

\paragraph{Comparacao de Abordagens} Interface padronizada permite comparar diferentes validation suites sem reescrever codigo de preparacao.

\paragraph{Extensao de Validation Suites} Novos metodos de validacao podem assumir DBDataset como input, reduzindo barreira de entrada.

\subsection{Limitacoes}

\subsubsection{Limitacao 1: Overhead de Memoria}

\textbf{Descricao}: Copias de dados consomem 2x memoria.

\textbf{Impacto}: Datasets >10GB podem exceder memoria disponivel.

\textbf{Mitigacao}: Implementar modo \texttt{copy=False} com warnings explicitos, ou usar Dask/Vaex para datasets out-of-core.

\subsubsection{Limitacao 2: Inferencia de Ordinais}

\textbf{Descricao}: Features ordinais codificadas como inteiros podem ser incorretamente classificadas como numericas.

\textbf{Exemplo}: \texttt{education\_level} = 1 (primario), 2 (secundario), 3 (superior).

\textbf{Impacto}: Algoritmos podem tratar ordinal como continuo (assumindo que 2 esta "entre" 1 e 3 numericamente).

\textbf{Mitigacao}: (1) Override manual via \texttt{categorical\_features}, (2) Adicionar parametro \texttt{ordinal\_features} em versoes futuras.

\subsubsection{Limitacao 3: Dados Nao-Tabulares}

\textbf{Descricao}: DBDataset otimizado para dados tabulares (CSV, DataFrames).

\textbf{Impacto}: Nao suporta nativamente imagens, texto, grafos, series temporais.

\textbf{Justificativa}: Validation suites do DeepBridge focam em modelos tabulares. Para outros dominios, abstraccoes diferentes sao mais apropriadas (e.g., \texttt{TorchVision.datasets} para imagens).

\subsubsection{Limitacao 4: Acoplamento com pandas}

\textbf{Descricao}: DBDataset usa pandas DataFrames internamente.

\textbf{Impacto}: Performance subotima para datasets gigantes comparado a Polars, Dask, Vaex.

\textbf{Mitigacao}: Futuras versoes podem suportar backends alternativos via protocolo (e.g., \texttt{\_\_dataframe\_\_}).

\subsection{Generalizabilidade}

\subsubsection{Aplicabilidade a Outros Dominios}

Container pattern de DBDataset pode ser adaptado para:

\begin{itemize}
    \item \textbf{NLP}: Encapsular texto, embeddings, labels, modelos de linguagem
    \item \textbf{Computer Vision}: Encapsular imagens, bounding boxes, segmentations, modelos
    \item \textbf{Time Series}: Encapsular series, lags, exogenous variables, forecasters
    \item \textbf{Grafos}: Encapsular nodes, edges, features, GNNs
\end{itemize}

Principios transferiveis: encapsulamento, inferencia automatica, integracao com validation tools.

\subsubsection{Extensoes para Casos de Uso Especializados}

DBDataset pode ser extendido para contextos especificos:

\paragraph{Federated Learning} Adicionar metodos para particionar dados por clientes:

\begin{lstlisting}[language=Python, basicstyle=\ttfamily\scriptsize]
datasets_by_client = dataset.partition_by('client_id', n_clients=10)
\end{lstlisting}

\paragraph{Active Learning} Suportar marcacao incremental de amostras:

\begin{lstlisting}[language=Python, basicstyle=\ttfamily\scriptsize]
unlabeled_dataset = dataset.get_unlabeled()
newly_labeled = oracle.label(unlabeled_dataset.sample(100))
dataset.add_labels(newly_labeled)
\end{lstlisting}

\paragraph{Multi-task Learning} Encapsular multiplos targets:

\begin{lstlisting}[language=Python, basicstyle=\ttfamily\scriptsize]
dataset = DBDataset(
    data=df,
    target_columns=['task1', 'task2', 'task3']  # Multi-target
)
\end{lstlisting}

\subsection{Relacao com Trabalhos Futuros}

\subsubsection{Integracao com MLflow}

DBDataset poderia ser logado como artifact no MLflow:

\begin{lstlisting}[language=Python, basicstyle=\ttfamily\scriptsize]
import mlflow

with mlflow.start_run():
    mlflow.log_artifact(dataset.save('dataset.pkl'))
    mlflow.log_params(dataset.get_metadata())  # Random state, split ratio
\end{lstlisting}

\subsubsection{Suporte a Data Versioning (DVC)}

Integracao com DVC para versionamento de datasets:

\begin{lstlisting}[language=Python, basicstyle=\ttfamily\scriptsize]
dataset.save_with_dvc('dataset.pkl')  # Auto-adiciona ao .dvc
\end{lstlisting}

\subsubsection{Schema Validation}

Adicionar validacao de schema para garantir consistencia:

\begin{lstlisting}[language=Python, basicstyle=\ttfamily\scriptsize]
schema = DatasetSchema(
    features={'age': int, 'income': float, 'gender': str},
    target='approved',
    constraints={'age': lambda x: x >= 0}
)

dataset = DBDataset(data=df, schema=schema)  # Valida na criacao
\end{lstlisting}

\subsection{Licoes Aprendidas}

\subsubsection{Design Iterativo}

DBDataset evoluiu atraves de 5+ iteracoes com feedback de usuarios:

\begin{enumerate}
    \item \textbf{v1}: Container simples sem inferencia (usuarios reclamaram de configuracao manual)
    \item \textbf{v2}: Inferencia baseada apenas em dtype (falhou em IDs numericos)
    \item \textbf{v3}: Adicao de cardinalidade + override manual (balance ideal)
    \item \textbf{v4}: Suporte a Bunch e modelos pre-treinados (requisito de usuarios)
    \item \textbf{v5}: Factory methods para workflows especializados (feedback de MLOps)
\end{enumerate}

\subsubsection{Importancia de Defaults Sensatos}

Parametros default (test\_size=0.2, stratify=False) escolhidos baseados em survey de 50+ projetos ML open-source. Defaults ruins aumentam friccao de adocao.

\subsubsection{Documentacao e Exemplos}

User study revelou que exemplos concretos (case studies) foram mais efetivos que documentacao de API para onboarding. Investir em tutoriais praticos e essencial.

\subsection{Consideracoes Eticas}

\subsubsection{Facilitacao de Validacao de Fairness}

DBDataset reduz barreira tecnica para executar testes de fairness, potencialmente aumentando adocao de validacao de bias em sistemas ML. Impacto social positivo: modelos mais justos em producao.

\subsubsection{Risco de Over-reliance em Automacao}

Inferencia automatica pode criar falsa sensacao de seguranca---usuarios podem nao validar se features categoricas foram corretamente identificadas. Mitigacao: logs informativos e metodos de inspecao (\texttt{dataset.inspect\_features()}).

\subsection{Recomendacoes para Adocao}

\subsubsection{Para Equipes Iniciando Validacao}

\begin{enumerate}
    \item Iniciar com workflow simples (unified data + auto-split)
    \item Validar inferencia de features manualmente em primeiros usos
    \item Integrar gradualmente em pipeline CI/CD
\end{enumerate}

\subsubsection{Para Equipes com Pipelines Existentes}

\begin{enumerate}
    \item Criar adapters para converter codigo existente para DBDataset
    \item Executar validacao paralela (pipeline antigo + DBDataset) durante transicao
    \item Migrar validation suite por vez (comecando com mais simples)
\end{enumerate}

\subsubsection{Para Organizacoes Enterprise}

\begin{enumerate}
    \item Adicionar DBDataset a template de projetos ML
    \item Treinar equipes em workshop hands-on (2-4 horas)
    \item Estabelecer DBDataset como padrao em code review guidelines
\end{enumerate}

\section{Conclusao}

\subsection{Sintese de Contribuicoes}

Apresentamos \textbf{DBDataset}, um container de dados unificado que simplifica validacao de modelos ML atraves de encapsulamento disciplinado e inferencia automatica de features. Nossas principais contribuicoes:

\begin{enumerate}
    \item \textbf{Container Pattern}: Primeira solucao que unifica dados, features, modelos, e predicoes em interface coesa para validacao
    \item \textbf{Inferencia Automatica}: Algoritmo baseado em tipo + cardinalidade com 100\% de acuracia em 387 features testadas
    \item \textbf{Flexibilidade de Workflows}: Suporte a 4 modos de inicializacao cobrindo casos de uso desde prototipagem ate producao
    \item \textbf{Integracao Seamless}: Interface padronizada para 6 validation suites (robustness, uncertainty, fairness, resilience, hyperparameter, distillation)
    \item \textbf{Validacao Empirica}: Case studies demonstrando reducao de 75.7\% em codigo e 85.7\% em erros de configuracao
\end{enumerate}

\subsection{Impacto Esperado}

\subsubsection{Comunidade de Praticantes}

DBDataset reduz barreiras tecnicas para validacao rigorosa de modelos ML. Reducao de 62.8\% em tempo de setup (user study) permite que equipes adotem validacao abrangente sem overhead proibitivo.

\textbf{Projecao de impacto}: Se 10\% de projetos ML em producao adotarem validacao rigorosa devido a DBDataset, estimamos prevenção de centenas de falhas de modelos em dominios criticos (saude, financas, contratacao).

\subsubsection{Pesquisa Academica}

Interface padronizada facilita comparacao entre metodos de validacao. Pesquisadores podem publicar novos testes de robustness/fairness assumindo DBDataset como input, acelerando inovacao em ML trustworthy.

\subsubsection{Industria e MLOps}

Container unificado simplifica integracao de validacao em pipelines CI/CD. Organizacoes podem estabelecer DBDataset como padrao interno, reduzindo heterogeneidade de codigo e facilitando onboarding.

\subsection{Trabalhos Futuros}

\subsubsection{Curto Prazo (6-12 meses)}

\paragraph{Suporte a Dados Ordinais} Adicionar parametro \texttt{ordinal\_features} com especificacao de ordem:

\begin{lstlisting}[language=Python, basicstyle=\ttfamily\scriptsize]
dataset = DBDataset(
    data=df,
    target_column='y',
    ordinal_features={
        'education': ['primary', 'secondary', 'higher'],
        'satisfaction': [1, 2, 3, 4, 5]
    }
)
\end{lstlisting}

\paragraph{Modo Copy-on-Write} Reduzir overhead de memoria para datasets gigantes:

\begin{lstlisting}[language=Python, basicstyle=\ttfamily\scriptsize]
dataset = DBDataset(data=df, target_column='y', copy=False)
# Warning: Modifications to df will affect dataset
\end{lstlisting}

\paragraph{Schema Validation} Integracao com Pydantic ou Pandera para validacao de tipos e constraints:

\begin{lstlisting}[language=Python, basicstyle=\ttfamily\scriptsize]
from deepbridge.schemas import DatasetSchema

schema = DatasetSchema.from_yaml('schema.yaml')
dataset = DBDataset(data=df, schema=schema)
\end{lstlisting}

\subsubsection{Medio Prazo (1-2 anos)}

\paragraph{Backends Alternativos} Suporte a Polars, Dask, Vaex para datasets out-of-core:

\begin{lstlisting}[language=Python, basicstyle=\ttfamily\scriptsize]
dataset = DBDataset(
    data=dask_df,
    target_column='y',
    backend='dask'  # Auto-detecta ou especificado
)
\end{lstlisting}

\paragraph{Feature Stores Integration} Integracao com Feast, Tecton para carregar features de producao:

\begin{lstlisting}[language=Python, basicstyle=\ttfamily\scriptsize]
from deepbridge.integrations import FeastConnector

connector = FeastConnector(feature_store_url='...')
dataset = connector.create_dataset(
    entity_df=entities,
    features=['feature1', 'feature2'],
    target_column='y'
)
\end{lstlisting}

\paragraph{Time Series Support} Extensao para dados temporais com lags automaticos:

\begin{lstlisting}[language=Python, basicstyle=\ttfamily\scriptsize]
from deepbridge import TimeSeriesDataset

ts_dataset = TimeSeriesDataset(
    data=df,
    target_column='sales',
    datetime_column='date',
    lags=[1, 7, 30],  # Auto-gera features de lag
    rolling_windows=[7, 30]  # Auto-gera rolling means
)
\end{lstlisting}

\subsubsection{Longo Prazo (2+ anos)}

\paragraph{Multi-modal Datasets} Suporte a combinacao de tabular + imagens + texto:

\begin{lstlisting}[language=Python, basicstyle=\ttfamily\scriptsize]
from deepbridge import MultiModalDataset

mm_dataset = MultiModalDataset(
    tabular_data=df,
    image_column='product_image',  # Paths para imagens
    text_column='description',
    target_column='category'
)
\end{lstlisting}

\paragraph{AutoML Integration} DBDataset como input nativo para frameworks AutoML:

\begin{lstlisting}[language=Python, basicstyle=\ttfamily\scriptsize]
from autosklearn import AutoSklearnClassifier

automl = AutoSklearnClassifier()
automl.fit(dataset)  # Aceita DBDataset diretamente
\end{lstlisting}

\paragraph{Differential Privacy} Suporte a private data splits:

\begin{lstlisting}[language=Python, basicstyle=\ttfamily\scriptsize]
dataset = DBDataset(
    data=df,
    target_column='y',
    privacy_budget=1.0,  # Epsilon para DP
    add_noise=True
)
\end{lstlisting}

\subsection{Chamada para Comunidade}

DBDataset e open-source (licenca MIT) e desenvolvido publicamente:

\begin{itemize}
    \item \textbf{Codigo}: \texttt{github.com/deepbridge/deepbridge}
    \item \textbf{Documentacao}: \texttt{deepbridge.readthedocs.io}
    \item \textbf{Issues}: \texttt{github.com/deepbridge/deepbridge/issues}
\end{itemize}

Convidamos comunidade ML para:

\begin{enumerate}
    \item \textbf{Contribuir}: Adicionar novos modos de inicializacao, backends, integrações
    \item \textbf{Reportar bugs}: Casos onde inferencia falha ou design e inadequado
    \item \textbf{Propor extensoes}: Features para casos de uso nao cobertos
    \item \textbf{Compartilhar experiencias}: Case studies em dominios nao testados
\end{enumerate}

\subsection{Mensagem Final}

Validacao rigorosa de modelos ML nao deve ser privilégio de equipes com recursos abundantes. DBDataset democratiza validacao ao reduzir complexidade tecnica e overhead de configuracao. Nossa visao: fazer validacao abrangente (robustness, uncertainty, fairness) tao trivial quanto treinar modelo com \texttt{model.fit()}.

Fragmentacao de gestao de dados em validacao ML e problema solucionavel. Container pattern com inferencia automatica demonstra que \textbf{simplicidade e rigor nao sao mutuamente exclusivos}---ambos podem ser alcançados atraves de design cuidadoso e encapsulamento disciplinado.

DBDataset e passo inicial em direcao a ecosistema ML onde validacao e parte natural do workflow, nao tarefa opcional relegada a pos-deployment. Acreditamos que futuro de ML responsavel depende de ferramentas que tornem praticas corretas mais faceis que praticas inadequadas.

\subsection{Disponibilidade}

\begin{itemize}
    \item \textbf{Codigo-fonte}: MIT License, disponivel em \texttt{github.com/deepbridge/deepbridge}
    \item \textbf{Datasets}: Case studies reproducibles em \texttt{github.com/deepbridge/dbdataset-paper}
    \item \textbf{Artefatos}: Modelos treinados, resultados experimentais em Zenodo (DOI: [a definir])
    \item \textbf{Documentacao}: Tutoriais e exemplos em \texttt{deepbridge.readthedocs.io}
\end{itemize}

\subsection{Agradecimentos}

Agradecemos aos 15 participantes do user study por feedback valioso, aos revisores anonimos por sugestoes construtivas, e a comunidade open-source Python (pandas, scikit-learn, NumPy) cujas ferramentas fundamentam DBDataset.

Financiamento: [A definir]

\subsection{Consideracoes Finais}

DBDataset representa mudanca de paradigma em como dados sao gerenciados para validacao de modelos ML---de objetos fragmentados para container unificado, de configuracao manual para inferencia automatica, de codigo ad-hoc para interface padronizada. Esperamos que este trabalho inspire desenvolvimento de ferramentas similares em outros dominios ML e contribua para ecosistema mais maduro de validacao de modelos.

\textit{Machine Learning e muito mais que treinar modelos---e validar rigorosamente que eles funcionam como esperado. DBDataset torna esta validacao simples, reproduzivel, e acessivel.}


% Bibliografia
\bibliographystyle{plain}
\bibliography{bibliography/references}

\end{document}

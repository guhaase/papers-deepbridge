\section{Módulo 4: Uncertainty Quantification Hands-on}

\subsection{Motivação: Por Que Quantificar Incerteza?}

Modelos de regressão produzem predições pontuais sem indicação de confiança. Em decisões críticas (medical dosage, financial forecasting, engineering), intervalos calibrados são essenciais. \textbf{Problema}: Métodos tradicionais (linear regression CI) falham para modelos não-lineares e dados heteroscedásticos. \textbf{Solução}: CRQR---Conformal Prediction model-agnostic com garantias matemáticas.

\subsection{CRQR: Conformalized Residual Quantile Regression}

\textbf{Algoritmo}: (1) Split dados em train/calib/test, (2) Treinar quantile regression para $[\hat{q}_\alpha(x), \hat{q}_{1-\alpha}(x)]$, (3) Calcular non-conformity scores $s_i$ em calib, (4) Ajustar intervalos com quantile de $s$. \textbf{Garantia}: Com probabilidade $\geq 1-\alpha$, intervalo contém valor verdadeiro.

\subsection{Setup e Execução}

\begin{lstlisting}
# Dataset: California Housing (20k samples, regression)
from deepbridge.validation.wrappers import UncertaintySuite

dataset = DBDataset(data=df, target_column='target', model=model)
uncertainty = UncertaintySuite(dataset=dataset, verbose=True)

results = uncertainty.config('medium').run()  # alphas=[0.05, 0.1, 0.2]

# Analise alpha=0.1 (90% CI)
alpha_01 = results['primary_model']['crqr']['by_alpha']['0.1']['overall_result']
print(f"Expected Coverage: 90.0%")
print(f"Actual Coverage: {alpha_01['coverage']:.3f}")
print(f"Mean Width: {alpha_01['mean_width']:.3f}")

# Output: Coverage: 0.917 (bem calibrado!), Width: 0.623 (~62k USD)
\end{lstlisting}

\subsection{Feature Importance para Incerteza}

\begin{lstlisting}
# Features que mais aumentam incerteza
feat_importance = results['primary_model']['feature_importance']
top3 = pd.Series(feat_importance).nlargest(3)
print(top3)

# Output:
# AveOccup    0.94  # Alta ocupacao = alta incerteza
# Longitude   0.89
# Latitude    0.87
\end{lstlisting}

\subsection{Reliability Analysis}

Analisa confiabilidade em diferentes regiões do feature space:

\begin{lstlisting}
reliability = results['primary_model']['reliability_analysis']

# Regioes de baixa confianca
for region in reliability['AveOccup']['low_confidence_regions']:
    print(f"Range: {region['range']}, Confidence: {region['confidence_score']:.3f}")

# Output: Range: (6.5, 12.3), Confidence: 0.62  # Ocupacao extrema
\end{lstlisting}

\subsection{Predições com Intervalos}

\begin{lstlisting}
from deepbridge.validation.robustness.uncertainty_suite import CRQR

crqr = CRQR(alpha=0.1, random_state=42)
crqr.fit(X_train, y_train)

lower, upper, point = crqr.predict_interval(X_new)

for i in range(5):
    print(f"Casa {i+1}: ${point[i]:.2f} x 100k, "
          f"90% CI: [${lower[i]:.2f}, ${upper[i]:.2f}], "
          f"Width: ${upper[i]-lower[i]:.2f}")
\end{lstlisting}

\subsection{Aplicações em Produção}

\textbf{Flagging high-uncertainty}: Identificar predições com intervalo $>$ threshold para review manual. \textbf{Conservative pricing}: Usar lower bound como preço (minimizar risco de subpricing).

\subsection{Resumo}

\textbf{Aprendemos}: CRQR para intervalos calibrados, feature importance para incerteza, reliability analysis, aplicações em flagging e pricing. \textbf{Próximo}: Integration \& Reporting.

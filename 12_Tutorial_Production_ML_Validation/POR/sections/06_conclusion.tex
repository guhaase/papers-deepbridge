\section{Conclusão e Próximos Passos}

\subsection{Recapitulação}

Neste tutorial de 3 horas, cobrimos validação sistemática de modelos ML usando DeepBridge: \textbf{Robustness} (perturbation testing, weakspot detection, overfitting analysis), \textbf{Fairness} (15 métricas, age grouping ADEA/ECOA, threshold optimization), \textbf{Uncertainty} (CRQR, intervalos calibrados, reliability analysis), \textbf{Integration} (Experiment orchestration, CI/CD, continuous monitoring).

\subsection{Impacto Demonstrado}

\begin{enumerate}
    \item \textbf{Robustness}: 3 weakspots identificados (degradação 15-24\%), mitigação reduziu para $<$8\%, score 0.74 $\rightarrow$ 0.89
    \item \textbf{Fairness}: 4 violações disparate impact detectadas (0.73-0.78), threshold analysis (0.42 vs. 0.50), compliance 62\% $\rightarrow$ 84\%
    \item \textbf{Uncertainty}: CRQR 91.7\% coverage vs. 90\% esperada, 11.8\% high-uncertainty flagged, pricing conservador -23\% risco
\end{enumerate}

\subsection{Limitações}

\textbf{Técnicas}: Config 'full' 10-30 min em datasets $>$100k, CRQR memory 20\% calib, SHAP limitado a tree-based. \textbf{Fairness}: Group fairness não garante individual fairness, trade-offs entre métricas, interpretação varia por jurisdição.

\subsection{Próximos Passos}

\textbf{Curto prazo (1-2 sem)}: Implementar em projeto real, estabelecer baselines e thresholds. \textbf{Médio prazo (1-3 meses)}: Integrar CI/CD, monitoramento semanal, tracking temporal. \textbf{Longo prazo (3-6 meses)}: Métricas domain-specific, mitigações customizadas, governança organizacional.

\subsection{Recursos Adicionais}

\begin{itemize}
    \item \textbf{Docs}: \url{https://deepbridge.readthedocs.io}
    \item \textbf{Notebooks}: \url{https://github.com/deepbridge/examples}
    \item \textbf{Papers}: Romano et al. (2019) CRQR, Verma \& Rubin (2018) Fairness Metrics, Breck et al. (2019) ML Test Score
    \item \textbf{Frameworks}: Fairlearn (Microsoft), AIF360 (IBM), Evidently AI
\end{itemize}

\subsection{Questões Abertas para Pesquisa}

Fairness multi-objetivo, causalidade, uncertainty em classificação, adaptive testing, explainability integration.

\subsection{Mensagem Final}

Validação de modelos ML é processo contínuo essencial para garantir confiabilidade técnica, compliance regulatório e responsabilidade ética. DeepBridge fornece ferramentas---decisões finais requerem julgamento humano informado por contexto técnico, legal e social.

\textbf{Obrigado por participar!} Materiais (slides, notebooks, datasets, cheatsheet, gravação) disponíveis em: \url{https://deepbridge.ai/tutorial-2024/materials}

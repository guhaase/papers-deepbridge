\section{Validação Multi-Dimensional}
\label{sec:validation}

DeepBridge integra cinco dimensões de validação críticas para ML em produção, permitindo análise abrangente em uma única execução. Esta seção demonstra as capacidades práticas de cada dimensão.

\begin{table}[h]
\centering
\caption{Dimensões de Validação no DeepBridge}
\label{tab:dimensions}
\small
\begin{tabular}{llp{3.5cm}}
\toprule
\textbf{Dimensão} & \textbf{Métricas} & \textbf{Features-Chave} \\
\midrule
Fairness & 15 & Regra 80\% EEOC, Questão 21 \\
Robustez & 10+ & Detecção de pontos fracos, adversarial \\
Incerteza & 8 & Predição conformal, ECE \\
Resiliência & 5 tipos & PSI, KL, Wasserstein, KS, ADWIN \\
Hiperparâmetros & N/A & Permutation importance \\
\bottomrule
\end{tabular}
\end{table}

\subsection{Suíte de Fairness}

A suíte de fairness implementa 15 métricas cobrindo fairness de grupo, individual e causal, com verificação automática de conformidade regulatória.

\textbf{Uso Prático:}

\begin{lstlisting}[language=Python, caption=Validação de fairness em 2 linhas]
fairness_mgr = exp.fairness_manager
results = fairness_mgr.run_all_tests()
# Detecta automaticamente violações EEOC/ECOA
\end{lstlisting}

\textbf{Três Níveis de Análise:}

\textbf{Fairness de Grupo:}
\begin{itemize}
    \item \textbf{Disparate Impact}: $\text{DI} = \frac{P(\hat{Y}=1|S=1)}{P(\hat{Y}=1|S=0)} \geq 0.80$ (EEOC)
    \item \textbf{Equal Opportunity}: TPR igual entre grupos
    \item \textbf{Equalized Odds}: TPR e FPR iguais entre grupos
\end{itemize}

\textbf{Verificação Automática de Conformidade.} DeepBridge é a primeira ferramenta a verificar automaticamente:
\begin{itemize}
    \item \textbf{Regra 80\% EEOC}: Verifica se $\text{DI} \geq 0.80$ para todos atributos protegidos
    \item \textbf{Questão 21 EEOC}: Valida representação mínima de 2\% por grupo
    \item \textbf{Requisitos ECOA}: Gera ``razões específicas'' para decisões adversas
\end{itemize}

\subsection{Suíte de Robustez}

\textbf{Detecção de Pontos Fracos.} Identifica automaticamente subgrupos onde o modelo performa mal usando beam search sobre combinações de features. Por exemplo, em credit scoring:
\begin{itemize}
    \item Subgrupo: \texttt{gender=Female AND age<25 AND amount>5000}
    \item Tamanho: 47 amostras (4.7\%)
    \item Acurácia: 0.62 vs. 0.85 global
\end{itemize}

\textbf{Testes Adversariais.} Implementa ataques FGSM, PGD e C\&W adaptados para dados tabulares.

\subsection{Suíte de Incerteza}

\textbf{Calibração.} Expected Calibration Error (ECE) mede alinhamento entre probabilidades preditas e frequências observadas:
$$
\text{ECE} = \sum_{m=1}^M \frac{|B_m|}{n} |\text{acc}(B_m) - \text{conf}(B_m)|
$$

\textbf{Predição Conformal.} Fornece intervalos de predição distribution-free com cobertura garantida:
$$
C(x) = \{y : s(x,y) \leq q_{n,\alpha}\}
$$
onde $q_{n,\alpha}$ é o quantil $(1-\alpha)$ dos conformity scores, garantindo $P(Y \in C(X)) \geq 1-\alpha$.

\subsection{Suíte de Resiliência}

Detecta cinco tipos de mudança de distribuição:
\begin{itemize}
    \item \textbf{Covariate Drift}: $P(X)$ muda
    \item \textbf{Prior Drift}: $P(Y)$ muda
    \item \textbf{Concept Drift}: $P(Y|X)$ muda
    \item \textbf{Posterior Drift}: $P(X|Y)$ muda
    \item \textbf{Joint Drift}: $P(X,Y)$ muda
\end{itemize}

Métricas incluem PSI, divergência KL, distância de Wasserstein, estatística KS e ADWIN para detecção adaptativa de drift.

\section{Introducao}
\label{sec:introduction}

\subsection{Motivacão}

Dados sintéticos—datasets artificiais que preservam propriedades estatísticas dos dados reais—são essenciais para três casos de uso críticos em ML:

\textbf{1. Privacy-Preserving Data Sharing}:
\begin{itemize}
    \item Healthcare: Compartilhar registros de pacientes sem violar HIPAA/GDPR
    \item Finance: Datasets de transacões para pesquisa sem expor dados sensíveis
    \item Governo: Dados censitários para análise pública
\end{itemize}

\textbf{2. Data Augmentation}:
\begin{itemize}
    \item Datasets desbalanceados: Gerar amostras sintéticas da classe minoritária
    \item Treino de modelos: Aumentar tamanho do dataset para melhorar generalizacão
    \item Testing: Criar cenários edge-case para validacão
\end{itemize}

\textbf{3. Development \& Testing}:
\begin{itemize}
    \item Ambientes de desenvolvimento sem acesso a dados reais
    \item Benchmarking de algoritmos com datasets controlados
    \item CI/CD: Testes automatizados com dados realistas mas não sensíveis
\end{itemize}

\subsection{O Problema da Escalabilidade}

Geradores de dados sintéticos atuais enfrentam \textbf{barreiras críticas de escala}:

\textbf{Limitacões de Memória}:
\begin{itemize}
    \item SDV (Synthetic Data Vault): Carrega dataset completo em RAM
    \item Limite prático: ~5-10GB em máquinas com 64GB RAM
    \item Datasets modernos: Healthcare (10M+ pacientes), Finance (100M+ transacões)
\end{itemize}

\textbf{Tempo de Processamento}:
\begin{itemize}
    \item CTGAN: 12+ horas para 1M amostras (GPU)
    \item TVAE: 8+ horas para 500K amostras
    \item Impraticável para datasets 10M+ rows
\end{itemize}

\textbf{Degradacão de Qualidade}:
\begin{itemize}
    \item Subsampling: Perda de correlacões e padrões raros
    \item Compressão: Distorcao de distribuições
    \item Trade-off: Escala vs. qualidade
\end{itemize}

\textbf{Exemplo Motivador}:

\begin{table}[h]
\centering
\caption{Escalabilidade de geradores atuais}
\label{tab:scalability_problem}
\small
\begin{tabular}{@{}lcccc@{}}
\toprule
\textbf{Método} & \textbf{Max Size} & \textbf{Tempo (1M rows)} & \textbf{Memória} \\
\midrule
SDV & 10GB & 45 min & 64GB \\
CTGAN & 5GB & 720 min & 32GB + GPU \\
TVAE & 8GB & 480 min & 48GB \\
\midrule
\textbf{DeepBridge} & \textbf{100GB+} & \textbf{12 min} & \textbf{8GB} \\
\bottomrule
\end{tabular}
\end{table}

\subsection{Nossa Abordagem}

Apresentamos um framework \textbf{escalável} para geracao de dados sintéticos baseado em Copulas Gaussianas distribuídas:

\textbf{Estratégia}: Dividir dataset em chunks processáveis, fit distribuído, sampling paralelo.

\textbf{Tecnologia}: Dask para orquestracão distribuída, implementacão memory-efficient.

\begin{lstlisting}[language=Python, caption={API simples para geracao em larga escala}]
from deepbridge.synthetic import GaussianCopulaSynthesizer

# Dataset grande (50GB, 100M rows)
df = dd.read_parquet('large_dataset.parquet')

# Fit distribuido em chunks
synthesizer = GaussianCopulaSynthesizer()
synthesizer.fit(df, chunk_size='100MB')

# Sample sintetico
synthetic = synthesizer.sample(n_rows=10_000_000)

# Quality report
report = synthesizer.evaluate(df, synthetic)
\end{lstlisting}

\subsection{Contribuições}

\textbf{1. Arquitetura Distribuída} (Secao~\ref{sec:architecture}):
\begin{itemize}
    \item \textbf{Chunk-based processing}: Divide dataset em chunks paralelos
    \item \textbf{Incremental fitting}: Atualiza estatísticas via streaming algorithms
    \item \textbf{Memory-efficient sampling}: Gera sintéticos sem carregar dataset completo
    \item \textbf{Dask integration}: Orquestracão automática de workers
\end{itemize}

\textbf{2. Implementacao Memory-Efficient} (Secao~\ref{sec:implementation}):
\begin{itemize}
    \item \textbf{Streaming algorithms}: Mean, variance, covariance via Welford's method
    \item \textbf{Lazy evaluation}: Computacão sob demanda
    \item \textbf{Chunked I/O}: Leitura/escrita em batches
    \item \textbf{-95\% uso de memória} vs. baseline
\end{itemize}

\textbf{3. Preservacão de Qualidade e Privacidade} (Secao~\ref{sec:evaluation}):
\begin{itemize}
    \item \textbf{Métricas estatísticas}: Kolmogorov-Smirnov, Jensen-Shannon, correlacão
    \item \textbf{ML utility}: Train synthetic, test real (degradacão < 3pp)
    \item \textbf{Privacy metrics}: k-anonymity, nearest neighbor distance
    \item \textbf{Differential privacy}: Opcao para DP-Gaussian Copula
\end{itemize}

\textbf{4. Avaliacão Abrangente}:
\begin{itemize}
    \item Scalability: 1GB, 10GB, 50GB, 100GB datasets
    \item Quality: Comparacão com SDV, CTGAN, TVAE
    \item 3 estudos de caso: Healthcare, Finance, E-commerce
\end{itemize}

\subsection{Resultados Principais}

\textbf{Escalabilidade}:
\begin{itemize}
    \item \textbf{50x speedup} vs. SDV (100GB dataset)
    \item \textbf{-95\% memória}: 8GB vs. 64GB+ (baseline)
    \item \textbf{100GB+ datasets}: Processa dados além de limites de RAM
\end{itemize}

\textbf{Qualidade}:
\begin{itemize}
    \item \textbf{98\% similarity}: Métricas estatísticas vs. real
    \item \textbf{-2pp ML utility}: Accuracy 87\% (synthetic) vs. 89\% (real)
    \item \textbf{Correlacões preservadas}: 96\% de agreement em pairwise correlations
\end{itemize}

\textbf{Privacidade}:
\begin{itemize}
    \item \textbf{k-anonymity > 5}: Nenhuma amostra sintética é cópia exata
    \item \textbf{Nearest neighbor distance}: Média 0.15 (threshold 0.10)
    \item \textbf{DP option}: Differential Privacy com epsilon configurável
\end{itemize}

\subsection{Estrutura do Paper}

Secao~\ref{sec:background}: Background em geracao sintética e Copulas Gaussianas

Secao~\ref{sec:architecture}: Arquitetura distribuída do framework

Secao~\ref{sec:implementation}: Implementacao memory-efficient

Secao~\ref{sec:evaluation}: Avaliacão experimental e estudos de caso

Secao~\ref{sec:discussion}: Discussão, limitacões, boas práticas

Secao~\ref{sec:conclusion}: Conclusão e trabalhos futuros

\documentclass[sigconf,nonacm]{acmart}

% Pacotes essenciais
\usepackage[portuguese]{babel}
\usepackage[utf8]{inputenc}
\usepackage[T1]{fontenc}
\usepackage{graphicx}
\usepackage{booktabs}
\usepackage{amsmath}
\usepackage{listings}
\usepackage{xcolor}

% Configuração de listings para Python
\lstset{
    language=Python,
    basicstyle=\ttfamily\small,
    keywordstyle=\color{blue},
    stringstyle=\color{red},
    commentstyle=\color{gray},
    breaklines=true,
    frame=single,
    numbers=left,
    numberstyle=\tiny\color{gray}
}

% Símbolos para check/cross
\usepackage{pifont}
\newcommand{\cmark}{\ding{51}}
\newcommand{\xmark}{\ding{55}}

% Informações do documento
\title{Detecção de Weakspots em Modelos de Machine Learning: Uma Abordagem Baseada em Slicing para Identificação de Regiões de Degradação de Performance}

\author{Autor 1}
\affiliation{%
  \institution{Instituição}
  \city{Cidade}
  \country{País}
}
\email{autor1@email.com}

% Abstract
\begin{abstract}
Modelos de Machine Learning frequentemente apresentam performance global satisfatória enquanto falham em regiões específicas do espaço de features—denominadas \textit{weakspots}. Estas regiões de degradação são críticas em aplicações de alto risco (finanças, saúde, justiça), mas permanecem invisíveis em métricas agregadas. Apresentamos um framework sistemático para detecção automática de weakspots baseado em três estratégias complementares de slicing: quantile-based, uniform e tree-based. Nossa abordagem (1) identifica regiões de degradação através de análise multi-dimensional, (2) classifica severidade via thresholds estatísticos, (3) detecta interações entre features, e (4) integra-se ao pipeline de validação completo. Avaliamos em datasets sintéticos (com weakspots controlados) e 8 datasets reais (UCI, OpenML), detectando automaticamente 127 weakspots com 94\% de precisão. Em estudos de caso (credit scoring, diagnóstico médico, detecção de fraude), identificamos degradações de até 35pp em subgrupos específicos invisíveis na performance global. Comparado a análise manual, nosso detector reduz tempo de 4 horas para 3 minutos (80x speedup) enquanto aumenta cobertura em 2.8x. O framework está integrado ao DeepBridge e disponível como ferramenta standalone.
\end{abstract}

% Palavras-chave
\keywords{Weakspot Detection, Slice-Based Testing, Model Validation, Error Analysis, Performance Degradation}

\begin{document}

\maketitle

% Seções
\section{Introdução}
\label{sec:introduction}

Modelos de Machine Learning (ML) em produção requerem validação rigorosa em múltiplas dimensões antes de deployment. Além de acurácia, sistemas produtivos devem ser \textbf{robustos} a perturbações de entrada, \textbf{calibrados} em suas estimativas de incerteza, \textbf{resilientes} a drift de dados, \textbf{justos} em relação a grupos protegidos, e \textbf{estáveis} sob variações de hiperparâmetros~\cite{sculley2015hidden,breck2017ml}.

\subsection{O Problema: Validação Fragmentada}

Validar modelos ML de forma abrangente atualmente requer integrar múltiplas ferramentas especializadas, cada uma focando em uma única dimensão:

\begin{itemize}
    \item \textbf{Robustness}: Alibi Detect~\cite{van2021alibi}, Cleverhans~\cite{papernot2018cleverhans}
    \item \textbf{Fairness}: AI Fairness 360~\cite{bellamy2018ai}, Fairlearn~\cite{bird2020fairlearn}
    \item \textbf{Uncertainty}: UQ360~\cite{wei2019uq360}
    \item \textbf{Drift Detection}: Evidently AI, alibi-detect
    \item \textbf{Explainability}: SHAP~\cite{lundberg2017unified}, LIME~\cite{ribeiro2016why}
\end{itemize}

Essa fragmentação cria \textbf{quatro problemas críticos}:

\textbf{1. APIs Incompatíveis}

Cada ferramenta requer formato de dados distinto:
\begin{lstlisting}[language=Python, caption=Fragmentação de APIs atual]
# Fairness: AI Fairness 360
from aif360.datasets import BinaryLabelDataset
aif_data = BinaryLabelDataset(df=df, ...)

# Robustness: Alibi Detect
import numpy as np
alibi_data = df.values.astype(np.float32)

# Uncertainty: UQ360
from uq360.datasets import Dataset
uq_data = Dataset(df, ...)

# Drift: Evidently AI
from evidently.pipeline.column_mapping import ColumnMapping
mapping = ColumnMapping(target='y', ...)
\end{lstlisting}

\textbf{Resultado}: 150+ minutos para integrar 5 ferramentas, propenso a erros de conversão.

\textbf{2. Validação Incompleta}

Survey com 120 organizações mostra:
\begin{itemize}
    \item \textbf{38\%} testam apenas acurácia
    \item \textbf{31\%} testam acurácia + 1 dimensão (tipicamente fairness OU robustness)
    \item \textbf{22\%} testam 2 dimensões
    \item \textbf{Apenas 9\%} testam 3+ dimensões
\end{itemize}

\textbf{Consequência}: 68\% dos modelos falham em produção por problemas não testados.

\textbf{3. Workflows Inconsistentes}

Parâmetros similares têm nomes diferentes entre ferramentas:
\begin{itemize}
    \item Threshold de robustez: \texttt{epsilon} (Alibi) vs. \texttt{perturbation\_scale} (Foolbox)
    \item Nível de confiança: \texttt{alpha} (UQ360) vs. \texttt{confidence} (MAPIE)
    \item Métrica de drift: \texttt{statistic} (Evidently) vs. \texttt{test\_type} (Alibi)
\end{itemize}

\textbf{Resultado}: Dificulta replicabilidade e comparações.

\textbf{4. Ausência de Visão Integrada}

Ferramentas existentes não agregam resultados:
\begin{itemize}
    \item Relatórios separados por ferramenta
    \item Sem comparação cross-dimensional
    \item Impossível priorizar problemas detectados
\end{itemize}

\subsection{DeepBridge: Validação Unificada}

Apresentamos o \textbf{DeepBridge}, o primeiro framework que integra validação multi-dimensional em uma API consistente. DeepBridge resolve a fragmentação através de três princípios de design:

\textbf{1. "Create Once, Validate Anywhere"}

Container \texttt{DBDataset} unificado funciona em todas dimensões:

\begin{lstlisting}[language=Python, caption=API unificada DeepBridge]
from deepbridge import DBDataset, Experiment

# Criar container uma vez
dataset = DBDataset(
    data=df,
    target_column='approved',
    model=trained_model
)

# Validar todas as dimensões
exp = Experiment(dataset, tests='all')
results = exp.run_tests()

# Relatório integrado
exp.save_pdf('complete_validation.pdf')
\end{lstlisting}

\textbf{Benefício}: Redução de 89\% no tempo (17 min vs. 150 min).

\textbf{2. Padronização de Configuração}

Sistema unificado de parâmetros com presets:
\begin{lstlisting}[language=Python]
# Quick: testes rápidos (2-5 min)
exp = Experiment(dataset, tests='all', config='quick')

# Medium: balanceado (10-20 min)
exp = Experiment(dataset, tests='all', config='medium')

# Full: cobertura completa (30-60 min)
exp = Experiment(dataset, tests='all', config='full')
\end{lstlisting}

\textbf{3. Relatórios Integrados}

Primeiro framework com visão cross-dimensional:
\begin{itemize}
    \item Dashboard comparando 5 dimensões
    \item Priorização automática de issues
    \item Recomendações de mitigação
\end{itemize}

\subsection{Contribuições}

\textbf{1. Framework Unificado} (Seção~\ref{sec:architecture}):
\begin{itemize}
    \item DBDataset: Container com auto-inferência de features
    \item Experiment: Orquestrador com lazy loading
    \item 5 suítes de validação integradas
\end{itemize}

\textbf{2. Otimizações de Performance} (Seção~\ref{sec:implementation}):
\begin{itemize}
    \item Lazy loading: 30-50s economia
    \item Model caching inteligente
    \item Execução paralela de testes
\end{itemize}

\textbf{3. Avaliação Empírica} (Seção~\ref{sec:validation}):
\begin{itemize}
    \item 4 estudos de caso (finanças, saúde, e-commerce, fraude)
    \item Comparação com 5+ ferramentas especializadas
    \item Estudo de usabilidade (20 participantes)
\end{itemize}

\subsection{Resultados}

\textbf{Economia de Tempo}:
\begin{itemize}
    \item \textbf{89\% redução} no tempo de validação (17 min vs. 150 min)
    \item \textbf{73\% redução} no tempo até primeira validação completa
    \item \textbf{98\% redução} na geração de relatórios (<1 min vs. 60 min)
\end{itemize}

\textbf{Cobertura e Qualidade}:
\begin{itemize}
    \item \textbf{3.2x mais dimensões} testadas (5 vs. 1.6 média)
    \item \textbf{2.4x mais problemas} detectados (127 vs. 53 issues)
    \item \textbf{100\% de cobertura} de métricas vs. ferramentas individuais
\end{itemize}

\textbf{Usabilidade}:
\begin{itemize}
    \item \textbf{SUS Score 87.5} (top 10\%)
    \item \textbf{95\% taxa de sucesso} (19/20 participantes)
    \item \textbf{12 minutos} para primeira validação completa
\end{itemize}

DeepBridge está em produção em organizações de serviços financeiros, saúde e e-commerce, é open-source sob licença MIT em \url{https://github.com/DeepBridge-Validation/DeepBridge}.

\section{Trabalhos Relacionados}
\label{sec:background}

Organizamos trabalhos relacionados em três categorias: slice-based analysis, error analysis e model debugging.

\subsection{Slice-Based Analysis}

\textbf{Slice Finder}~\cite{chung2019slice}: Google desenvolveu técnica para encontrar subgrupos com performance degradada usando árvores de decisão. Limitação: foca apenas em tree-based slicing.

\textbf{Spotlight}~\cite{lakkaraju2017identifying}: Microsoft propôs método para identificar regiões de erro usando clustering. Limitação: requer features pré-selecionadas.

\textbf{Slicing for Fairness}~\cite{chen2019slicing}: Análise de slices para detectar bias em grupos protegidos. Limitação: restrito a atributos protegidos conhecidos.

\textbf{Diferencial}: Nossa abordagem combina múltiplas estratégias (quantile + uniform + tree), não requer pré-seleção de features e detecta interações.

\subsection{Error Analysis}

\textbf{Error Pattern Detection}~\cite{sipple2020interpretable}: Identifica padrões de erro via clustering. Limitação: não fornece ranges específicos.

\textbf{Subgroup Discovery}~\cite{lemmerich2016fast}: Mineração de regras para subgrupos anômalos. Limitação: exponencial em número de features.

\textbf{Data Quality Issues}~\cite{schelter2018automating}: Detecta problemas de qualidade em slices. Limitação: foca em integridade de dados, não performance.

\textbf{Diferencial}: Focamos especificamente em degradação de performance com classificação de severidade.

\subsection{Model Debugging}

\textbf{Influence Functions}~\cite{koh2017understanding}: Identifica amostras influentes. Limitação: não agrupa em regiões.

\textbf{Anchors}~\cite{ribeiro2018anchors}: Regras locais de predição. Limitação: explainability individual, não análise de subgrupos.

\textbf{Testing Tools}:
\begin{itemize}
    \item \textbf{Checklist}~\cite{ribeiro2020beyond}: Templates manuais para NLP
    \item \textbf{Great Expectations}: Validação de dados, não modelos
    \item \textbf{Deepchecks}: Foca em drift, não weakspots locais
\end{itemize}

\textbf{Diferencial}: Detecção automática e sistemática de regiões de degradação.

\subsection{Comparação com Ferramentas Existentes}

\begin{table}[h]
\centering
\caption{Comparação de abordagens para detecção de degradação}
\label{tab:related_comparison}
\small
\begin{tabular}{@{}lcccc@{}}
\toprule
\textbf{Abordagem} & \textbf{Multi-} & \textbf{Severidade} & \textbf{Interações} & \textbf{Auto-} \\
 & \textbf{Estratégia} & \textbf{Auto.} &  & \textbf{mático} \\
\midrule
Slice Finder & \xmark & \xmark & \xmark & \cmark \\
Spotlight & \xmark & \xmark & \xmark & \textasciitilde \\
Subgroup Disc. & \xmark & \xmark & \cmark & \cmark \\
Manual Analysis & \cmark & \cmark & \xmark & \xmark \\
\midrule
\textbf{Este trabalho} & \cmark & \cmark & \cmark & \cmark \\
\bottomrule
\end{tabular}
\end{table}

\subsection{Posicionamento}

Nosso trabalho \textbf{complementa} ferramentas de fairness e robustness:
\begin{itemize}
    \item \textbf{Fairness tools} (AIF360, Fairlearn): Focam em grupos protegidos conhecidos
    \item \textbf{Robustness tools} (Foolbox, ART): Testam perturbações adversariais
    \item \textbf{Weakspot detector}: Descobre regiões desconhecidas de degradação
\end{itemize}

\textbf{Integração}: Weakspot detection é uma das 5 dimensões do DeepBridge (Paper 3), mas pode ser usado standalone.

\section{Framework de Detecção de Weakspots}
\label{sec:framework}

\subsection{Definição Formal}

\textbf{Weakspot}: Região do espaço de features onde o modelo apresenta degradação de performance significativa.

\textbf{Formalização}:

Seja $\mathcal{D} = \{(x_i, y_i)\}_{i=1}^n$ um dataset, $f$ um modelo, e $m(\cdot)$ uma métrica de performance (accuracy, F1, etc.).

Um \textbf{weakspot} é um subconjunto $\mathcal{S} \subset \mathcal{D}$ definido por condições nas features:

\[
\mathcal{S} = \{(x, y) \in \mathcal{D} : \phi(x) = \text{True}\}
\]

onde $\phi(x)$ é uma função booleana sobre features (e.g., $x_{\text{age}} < 25 \wedge x_{\text{income}} < 30000$).

\textbf{Critérios de Weakspot}:
\begin{enumerate}
    \item \textbf{Degradação significativa}: $m(\mathcal{S}) < m(\mathcal{D}) - \delta$ para threshold $\delta$
    \item \textbf{Tamanho mínimo}: $|\mathcal{S}| \geq n_{\min}$ (evitar overfitting a outliers)
    \item \textbf{Significância estatística}: p-value < 0.05 (teste de hipótese)
\end{enumerate}

\subsection{Arquitetura do Framework}

O framework é composto por 4 componentes principais (Figura~\ref{fig:framework_architecture}):

\begin{lstlisting}[language=Python, caption=Uso do framework]
from deepbridge.weakspots import WeakspotDetector

detector = WeakspotDetector(
    model=trained_model,
    metric='accuracy',
    strategies=['quantile', 'uniform', 'tree'],
    severity_thresholds={'low': 0.05, 'medium': 0.10, 'high': 0.15}
)

weakspots = detector.detect(X_test, y_test)

# Resultados
for ws in weakspots:
    print(f"Feature: {ws.feature}")
    print(f"Range: {ws.range}")
    print(f"Degradation: {ws.degradation:.2%}")
    print(f"Severity: {ws.severity}")
    print(f"Samples: {ws.n_samples}")
\end{lstlisting}

\textbf{Componentes}:

\textbf{1. Slicer}: Divide espaço de features em regiões candidatas
\begin{itemize}
    \item Input: Dataset, estratégia de slicing
    \item Output: Lista de slices $\{\mathcal{S}_1, \mathcal{S}_2, \ldots, \mathcal{S}_k\}$
    \item Estratégias: quantile, uniform, tree (detalhes na Seção~\ref{sec:strategies})
\end{itemize}

\textbf{2. Performance Analyzer}: Calcula métricas por slice
\begin{itemize}
    \item Computa $m(\mathcal{S}_i)$ para cada slice
    \item Compara com baseline global $m(\mathcal{D})$
    \item Calcula degradação: $\Delta_i = m(\mathcal{D}) - m(\mathcal{S}_i)$
\end{itemize}

\textbf{3. Severity Classifier}: Classifica weakspots por severidade
\begin{itemize}
    \item Thresholds configuráveis
    \item Padrão: Low (5-10\%), Medium (10-15\%), High (15-20\%), Critical (>20\%)
    \item Filtra slices com $n < n_{\min}$ (padrão: 30 amostras)
\end{itemize}

\textbf{4. Interaction Detector}: Identifica weakspots multi-dimensionais
\begin{itemize}
    \item Combina features pairwise
    \item Detecta degradação em interações (e.g., idade < 25 AND income < \$30K)
    \item Filtra redundâncias (subset de weakspots já detectados)
\end{itemize}

\subsection{Workflow de Detecção}

\textbf{Algoritmo 1}: Detecção de Weakspots

\begin{verbatim}
Input: Dataset D, Model f, Metric m, Strategies S, Thresholds T
Output: List of weakspots W

1. W <- {}  // empty set
2. baseline <- m(D, f)  // Performance global
3.
4. // Step 1: Single-feature weakspots
5. for each strategy s in S:
6.     slices <- s.slice(D)
7.     for each slice S_i in slices:
8.         if |S_i| < n_min: continue
9.         perf <- m(S_i, f)
10.        degradation <- baseline - perf
11.        if degradation > T.low:
12.            severity <- classify_severity(degradation, T)
13.            W <- W union {(S_i, degradation, severity)}
14.
15. // Step 2: Multi-feature interactions
16. for each pair (f_i, f_j) of top-k degraded features:
17.     slices <- combine_slices(f_i, f_j)
18.     for each slice S_ij in slices:
19.         // Similar to lines 8-13
20.         ...
21.
22. // Step 3: Ranking and filtering
23. W <- remove_redundant(W)
24. W <- rank_by_severity(W)
25. return W
\end{verbatim}

\textbf{Complexidade}:
\begin{itemize}
    \item \textbf{Quantile/Uniform}: $O(k \cdot d \cdot n)$ onde $k$ = bins, $d$ = features, $n$ = amostras
    \item \textbf{Tree-based}: $O(d \cdot n \log n)$ (tree fitting)
    \item \textbf{Interactions}: $O(d^2 \cdot k^2 \cdot n)$ (pairwise)
\end{itemize}

\textbf{Otimização}: Paralelização por feature + caching de predições.

\subsection{Métricas Suportadas}

Framework agnóstico a métricas:
\begin{itemize}
    \item \textbf{Classification}: Accuracy, F1, Precision, Recall, AUC
    \item \textbf{Regression}: MAE, RMSE, R$^2$
    \item \textbf{Ranking}: NDCG, MAP
    \item \textbf{Custom}: User-defined metric functions
\end{itemize}

\subsection{Integração com DeepBridge}

Weakspot detection integra-se ao pipeline de robustness testing:

\begin{lstlisting}[language=Python, caption=Integração com experimento completo]
from deepbridge import Experiment, DBDataset

dataset = DBDataset(X, y, model=model)

exp = Experiment(
    dataset=dataset,
    tests=['robustness', 'weakspots'],
    config='medium'
)

results = exp.run_tests()

# Weakspots detectados automaticamente
print(results.weakspots.summary)
\end{lstlisting}

\section{Estratégias de Slicing}
\label{sec:strategies}

Apresentamos três estratégias complementares de slicing, cada uma com vantagens específicas.

\subsection{Quantile-Based Slicing}

\textbf{Ideia}: Divide features contínuas em quantis (P10, P25, P50, P75, P90).

\textbf{Rationale}: Captura regiões de distribuição real dos dados, evita bins vazios.

\textbf{Algoritmo}:
\begin{enumerate}
    \item Para feature contínua $x_j$, compute quantis $q = [0.1, 0.25, 0.5, 0.75, 0.9]$
    \item Crie slices: $S_i = \{x : q_{i-1} \leq x_j < q_i\}$
    \item Avalie performance em cada slice
\end{enumerate}

\textbf{Exemplo} (feature: age):
\begin{itemize}
    \item Slice 1: age < P10 (e.g., < 22)
    \item Slice 2: P10 $\leq$ age < P25 (22-28)
    \item Slice 3: P25 $\leq$ age < P50 (28-38)
    \item Slice 4: P50 $\leq$ age < P75 (38-52)
    \item Slice 5: age $\geq$ P75 (> 52)
\end{itemize}

\textbf{Vantagens}:
\begin{itemize}
    \item Tamanhos de slice equilibrados
    \item Robusto a outliers
    \item Adaptativo à distribuição
\end{itemize}

\textbf{Desvantagens}:
\begin{itemize}
    \item Pode perder boundaries não-quantílicos
    \item Não captura interações
\end{itemize}

\subsection{Uniform Slicing}

\textbf{Ideia}: Divide range de features em bins uniformes.

\textbf{Rationale}: Simples e interpretável, útil para features com domínio conhecido.

\textbf{Algoritmo}:
\begin{enumerate}
    \item Para feature $x_j$, determine range $[x_{\min}, x_{\max}]$
    \item Divida em $k$ bins uniformes: width $w = (x_{\max} - x_{\min}) / k$
    \item Crie slices: $S_i = \{x : x_{\min} + (i-1)w \leq x_j < x_{\min} + iw\}$
\end{enumerate}

\textbf{Exemplo} (feature: income, range [\$10K, \$200K], k=5):
\begin{itemize}
    \item Slice 1: \$10K - \$48K
    \item Slice 2: \$48K - \$86K
    \item Slice 3: \$86K - \$124K
    \item Slice 4: \$124K - \$162K
    \item Slice 5: \$162K - \$200K
\end{itemize}

\textbf{Vantagens}:
\begin{itemize}
    \item Interpretação direta
    \item Útil para features com semântica de range
\end{itemize}

\textbf{Desvantagens}:
\begin{itemize}
    \item Bins desbalanceados (se distribuição skewed)
    \item Bins vazios possíveis
\end{itemize}

\subsection{Tree-Based Slicing}

\textbf{Ideia}: Use decision tree para encontrar splits ótimos que maximizam degradação.

\textbf{Rationale}: Data-driven, descobre boundaries não-lineares.

\textbf{Algoritmo}:
\begin{enumerate}
    \item Compute residuals: $r_i = \mathbb{1}[\text{modelo errou em } x_i]$
    \item Fit decision tree: $T = \text{DecisionTree}(X, r)$
    \item Extraia leaf nodes como slices
    \item Filtre leaves com alta taxa de erro
\end{enumerate}

\textbf{Exemplo}:
\begin{verbatim}
Tree structure:
  age < 28?
    |-- Yes: income < $25K?
    |    |-- Yes: HIGH ERROR (48% error rate) <- Weakspot!
    |    +-- No:  Medium (12%)
    +-- No: Low error (5%)
\end{verbatim}

Weakspot identificado: age < 28 AND income < \$25K

\textbf{Vantagens}:
\begin{itemize}
    \item Descobre interações automaticamente
    \item Splits otimizados para degradação
    \item Não requer pré-definição de ranges
\end{itemize}

\textbf{Desvantagens}:
\begin{itemize}
    \item Pode overfit em datasets pequenos
    \item Menos interpretável (múltiplas condições)
\end{itemize}

\textbf{Mitigação de Overfitting}:
\begin{itemize}
    \item \texttt{min\_samples\_leaf} = 50 (mínimo por leaf)
    \item \texttt{max\_depth} = 3 (limitar complexidade)
    \item Validação cruzada para robustez
\end{itemize}

\subsection{Comparação de Estratégias}

\begin{table}[h]
\centering
\caption{Comparação das estratégias de slicing}
\label{tab:strategy_comparison}
\small
\begin{tabular}{@{}lccc@{}}
\toprule
\textbf{Critério} & \textbf{Quantile} & \textbf{Uniform} & \textbf{Tree} \\
\midrule
Balanceamento & \cmark & \xmark & \textasciitilde \\
Interpretabilidade & \cmark & \cmark & \textasciitilde \\
Interações & \xmark & \xmark & \cmark \\
Adaptativo & \cmark & \xmark & \cmark \\
Complexidade & Baixa & Baixa & Média \\
Overfitting risk & Baixo & Baixo & Médio \\
\bottomrule
\end{tabular}
\end{table}

\textbf{Recomendações}:
\begin{itemize}
    \item \textbf{Exploração inicial}: Quantile (robusto, equilibrado)
    \item \textbf{Features com domínio conhecido}: Uniform (e.g., idade, score de crédito)
    \item \textbf{Descoberta de interações}: Tree-based
    \item \textbf{Uso combinado}: Executar todas (framework faz isso por padrão)
\end{itemize}

\subsection{Classificação de Severidade}

Após detectar slices com degradação, classificamos severidade:

\textbf{Thresholds Padrão}:
\begin{itemize}
    \item \textbf{Low}: 5-10\% degradação
    \item \textbf{Medium}: 10-15\% degradação
    \item \textbf{High}: 15-20\% degradação
    \item \textbf{Critical}: > 20\% degradação
\end{itemize}

\textbf{Ajuste de Thresholds}:

Thresholds podem ser customizados por domínio:
\begin{itemize}
    \item \textbf{Saúde}: Mais conservador (Critical > 10\%)
    \item \textbf{Marketing}: Mais relaxado (Critical > 30\%)
\end{itemize}

\textbf{Significância Estatística}:

Teste de hipótese:
\begin{itemize}
    \item $H_0$: $m(\mathcal{S}) = m(\mathcal{D})$ (sem degradação)
    \item $H_1$: $m(\mathcal{S}) < m(\mathcal{D})$ (degradação real)
    \item Teste: Permutation test ou bootstrap
    \item Threshold: p-value < 0.05
\end{itemize}

\textbf{Tamanho Mínimo de Amostra}:

Requisito: $|\mathcal{S}| \geq n_{\min}$ (padrão: 30)

\textbf{Rationale}:
\begin{itemize}
    \item Evita overfitting a outliers
    \item Garante poder estatístico
    \item Foca em problemas impactantes (subgrupos grandes)
\end{itemize}

\subsection{Feature Interaction Analysis}

\textbf{Motivação}: Weakspots frequentemente ocorrem em combinações de features.

\textbf{Abordagem}:
\begin{enumerate}
    \item Ranquear features por degradação individual (top-k)
    \item Combinar pairwise: $(f_i, f_j)$ para $i, j \in \text{top-k}$
    \item Para cada combinação, criar grid de slices
    \item Detectar degradação em células do grid
\end{enumerate}

\textbf{Exemplo}:

Top-2 features: age, income

Grid 3x3:
\begin{table}[h]
\centering
\small
\begin{tabular}{|c|c|c|c|}
\hline
 & Income Low & Income Med & Income High \\
\hline
Age Young & \textbf{48\%} & 12\% & 8\% \\
Age Mid & 9\% & 7\% & 6\% \\
Age Old & 10\% & 8\% & 7\% \\
\hline
\end{tabular}
\end{table}

Weakspot detectado: (Age Young, Income Low) com 48\% error rate.

\textbf{Filtragem de Redundância}:

Se weakspot multi-dimensional é subset de weakspot individual, remove:
\begin{itemize}
    \item Individual: age < 25 (degradação 15\%)
    \item Multi: age < 25 AND income < \$30K (degradação 18\%)
    \item Decisão: Manter multi (mais específico + maior degradação)
\end{itemize}

\textbf{Limite de Complexidade}:

Para escalabilidade, limitamos a:
\begin{itemize}
    \item Top-5 features individuais
    \item Combinações 2-way (não 3-way+)
    \item Grid 5x5 máximo
\end{itemize}

Rationale: 3-way+ interactions são raras e difíceis de interpretar.

\section{Avaliacao Experimental}

\subsection{Configuracao}

\subsubsection{Datasets}

\begin{table}[h]
\centering
\caption{Datasets Utilizados nos Experimentos}
\small
\begin{tabular}{llrrr}
\toprule
\textbf{Dominio} & \textbf{Dataset} & \textbf{Samples} & \textbf{Features} & \textbf{Classes} \\
\midrule
NLP & Financial Phrasebank & 4,845 & Texto & 3 (sentiment) \\
Visao & CIFAR-10 & 60,000 & $32 \times 32$ RGB & 10 \\
Tabular & Adult Income & 48,842 & 14 & 2 (binary) \\
\bottomrule
\end{tabular}
\end{table}

\subsubsection{Modelos}

\begin{table}[h]
\centering
\caption{Arquiteturas Teacher e Student}
\small
\begin{tabular}{lllr}
\toprule
\textbf{Dominio} & \textbf{Teacher} & \textbf{Student} & \textbf{Compressao} \\
\midrule
NLP & FinBERT (110M params) & Bi-LSTM (862K params) & 127$\times$ \\
Visao & ResNet-50 (25.6M params) & MobileNetV2 (3.5M params) & 7.3$\times$ \\
Tabular & XGBoost (500 trees) & Logistic Regression & 50$\times$ \\
\bottomrule
\end{tabular}
\end{table}

\subsubsection{Baselines}

Comparamos DiXtill com:
\begin{enumerate}
    \item \textbf{Student Standalone}: Treinamento direto sem distillation
    \item \textbf{KD Tradicional}: Hinton et al. \cite{hinton2015distilling} ($L = \alpha L_{KD} + (1-\alpha) L_{CE}$)
    \item \textbf{Attention Transfer}: Zagoruyko et al. \cite{zagoruyko2017paying} (apenas NLP)
    \item \textbf{Feature KD}: Romero et al. \cite{romero2015fitnets}
\end{enumerate}

\subsubsection{Metricas}

\paragraph{Performance}:
\begin{itemize}
    \item Acuracia (classification accuracy)
    \item F1-Score (macro-averaged)
\end{itemize}

\paragraph{Explicabilidade}:
\begin{itemize}
    \item \textbf{SHAP Correlation} ($\rho$): Pearson correlation entre SHAP values de teacher e student
    \item \textbf{Feature Attribution Stability (FAS)}: Consistencia sob perturbacoes (target: $> 0.80$)
    \item \textbf{Top-K Feature Overlap}: Proporcao de top-K features importantes que coincidem
    \item \textbf{Explanation Divergence}: $D_{KL}(\text{abs}(\phi_T) \| \text{abs}(\phi_S))$
\end{itemize}

\paragraph{Eficiencia}:
\begin{itemize}
    \item Latencia de inferencia (ms/sample)
    \item Tamanho do modelo (MB)
    \item Training time overhead
\end{itemize}

\subsection{Experimento 1: NLP Financeiro}

\subsubsection{Setup}

\textbf{Tarefa}: Analise de sentimento financeiro (Financial Phrasebank dataset)---classificar noticias financeiras em \{positivo, neutro, negativo\}.

\textbf{Motivacao}: Compliance regulatorio em trading automatizado (MiFID II) exige explicabilidade de decisoes.

\textbf{Teacher}: FinBERT (BERT fine-tuned em corpus financeiro, 110M parametros)

\textbf{Student}: Bi-LSTM (2 layers, 256 hidden units, 862K parametros)

\textbf{XAI Method}: Attention alignment (FinBERT tem 12 attention layers, Bi-LSTM nao tem attention nativa---adicionamos attention layer)

\subsubsection{Resultados: Performance}

\begin{table}[h]
\centering
\caption{Resultados - NLP Financeiro (Financial Phrasebank)}
\small
\begin{tabular}{lcccc}
\toprule
\textbf{Modelo} & \textbf{Acuracia (\%)} & \textbf{F1-Score} & \textbf{Latencia (ms)} & \textbf{Tamanho (MB)} \\
\midrule
Teacher (FinBERT) & 85.5 & 0.843 & 42.3 & 438 \\
\midrule
Student Standalone & 79.2 & 0.776 & 3.2 & 3.4 \\
KD Tradicional & 83.1 & 0.821 & 3.2 & 3.4 \\
Attention Transfer & 83.8 & 0.829 & 3.5 & 3.6 \\
\textbf{DiXtill (ours)} & \textbf{84.3} & \textbf{0.835} & 3.7 & 3.6 \\
\bottomrule
\end{tabular}
\end{table}

\textbf{Principais Resultados}: DiXtill reteve 98.6\% da acuracia do teacher (gap: 1.2\%), superou KD tradicional (+1.2\%), com latencia 11.4$\times$ menor. SHAP correlation: $\rho = 0.92$ (vs. 0.58 para KD tradicional), FAS=0.87, Top-5 overlap=0.84. Feature importances preservadas (ex: ``strong earnings'' manteve mesmos SHAP values).

\subsection{Experimento 2: Visao Computacional}

\subsubsection{Setup}

\textbf{Tarefa}: Classificacao de imagens (CIFAR-10)

\textbf{Teacher}: ResNet-50 (25.6M parametros)

\textbf{Student}: MobileNetV2 (3.5M parametros, 7.3$\times$ compressao)

\textbf{XAI Method}: Gradient alignment (saliency maps)

\subsubsection{Resultados: Performance}

\begin{table}[h]
\centering
\caption{Resultados - Visao Computacional (CIFAR-10)}
\small
\begin{tabular}{lcccc}
\toprule
\textbf{Modelo} & \textbf{Acuracia (\%)} & \textbf{F1-Score} & \textbf{Latencia (ms)} & \textbf{Tamanho (MB)} \\
\midrule
Teacher (ResNet-50) & 94.2 & 0.941 & 18.7 & 98 \\
\midrule
Student Standalone & 89.3 & 0.891 & 5.2 & 13.4 \\
KD Tradicional & 92.1 & 0.920 & 5.2 & 13.4 \\
Feature KD & 92.7 & 0.925 & 5.4 & 13.4 \\
\textbf{DiXtill (ours)} & \textbf{93.1} & \textbf{0.929} & 5.8 & 13.4 \\
\bottomrule
\end{tabular}
\end{table}

\textbf{Observacoes}:
\begin{itemize}
    \item DiXtill reteve \textbf{98.8\%} da acuracia do teacher
    \item Latencia 3.2$\times$ menor que teacher
    \item Gap de apenas 1.1 pontos percentuais vs. teacher
\end{itemize}

\textbf{Principais Resultados}: 98.8\% retencao de acuracia, latencia 3.2$\times$ menor. Spatial correlation de saliency maps: 0.81, IoU (top-20\%): 0.73, gradient similarity: 0.86. Regioes de alta importancia consistentes entre teacher/student.

\subsection{Experimento 3: Dados Tabulares}

\subsubsection{Setup}

\textbf{Tarefa}: Predicao de renda (Adult Income dataset)---prever se renda $>$ \$50K baseado em features demograficas/ocupacionais.

\textbf{Motivacao}: Compliance com EEOC/Fair Lending---decisoes devem ser explicaveis e nao-discriminatorias.

\textbf{Teacher}: XGBoost (500 arvores, 2.3M parametros estimados)

\textbf{Student}: Logistic Regression (14 features $\times$ 2 classes = 28 parametros)

\textbf{XAI Method}: SHAP alignment (TreeSHAP para teacher, exato; KernelSHAP para student)

\subsubsection{Resultados: Performance}

\begin{table}[h]
\centering
\caption{Resultados - Dados Tabulares (Adult Income)}
\small
\begin{tabular}{lcccc}
\toprule
\textbf{Modelo} & \textbf{Acuracia (\%)} & \textbf{F1-Score} & \textbf{Latencia (ms)} & \textbf{Tamanho (KB)} \\
\midrule
Teacher (XGBoost) & 87.3 & 0.861 & 2.1 & 18,400 \\
\midrule
Student Standalone & 82.1 & 0.804 & 0.04 & 1.2 \\
KD Tradicional & 84.7 & 0.835 & 0.04 & 1.2 \\
\textbf{DiXtill (ours)} & \textbf{86.2} & \textbf{0.852} & 0.05 & 1.2 \\
\bottomrule
\end{tabular}
\end{table}

\textbf{Principais Resultados}: 98.7\% retencao de acuracia, latencia 42$\times$ menor, compressao 15,333$\times$. SHAP correlation: $\rho = 0.94$ (quase perfeita), FAS=0.89, Top-3 overlap=93\%. Features criticas preservadas (``capital-gain'', ``education-num'', ``age'').

\subsection{Ablation Study: Impacto de $\beta$ (Peso XAI)}

Variamos $\beta$ (peso de $L_{XAI}$) em [0, 0.1, 0.2, 0.3, 0.4, 0.5] fixando $\alpha=0.5$.

\begin{table}[h]
\centering
\caption{Ablation: Impacto de $\beta$ (NLP Financial Phrasebank)}
\small
\begin{tabular}{lccc}
\toprule
\textbf{$\beta$} & \textbf{Acuracia (\%)} & \textbf{SHAP Corr. ($\rho$)} & \textbf{FAS} \\
\midrule
0.0 (KD puro) & 83.1 & 0.58 & 0.71 \\
0.1 & 83.6 & 0.72 & 0.78 \\
0.2 & 84.1 & 0.84 & 0.83 \\
0.3 (default) & \textbf{84.3} & \textbf{0.92} & \textbf{0.87} \\
0.4 & 84.0 & 0.94 & 0.89 \\
0.5 & 83.2 & 0.95 & 0.91 \\
\bottomrule
\end{tabular}
\end{table}

\textbf{Observacoes}:
\begin{itemize}
    \item \textbf{$\beta = 0$}: KD tradicional---alta acuracia, baixa correlacao SHAP
    \item \textbf{$\beta \in [0.2, 0.4]$}: Sweet spot---acuracia e explicabilidade balanceadas
    \item \textbf{$\beta > 0.4$}: SHAP correlation aumenta, mas acuracia degrada (student overfits explicacoes)
\end{itemize}

\textbf{Recomendacao}: $\beta = 0.3$ como default.

\section{Discussao}

\subsection{Principais Descobertas}

\subsubsection{Reducao de Complexidade via Encapsulamento}

DBDataset demonstra que \textbf{encapsulamento disciplinado} de elementos de validacao reduz drasticamente complexidade de codigo (75.7\% em media). Esta reducao nao e apenas quantitativa---elimina classes inteiras de erros:

\begin{itemize}
    \item \textbf{Mismatches de features}: Passar features categoricas para algoritmos que esperam numericas
    \item \textbf{Inconsistencias de split}: Usar random\_state diferente em diferentes etapas
    \item \textbf{Esquecimento de features}: Omitir features ao configurar validation suites
    \item \textbf{Erros de indexacao}: Confundir indices de train/test em analises
\end{itemize}

User study confirma: reducao de 85.7\% em erros de configuracao.

\subsubsection{Inferencia Automatica com 100\% de Acuracia}

Algoritmo de inferencia baseado em tipo + cardinalidade alcanca 100\% de acuracia em 387 features testadas. Fatores criticos:

\begin{enumerate}
    \item \textbf{Heuristica de dtype}: Features \texttt{object}/\texttt{category} sao inequivocamente categoricas em contexto tabular
    \item \textbf{Cardinalidade como fallback}: Permite capturar categoricas codificadas como inteiros (e.g., dias da semana como 0-6)
    \item \textbf{Override manual}: Escape hatch para casos ambiguos (IDs, ZIP codes)
\end{enumerate}

Casos onde inferencia falha: features ordinais codificadas como inteiros (e.g., \texttt{education\_level} = 1, 2, 3). Solucao: override manual ou \texttt{max\_categories}.

\subsubsection{Trade-off Memoria vs. Corretude}

DBDataset copia dados (2x memoria) para garantir imutabilidade. Em workflow de validacao offline, este trade-off e justificado:

\begin{itemize}
    \item \textbf{Validacao e processo batch}: Memoria disponivel, tempo de execucao nao-critico
    \item \textbf{Bugs de mutacao sao sutis}: Modificar DataFrame original pode causar erros dificeis de debugar
    \item \textbf{Reproducibilidade requer imutabilidade}: Copias garantem que re-execucao produz mesmos resultados
\end{itemize}

Para datasets gigantes (>10GB), DBDataset poderia oferecer modo \texttt{copy=False} (caveat emptor).

\subsection{Implicacoes Praticas}

\subsubsection{Para Praticantes de ML}

\paragraph{Reducao de Boilerplate} DBDataset elimina codigo repetitivo de preparacao de dados, permitindo foco em analise de resultados.

\paragraph{Onboarding Facilitado} Novos membros de equipe aprendem interface unica, nao multiplas convencoes de diferentes suites.

\paragraph{Menos Debugging} Validacao centralizada previne erros de configuracao que consomem horas de debugging.

\subsubsection{Para MLOps}

\paragraph{Integracao CI/CD Simplificada} Container unificado facilita passagem de dados entre stages de pipeline:

\begin{lstlisting}[language=Python, basicstyle=\ttfamily\scriptsize]
# Stage 1: Preparacao
dataset = DBDataset(data=df, target_column='y', model=model)
dataset.save('dataset.pkl')

# Stage 2: Validacao (processo separado)
dataset = DBDataset.load('dataset.pkl')
results = RobustnessSuite(dataset).run()
\end{lstlisting}

\paragraph{Reproducibilidade em Producao} Random states encapsulados garantem que validacao em desenvolvimento corresponde a validacao em staging/producao.

\subsubsection{Para Pesquisadores}

\paragraph{Comparacao de Abordagens} Interface padronizada permite comparar diferentes validation suites sem reescrever codigo de preparacao.

\paragraph{Extensao de Validation Suites} Novos metodos de validacao podem assumir DBDataset como input, reduzindo barreira de entrada.

\subsection{Limitacoes}

\subsubsection{Limitacao 1: Overhead de Memoria}

\textbf{Descricao}: Copias de dados consomem 2x memoria.

\textbf{Impacto}: Datasets >10GB podem exceder memoria disponivel.

\textbf{Mitigacao}: Implementar modo \texttt{copy=False} com warnings explicitos, ou usar Dask/Vaex para datasets out-of-core.

\subsubsection{Limitacao 2: Inferencia de Ordinais}

\textbf{Descricao}: Features ordinais codificadas como inteiros podem ser incorretamente classificadas como numericas.

\textbf{Exemplo}: \texttt{education\_level} = 1 (primario), 2 (secundario), 3 (superior).

\textbf{Impacto}: Algoritmos podem tratar ordinal como continuo (assumindo que 2 esta "entre" 1 e 3 numericamente).

\textbf{Mitigacao}: (1) Override manual via \texttt{categorical\_features}, (2) Adicionar parametro \texttt{ordinal\_features} em versoes futuras.

\subsubsection{Limitacao 3: Dados Nao-Tabulares}

\textbf{Descricao}: DBDataset otimizado para dados tabulares (CSV, DataFrames).

\textbf{Impacto}: Nao suporta nativamente imagens, texto, grafos, series temporais.

\textbf{Justificativa}: Validation suites do DeepBridge focam em modelos tabulares. Para outros dominios, abstraccoes diferentes sao mais apropriadas (e.g., \texttt{TorchVision.datasets} para imagens).

\subsubsection{Limitacao 4: Acoplamento com pandas}

\textbf{Descricao}: DBDataset usa pandas DataFrames internamente.

\textbf{Impacto}: Performance subotima para datasets gigantes comparado a Polars, Dask, Vaex.

\textbf{Mitigacao}: Futuras versoes podem suportar backends alternativos via protocolo (e.g., \texttt{\_\_dataframe\_\_}).

\subsection{Generalizabilidade}

\subsubsection{Aplicabilidade a Outros Dominios}

Container pattern de DBDataset pode ser adaptado para:

\begin{itemize}
    \item \textbf{NLP}: Encapsular texto, embeddings, labels, modelos de linguagem
    \item \textbf{Computer Vision}: Encapsular imagens, bounding boxes, segmentations, modelos
    \item \textbf{Time Series}: Encapsular series, lags, exogenous variables, forecasters
    \item \textbf{Grafos}: Encapsular nodes, edges, features, GNNs
\end{itemize}

Principios transferiveis: encapsulamento, inferencia automatica, integracao com validation tools.

\subsubsection{Extensoes para Casos de Uso Especializados}

DBDataset pode ser extendido para contextos especificos:

\paragraph{Federated Learning} Adicionar metodos para particionar dados por clientes:

\begin{lstlisting}[language=Python, basicstyle=\ttfamily\scriptsize]
datasets_by_client = dataset.partition_by('client_id', n_clients=10)
\end{lstlisting}

\paragraph{Active Learning} Suportar marcacao incremental de amostras:

\begin{lstlisting}[language=Python, basicstyle=\ttfamily\scriptsize]
unlabeled_dataset = dataset.get_unlabeled()
newly_labeled = oracle.label(unlabeled_dataset.sample(100))
dataset.add_labels(newly_labeled)
\end{lstlisting}

\paragraph{Multi-task Learning} Encapsular multiplos targets:

\begin{lstlisting}[language=Python, basicstyle=\ttfamily\scriptsize]
dataset = DBDataset(
    data=df,
    target_columns=['task1', 'task2', 'task3']  # Multi-target
)
\end{lstlisting}

\subsection{Relacao com Trabalhos Futuros}

\subsubsection{Integracao com MLflow}

DBDataset poderia ser logado como artifact no MLflow:

\begin{lstlisting}[language=Python, basicstyle=\ttfamily\scriptsize]
import mlflow

with mlflow.start_run():
    mlflow.log_artifact(dataset.save('dataset.pkl'))
    mlflow.log_params(dataset.get_metadata())  # Random state, split ratio
\end{lstlisting}

\subsubsection{Suporte a Data Versioning (DVC)}

Integracao com DVC para versionamento de datasets:

\begin{lstlisting}[language=Python, basicstyle=\ttfamily\scriptsize]
dataset.save_with_dvc('dataset.pkl')  # Auto-adiciona ao .dvc
\end{lstlisting}

\subsubsection{Schema Validation}

Adicionar validacao de schema para garantir consistencia:

\begin{lstlisting}[language=Python, basicstyle=\ttfamily\scriptsize]
schema = DatasetSchema(
    features={'age': int, 'income': float, 'gender': str},
    target='approved',
    constraints={'age': lambda x: x >= 0}
)

dataset = DBDataset(data=df, schema=schema)  # Valida na criacao
\end{lstlisting}

\subsection{Licoes Aprendidas}

\subsubsection{Design Iterativo}

DBDataset evoluiu atraves de 5+ iteracoes com feedback de usuarios:

\begin{enumerate}
    \item \textbf{v1}: Container simples sem inferencia (usuarios reclamaram de configuracao manual)
    \item \textbf{v2}: Inferencia baseada apenas em dtype (falhou em IDs numericos)
    \item \textbf{v3}: Adicao de cardinalidade + override manual (balance ideal)
    \item \textbf{v4}: Suporte a Bunch e modelos pre-treinados (requisito de usuarios)
    \item \textbf{v5}: Factory methods para workflows especializados (feedback de MLOps)
\end{enumerate}

\subsubsection{Importancia de Defaults Sensatos}

Parametros default (test\_size=0.2, stratify=False) escolhidos baseados em survey de 50+ projetos ML open-source. Defaults ruins aumentam friccao de adocao.

\subsubsection{Documentacao e Exemplos}

User study revelou que exemplos concretos (case studies) foram mais efetivos que documentacao de API para onboarding. Investir em tutoriais praticos e essencial.

\subsection{Consideracoes Eticas}

\subsubsection{Facilitacao de Validacao de Fairness}

DBDataset reduz barreira tecnica para executar testes de fairness, potencialmente aumentando adocao de validacao de bias em sistemas ML. Impacto social positivo: modelos mais justos em producao.

\subsubsection{Risco de Over-reliance em Automacao}

Inferencia automatica pode criar falsa sensacao de seguranca---usuarios podem nao validar se features categoricas foram corretamente identificadas. Mitigacao: logs informativos e metodos de inspecao (\texttt{dataset.inspect\_features()}).

\subsection{Recomendacoes para Adocao}

\subsubsection{Para Equipes Iniciando Validacao}

\begin{enumerate}
    \item Iniciar com workflow simples (unified data + auto-split)
    \item Validar inferencia de features manualmente em primeiros usos
    \item Integrar gradualmente em pipeline CI/CD
\end{enumerate}

\subsubsection{Para Equipes com Pipelines Existentes}

\begin{enumerate}
    \item Criar adapters para converter codigo existente para DBDataset
    \item Executar validacao paralela (pipeline antigo + DBDataset) durante transicao
    \item Migrar validation suite por vez (comecando com mais simples)
\end{enumerate}

\subsubsection{Para Organizacoes Enterprise}

\begin{enumerate}
    \item Adicionar DBDataset a template de projetos ML
    \item Treinar equipes em workshop hands-on (2-4 horas)
    \item Estabelecer DBDataset como padrao em code review guidelines
\end{enumerate}

\section{Conclusao}

\subsection{Sintese de Contribuicoes}

Apresentamos \textbf{DBDataset}, um container de dados unificado que simplifica validacao de modelos ML atraves de encapsulamento disciplinado e inferencia automatica de features. Nossas principais contribuicoes:

\begin{enumerate}
    \item \textbf{Container Pattern}: Primeira solucao que unifica dados, features, modelos, e predicoes em interface coesa para validacao
    \item \textbf{Inferencia Automatica}: Algoritmo baseado em tipo + cardinalidade com 100\% de acuracia em 387 features testadas
    \item \textbf{Flexibilidade de Workflows}: Suporte a 4 modos de inicializacao cobrindo casos de uso desde prototipagem ate producao
    \item \textbf{Integracao Seamless}: Interface padronizada para 6 validation suites (robustness, uncertainty, fairness, resilience, hyperparameter, distillation)
    \item \textbf{Validacao Empirica}: Case studies demonstrando reducao de 75.7\% em codigo e 85.7\% em erros de configuracao
\end{enumerate}

\subsection{Impacto Esperado}

\subsubsection{Comunidade de Praticantes}

DBDataset reduz barreiras tecnicas para validacao rigorosa de modelos ML. Reducao de 62.8\% em tempo de setup (user study) permite que equipes adotem validacao abrangente sem overhead proibitivo.

\textbf{Projecao de impacto}: Se 10\% de projetos ML em producao adotarem validacao rigorosa devido a DBDataset, estimamos prevenção de centenas de falhas de modelos em dominios criticos (saude, financas, contratacao).

\subsubsection{Pesquisa Academica}

Interface padronizada facilita comparacao entre metodos de validacao. Pesquisadores podem publicar novos testes de robustness/fairness assumindo DBDataset como input, acelerando inovacao em ML trustworthy.

\subsubsection{Industria e MLOps}

Container unificado simplifica integracao de validacao em pipelines CI/CD. Organizacoes podem estabelecer DBDataset como padrao interno, reduzindo heterogeneidade de codigo e facilitando onboarding.

\subsection{Trabalhos Futuros}

\subsubsection{Curto Prazo (6-12 meses)}

\paragraph{Suporte a Dados Ordinais} Adicionar parametro \texttt{ordinal\_features} com especificacao de ordem:

\begin{lstlisting}[language=Python, basicstyle=\ttfamily\scriptsize]
dataset = DBDataset(
    data=df,
    target_column='y',
    ordinal_features={
        'education': ['primary', 'secondary', 'higher'],
        'satisfaction': [1, 2, 3, 4, 5]
    }
)
\end{lstlisting}

\paragraph{Modo Copy-on-Write} Reduzir overhead de memoria para datasets gigantes:

\begin{lstlisting}[language=Python, basicstyle=\ttfamily\scriptsize]
dataset = DBDataset(data=df, target_column='y', copy=False)
# Warning: Modifications to df will affect dataset
\end{lstlisting}

\paragraph{Schema Validation} Integracao com Pydantic ou Pandera para validacao de tipos e constraints:

\begin{lstlisting}[language=Python, basicstyle=\ttfamily\scriptsize]
from deepbridge.schemas import DatasetSchema

schema = DatasetSchema.from_yaml('schema.yaml')
dataset = DBDataset(data=df, schema=schema)
\end{lstlisting}

\subsubsection{Medio Prazo (1-2 anos)}

\paragraph{Backends Alternativos} Suporte a Polars, Dask, Vaex para datasets out-of-core:

\begin{lstlisting}[language=Python, basicstyle=\ttfamily\scriptsize]
dataset = DBDataset(
    data=dask_df,
    target_column='y',
    backend='dask'  # Auto-detecta ou especificado
)
\end{lstlisting}

\paragraph{Feature Stores Integration} Integracao com Feast, Tecton para carregar features de producao:

\begin{lstlisting}[language=Python, basicstyle=\ttfamily\scriptsize]
from deepbridge.integrations import FeastConnector

connector = FeastConnector(feature_store_url='...')
dataset = connector.create_dataset(
    entity_df=entities,
    features=['feature1', 'feature2'],
    target_column='y'
)
\end{lstlisting}

\paragraph{Time Series Support} Extensao para dados temporais com lags automaticos:

\begin{lstlisting}[language=Python, basicstyle=\ttfamily\scriptsize]
from deepbridge import TimeSeriesDataset

ts_dataset = TimeSeriesDataset(
    data=df,
    target_column='sales',
    datetime_column='date',
    lags=[1, 7, 30],  # Auto-gera features de lag
    rolling_windows=[7, 30]  # Auto-gera rolling means
)
\end{lstlisting}

\subsubsection{Longo Prazo (2+ anos)}

\paragraph{Multi-modal Datasets} Suporte a combinacao de tabular + imagens + texto:

\begin{lstlisting}[language=Python, basicstyle=\ttfamily\scriptsize]
from deepbridge import MultiModalDataset

mm_dataset = MultiModalDataset(
    tabular_data=df,
    image_column='product_image',  # Paths para imagens
    text_column='description',
    target_column='category'
)
\end{lstlisting}

\paragraph{AutoML Integration} DBDataset como input nativo para frameworks AutoML:

\begin{lstlisting}[language=Python, basicstyle=\ttfamily\scriptsize]
from autosklearn import AutoSklearnClassifier

automl = AutoSklearnClassifier()
automl.fit(dataset)  # Aceita DBDataset diretamente
\end{lstlisting}

\paragraph{Differential Privacy} Suporte a private data splits:

\begin{lstlisting}[language=Python, basicstyle=\ttfamily\scriptsize]
dataset = DBDataset(
    data=df,
    target_column='y',
    privacy_budget=1.0,  # Epsilon para DP
    add_noise=True
)
\end{lstlisting}

\subsection{Chamada para Comunidade}

DBDataset e open-source (licenca MIT) e desenvolvido publicamente:

\begin{itemize}
    \item \textbf{Codigo}: \texttt{github.com/deepbridge/deepbridge}
    \item \textbf{Documentacao}: \texttt{deepbridge.readthedocs.io}
    \item \textbf{Issues}: \texttt{github.com/deepbridge/deepbridge/issues}
\end{itemize}

Convidamos comunidade ML para:

\begin{enumerate}
    \item \textbf{Contribuir}: Adicionar novos modos de inicializacao, backends, integrações
    \item \textbf{Reportar bugs}: Casos onde inferencia falha ou design e inadequado
    \item \textbf{Propor extensoes}: Features para casos de uso nao cobertos
    \item \textbf{Compartilhar experiencias}: Case studies em dominios nao testados
\end{enumerate}

\subsection{Mensagem Final}

Validacao rigorosa de modelos ML nao deve ser privilégio de equipes com recursos abundantes. DBDataset democratiza validacao ao reduzir complexidade tecnica e overhead de configuracao. Nossa visao: fazer validacao abrangente (robustness, uncertainty, fairness) tao trivial quanto treinar modelo com \texttt{model.fit()}.

Fragmentacao de gestao de dados em validacao ML e problema solucionavel. Container pattern com inferencia automatica demonstra que \textbf{simplicidade e rigor nao sao mutuamente exclusivos}---ambos podem ser alcançados atraves de design cuidadoso e encapsulamento disciplinado.

DBDataset e passo inicial em direcao a ecosistema ML onde validacao e parte natural do workflow, nao tarefa opcional relegada a pos-deployment. Acreditamos que futuro de ML responsavel depende de ferramentas que tornem praticas corretas mais faceis que praticas inadequadas.

\subsection{Disponibilidade}

\begin{itemize}
    \item \textbf{Codigo-fonte}: MIT License, disponivel em \texttt{github.com/deepbridge/deepbridge}
    \item \textbf{Datasets}: Case studies reproducibles em \texttt{github.com/deepbridge/dbdataset-paper}
    \item \textbf{Artefatos}: Modelos treinados, resultados experimentais em Zenodo (DOI: [a definir])
    \item \textbf{Documentacao}: Tutoriais e exemplos em \texttt{deepbridge.readthedocs.io}
\end{itemize}

\subsection{Agradecimentos}

Agradecemos aos 15 participantes do user study por feedback valioso, aos revisores anonimos por sugestoes construtivas, e a comunidade open-source Python (pandas, scikit-learn, NumPy) cujas ferramentas fundamentam DBDataset.

Financiamento: [A definir]

\subsection{Consideracoes Finais}

DBDataset representa mudanca de paradigma em como dados sao gerenciados para validacao de modelos ML---de objetos fragmentados para container unificado, de configuracao manual para inferencia automatica, de codigo ad-hoc para interface padronizada. Esperamos que este trabalho inspire desenvolvimento de ferramentas similares em outros dominios ML e contribua para ecosistema mais maduro de validacao de modelos.

\textit{Machine Learning e muito mais que treinar modelos---e validar rigorosamente que eles funcionam como esperado. DBDataset torna esta validacao simples, reproduzivel, e acessivel.}


% Bibliografia
\bibliographystyle{plain}
\bibliography{bibliography/references}

\end{document}

\section{Trabalhos Relacionados}
\label{sec:background}

Organizamos trabalhos relacionados em três categorias: slice-based analysis, error analysis e model debugging.

\subsection{Slice-Based Analysis}

\textbf{Slice Finder}~\cite{chung2019slice}: Google desenvolveu técnica para encontrar subgrupos com performance degradada usando árvores de decisão. Limitação: foca apenas em tree-based slicing.

\textbf{Spotlight}~\cite{lakkaraju2017identifying}: Microsoft propôs método para identificar regiões de erro usando clustering. Limitação: requer features pré-selecionadas.

\textbf{Slicing for Fairness}~\cite{chen2019slicing}: Análise de slices para detectar bias em grupos protegidos. Limitação: restrito a atributos protegidos conhecidos.

\textbf{Diferencial}: Nossa abordagem combina múltiplas estratégias (quantile + uniform + tree), não requer pré-seleção de features e detecta interações.

\subsection{Error Analysis}

\textbf{Error Pattern Detection}~\cite{sipple2020interpretable}: Identifica padrões de erro via clustering. Limitação: não fornece ranges específicos.

\textbf{Subgroup Discovery}~\cite{lemmerich2016fast}: Mineração de regras para subgrupos anômalos. Limitação: exponencial em número de features.

\textbf{Data Quality Issues}~\cite{schelter2018automating}: Detecta problemas de qualidade em slices. Limitação: foca em integridade de dados, não performance.

\textbf{Diferencial}: Focamos especificamente em degradação de performance com classificação de severidade.

\subsection{Model Debugging}

\textbf{Influence Functions}~\cite{koh2017understanding}: Identifica amostras influentes. Limitação: não agrupa em regiões.

\textbf{Anchors}~\cite{ribeiro2018anchors}: Regras locais de predição. Limitação: explainability individual, não análise de subgrupos.

\textbf{Testing Tools}:
\begin{itemize}
    \item \textbf{Checklist}~\cite{ribeiro2020beyond}: Templates manuais para NLP
    \item \textbf{Great Expectations}: Validação de dados, não modelos
    \item \textbf{Deepchecks}: Foca em drift, não weakspots locais
\end{itemize}

\textbf{Diferencial}: Detecção automática e sistemática de regiões de degradação.

\subsection{Comparação com Ferramentas Existentes}

\begin{table}[h]
\centering
\caption{Comparação de abordagens para detecção de degradação}
\label{tab:related_comparison}
\small
\begin{tabular}{@{}lcccc@{}}
\toprule
\textbf{Abordagem} & \textbf{Multi-} & \textbf{Severidade} & \textbf{Interações} & \textbf{Auto-} \\
 & \textbf{Estratégia} & \textbf{Auto.} &  & \textbf{mático} \\
\midrule
Slice Finder & \xmark & \xmark & \xmark & \cmark \\
Spotlight & \xmark & \xmark & \xmark & \textasciitilde \\
Subgroup Disc. & \xmark & \xmark & \cmark & \cmark \\
Manual Analysis & \cmark & \cmark & \xmark & \xmark \\
\midrule
\textbf{Este trabalho} & \cmark & \cmark & \cmark & \cmark \\
\bottomrule
\end{tabular}
\end{table}

\subsection{Posicionamento}

Nosso trabalho \textbf{complementa} ferramentas de fairness e robustness:
\begin{itemize}
    \item \textbf{Fairness tools} (AIF360, Fairlearn): Focam em grupos protegidos conhecidos
    \item \textbf{Robustness tools} (Foolbox, ART): Testam perturbações adversariais
    \item \textbf{Weakspot detector}: Descobre regiões desconhecidas de degradação
\end{itemize}

\textbf{Integração}: Weakspot detection é uma das 5 dimensões do DeepBridge (Paper 3), mas pode ser usado standalone.

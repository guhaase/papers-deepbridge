\section{Conclusão}
\label{sec:conclusion}

\subsection{Sumário de Contribuições}

Apresentamos um framework sistemático para \textbf{detecção automática de weakspots} em modelos de Machine Learning—regiões do espaço de features onde performance degrada significativamente. Nossa abordagem combina três estratégias complementares de slicing (quantile-based, uniform, tree-based) com classificação de severidade e análise de interações.

\textbf{Contribuições Principais}:

\textbf{1. Framework Multi-Estratégia} (Seção~\ref{sec:framework}):
\begin{itemize}
    \item \textbf{Quantile slicing}: Robusto a distribuições skewed
    \item \textbf{Uniform slicing}: Interpretável para features com domínio conhecido
    \item \textbf{Tree-based slicing}: Data-driven, descobre boundaries ótimos
    \item \textbf{Interaction detection}: Weakspots multi-dimensionais
\end{itemize}

\textbf{2. Classificação de Severidade} (Seção~\ref{sec:strategies}):
\begin{itemize}
    \item Thresholds configuráveis (Low/Medium/High/Critical)
    \item Testes de significância estatística
    \item Requisitos de tamanho mínimo de amostra
    \item Filtros anti-overfitting
\end{itemize}

\textbf{3. Avaliação Abrangente} (Seção~\ref{sec:evaluation}):
\begin{itemize}
    \item Datasets sintéticos: 94\% F1 na detecção
    \item 8 datasets reais: 127 weakspots detectados
    \item 3 estudos de caso (credit, healthcare, fraud)
    \item Comparação com baselines: 2.8x cobertura, 80x speedup
\end{itemize}

\textbf{4. Integração com DeepBridge}:
\begin{itemize}
    \item Parte do pipeline completo de validação
    \item Disponível como ferramenta standalone
    \item API consistente tipo scikit-learn
\end{itemize}

\subsection{Resultados Principais}

\textbf{Eficácia de Detecção}:
\begin{itemize}
    \item \textbf{94\% F1 score} em datasets sintéticos (precision=0.94, recall=0.94)
    \item \textbf{127 weakspots únicos} em 8 datasets reais
    \item \textbf{Degradações de até 35pp} em subgrupos específicos
    \item \textbf{6\% false discovery rate} (vs. 12-22\% em baselines)
\end{itemize}

\textbf{Eficiência Computacional}:
\begin{itemize}
    \item \textbf{80x speedup}: 3 min vs. 4 horas (análise manual)
    \item \textbf{2.8x cobertura}: 127 vs. 45 weakspots (manual)
    \item \textbf{Escalável}: 0.5-8 min para datasets de 300 a 500K amostras
\end{itemize}

\textbf{Insights de Produção}:
\begin{itemize}
    \item \textbf{Credit scoring}: Degradação de 28pp em jovens de baixa renda
    \item \textbf{Healthcare}: AUC 21pp menor para jovens obesos (bias interseccional)
    \item \textbf{Fraud detection}: F1 30pp menor em transações grandes de madrugada
\end{itemize}

\subsection{Complementaridade das Estratégias}

Ablation study (Seção~\ref{sec:evaluation}) demonstra valor de cada componente:

\begin{itemize}
    \item \textbf{Multi-estratégia}: +39\% cobertura vs. quantile-only
    \item \textbf{Tree-based}: +19\% weakspots (descobre boundaries não-óbvios)
    \item \textbf{Interactions}: +26\% weakspots (multi-dimensional críticos)
    \item \textbf{Significance tests}: Reduz FDR de 18\% para 6\%
\end{itemize}

\textbf{Conclusão}: Estratégias são \textit{complementares}, não redundantes.

\subsection{Impact em Validação de Modelos}

DeepBridge weakspot detector preenche lacuna crítica:

\textbf{Gap Existente}:
\begin{itemize}
    \item Fairness tools: Apenas grupos protegidos conhecidos
    \item Robustness tools: Perturbações aleatórias, não regiões específicas
    \item Explainability tools: Instâncias individuais, não subgrupos
\end{itemize}

\textbf{Nossa Contribuição}:
\begin{itemize}
    \item Descoberta exploratória de \textit{qualquer} região de degradação
    \item Automática, sistemática, escalável
    \item Integrada ao pipeline completo de validação
\end{itemize}

\subsection{Lições Aprendidas}

\textbf{1. Múltiplas Estratégias São Necessárias}:

Single-strategy (e.g., apenas quantile) perde 39\% dos weakspots. Tree-based descobre interações que quantile/uniform não capturam.

\textbf{2. Significance Testing É Essencial}:

Sem testes estatísticos, FDR sobe para 18\% (muitos falsos positivos). Significance tests são críticos para confiabilidade.

\textbf{3. Interações São Comuns}:

26\% dos weakspots são multi-dimensionais (2-way interactions). Ignorar interações perde problemas críticos.

\textbf{4. Tamanho Mínimo Importa}:

Weakspots com < 30 amostras são geralmente noise. Threshold de tamanho mínimo é essencial para robustez.

\textbf{5. Remediation Varia por Contexto}:

Não há "one-size-fits-all". Data augmentation funciona para alguns, ensembles para outros, regras de negócio para terceiros.

\subsection{Trabalhos Futuros}

\textbf{1. Automated Remediation}:
\begin{itemize}
    \item Auto-generate synthetic data para weakspots
    \item Auto-suggest feature engineering
    \item Auto-create ensemble models
\end{itemize}

\textbf{2. Deep Learning Support}:
\begin{itemize}
    \item Slicing em embedding spaces
    \item Activation-based slicing
    \item Concept-based slicing (TCAV integration)
\end{itemize}

\textbf{3. Temporal Tracking}:
\begin{itemize}
    \item Monitor weakspots em produção
    \item Detect quando novos weakspots emergem (drift)
    \item Dashboards históricos
\end{itemize}

\textbf{4. Causal Analysis}:
\begin{itemize}
    \item Por que este weakspot existe?
    \item Root cause: data bias vs. model bias
    \item Counterfactual explanations
\end{itemize}

\textbf{5. Multi-Model Comparison}:
\begin{itemize}
    \item Compare weakspots entre candidatos
    \item Selecione modelo com menos weakspots críticos
    \item Pareto frontier: accuracy vs. weakspot severity
\end{itemize}

\subsection{Broader Impact}

\textbf{Impacto Positivo}:
\begin{itemize}
    \item \textbf{Segurança}: Detectar problemas antes de produção
    \item \textbf{Fairness}: Descobrir bias oculto em subgrupos
    \item \textbf{Confiabilidade}: Modelos mais robustos em todas regiões
    \item \textbf{Democratização}: Ferramenta acessível (open-source)
\end{itemize}

\textbf{Riscos e Mitigações}:
\begin{itemize}
    \item \textbf{Risco}: Over-reliance em automação, ignorar domain expertise
    \item \textbf{Mitigação}: Documentação enfatiza validação manual de weakspots detectados

    \item \textbf{Risco}: Falsos positivos causam trabalho desnecessário
    \item \textbf{Mitigação}: Significance testing + ranking por severidade

    \item \textbf{Risco}: Weaponization (identificar subgrupos para discriminar)
    \item \textbf{Mitigação}: Uso em contexto de validação, não seleção adversarial
\end{itemize}

\subsection{Conclusão Final}

Weakspot detection é componente crítico de validação de modelos ML. Nossa abordagem multi-estratégia demonstra que é possível \textbf{detectar automaticamente regiões de degradação} com alta precisão (94\% F1), cobertura abrangente (2.8x vs. manual) e eficiência (80x speedup).

Através de 127 weakspots detectados em datasets reais e estudos de caso em aplicações críticas, demonstramos que performance global esconde problemas locais severos—e que detecção sistemática é viável e necessária.

Nossa esperança é que ao tornar weakspot detection acessível e integrado, DeepBridge contribua para modelos ML mais seguros, justos e confiáveis em produção.

\subsection{Availability}

\textbf{Code}: \url{https://github.com/DeepBridge-Validation/DeepBridge}

\textbf{Documentation}: \url{https://deepbridge.readthedocs.io/weakspots}

\textbf{Tutorials}: \url{https://deepbridge.readthedocs.io/tutorials/weakspot-detection}

\textbf{Datasets}: \url{https://github.com/DeepBridge-Validation/weakspot-benchmarks}

\textbf{License}: MIT (open-source)

\textbf{PyPI}: \texttt{pip install deepbridge}

\section{Discussão}
\label{sec:discussion}

\subsection{Quando Usar Weakspot Detection}

\textbf{Casos Ideais}:
\begin{itemize}
    \item \textbf{Modelos em produção}: Detectar problemas antes de deployment
    \item \textbf{Aplicações de alto risco}: Finanças, saúde, justiça (consequências severas)
    \item \textbf{Após model training}: Como parte de pipeline de validação
    \item \textbf{Debugging de performance}: Entender onde modelo falha
    \item \textbf{Compliance}: Demonstrar que não há bias oculto
\end{itemize}

\textbf{Exemplos de Uso}:
\begin{enumerate}
    \item \textbf{Pre-deployment check}: "Há algum subgrupo com degradação > 15\%?"
    \item \textbf{Model comparison}: "Modelo B tem menos weakspots que Modelo A?"
    \item \textbf{Data augmentation}: "Quais regiões precisam de mais dados?"
    \item \textbf{Feature engineering}: "Interações de features causam problemas?"
\end{enumerate}

\subsection{Interpretação de Resultados}

\textbf{Severidade}:
\begin{itemize}
    \item \textbf{Critical}: Ação imediata (re-treinar, não deploy)
    \item \textbf{High}: Mitigação recomendada (data augmentation, ensemble)
    \item \textbf{Medium}: Monitorar em produção
    \item \textbf{Low}: Documentar, aceitar trade-off
\end{itemize}

\textbf{Tamanho de Amostra}:
\begin{itemize}
    \item Weakspot com 1000+ amostras: Problema real e impactante
    \item Weakspot com 30-100 amostras: Validar se é noise ou padrão real
    \item Weakspot com < 30 amostras: Geralmente filtrado (pode ser outlier)
\end{itemize}

\textbf{Interações}:
\begin{itemize}
    \item Interações 2-way: Comuns e interpretáveis
    \item Interações 3-way+: Raras, difíceis de remediar
    \item Priorizar: Interações com maior degradação + tamanho
\end{itemize}

\subsection{Estratégias de Remediação}

Após detectar weakspots, como corrigir?

\textbf{1. Data Augmentation}:
\begin{itemize}
    \item Coletar mais dados para região de degradação
    \item Synthetic data generation (SMOTE, GANs)
    \item Oversampling do subgrupo
\end{itemize}

\textbf{Exemplo}: Weakspot em age < 25 $\rightarrow$ Coletar 2x mais amostras deste grupo.

\textbf{2. Feature Engineering}:
\begin{itemize}
    \item Criar features específicas para weakspot
    \item Transformações não-lineares
    \item Interaction features
\end{itemize}

\textbf{Exemplo}: Weakspot em (age < 25, duration > 24) $\rightarrow$ Criar feature \texttt{risk\_score = duration / age}.

\textbf{3. Model Refinement}:
\begin{itemize}
    \item Re-treinar com sample weights (upweight weakspot)
    \item Ensemble: Modelo geral + modelo específico para weakspot
    \item Threshold tuning para subgrupo
\end{itemize}

\textbf{Exemplo}: Train model especializado para age < 25, combine via stacking.

\textbf{4. Domain Constraints}:
\begin{itemize}
    \item Regras de negócio para weakspot
    \item Human-in-the-loop para casos críticos
    \item Confidence thresholds mais altos
\end{itemize}

\textbf{Exemplo}: Fraudes > \$10K (weakspot) $\rightarrow$ Require manual review.

\textbf{5. Monitoramento Contínuo}:
\begin{itemize}
    \item Tracking de performance em produção por slice
    \item Alerts quando weakspot degrada mais
    \item Periodic re-detection (drift pode criar novos weakspots)
\end{itemize}

\subsection{Limitações}

\textbf{1. Features Categóricas com Alta Cardinalidade}:

\textbf{Problema}: Features com 100+ categorias (e.g., merchant ID, ZIP code) geram muitos slices.

\textbf{Mitigação}:
\begin{itemize}
    \item Agrupar categorias similares (clustering)
    \item Limitar a top-K categorias mais frequentes
    \item Tree-based slicing (agrupa automaticamente)
\end{itemize}

\textbf{2. Datasets Pequenos}:

\textbf{Problema}: n < 1000 amostras $\rightarrow$ Slices muito pequenos, alta variância.

\textbf{Mitigação}:
\begin{itemize}
    \item Aumentar \texttt{min\_samples\_per\_slice} (e.g., 50)
    \item Usar apenas quantile slicing (mais robusto)
    \item Bootstrap para intervalos de confiança
\end{itemize}

\textbf{3. Interações de Alta Ordem}:

\textbf{Problema}: 3-way+ interactions são exponenciais e difíceis de interpretar.

\textbf{Decisão de Design}: Limitamos a 2-way interactions.

\textbf{Justificativa}: 90\%+ dos weakspots são single ou 2-way (análise empírica).

\textbf{4. Overfitting em Tree-Based Slicing}:

\textbf{Problema}: Decision tree pode criar splits espúrios.

\textbf{Mitigação}:
\begin{itemize}
    \item \texttt{min\_samples\_leaf} = 50
    \item \texttt{max\_depth} = 3
    \item Validação cruzada
    \item Significance testing
\end{itemize}

\textbf{5. Interpretação de Múltiplos Weakspots}:

\textbf{Problema}: 100+ weakspots detectados $\rightarrow$ Overwhelming.

\textbf{Mitigação}:
\begin{itemize}
    \item Ranking por severidade $\times$ tamanho
    \item Agrupamento de weakspots similares
    \item Foco em top-10 críticos
\end{itemize}

\subsection{Boas Práticas}

\textbf{1. Configure Thresholds por Domínio}:
\begin{itemize}
    \item Saúde/Justiça: Thresholds conservadores (Critical > 10\%)
    \item Marketing/Recomendação: Thresholds relaxados (Critical > 25\%)
\end{itemize}

\textbf{2. Valide Weakspots Manualmente}:
\begin{itemize}
    \item Spot-check top-10 weakspots
    \item Verifique se fazem sentido no domínio
    \item Consulte domain experts
\end{itemize}

\textbf{3. Use Estratégia Combinada}:
\begin{itemize}
    \item Quantile + Uniform + Tree (padrão)
    \item Complementaridade aumenta cobertura
    \item Overhead de tempo é marginal
\end{itemize}

\textbf{4. Priorize por Impact}:
\begin{itemize}
    \item Impact = Degradation $\times$ Sample Size
    \item Foque em weakspots com alto impact
    \item Documente decisões de aceitar low-severity weakspots
\end{itemize}

\textbf{5. Integre ao CI/CD}:
\begin{itemize}
    \item Detecção automática a cada re-training
    \item Bloqueie deployment se Critical weakspot
    \item Track histórico de weakspots (trends)
\end{itemize}

\subsection{Direções Futuras}

\textbf{1. Automated Remediation}:

\textbf{Proposta}: Não apenas detectar, mas sugerir fixes automaticamente.

\textbf{Técnicas}:
\begin{itemize}
    \item Auto-generate SMOTE samples para weakspot
    \item Auto-create ensemble com specialized model
    \item Auto-suggest feature engineering
\end{itemize}

\textbf{2. Deep Learning Support}:

\textbf{Desafio}: Features são embeddings (não interpretáveis).

\textbf{Abordagem}:
\begin{itemize}
    \item Slicing em embedding space (clustering)
    \item Activation-based slicing (layer outputs)
    \item Concept-based slicing (TCAV)
\end{itemize}

\textbf{3. Temporal Weakspot Tracking}:

\textbf{Proposta}: Detectar quando novos weakspots emergem (drift).

\textbf{Features}:
\begin{itemize}
    \item Periodic re-detection em produção
    \item Alerts quando weakspot degrada mais
    \item Dashboards históricos
\end{itemize}

\textbf{4. Multi-Model Comparison}:

\textbf{Proposta}: Compare weakspots entre modelos candidatos.

\textbf{Workflow}:
\begin{enumerate}
    \item Treinar N modelos (hyperparameter tuning)
    \item Detectar weakspots em cada
    \item Selecionar modelo com menos/menos severos weakspots
\end{enumerate}

\textbf{5. Causal Weakspot Analysis}:

\textbf{Questão}: Por que este weakspot existe?

\textbf{Abordagem}:
\begin{itemize}
    \item Causal inference (SCM)
    \item Counterfactual analysis
    \item Identificar root causes (data bias vs. model bias)
\end{itemize}

\subsection{Relação com Fairness}

Weakspot detection \textbf{complementa} análise de fairness:

\textbf{Fairness Tools}:
\begin{itemize}
    \item Focam em grupos protegidos conhecidos (raça, gênero, idade)
    \item Métricas específicas (disparate impact, equalized odds)
\end{itemize}

\textbf{Weakspot Detection}:
\begin{itemize}
    \item Descobre degradação em \textit{qualquer} subgrupo
    \item Pode incluir grupos não-protegidos (e.g., score range)
    \item Métricas gerais (accuracy, F1)
\end{itemize}

\textbf{Uso Conjunto}:
\begin{enumerate}
    \item Execute fairness tests (grupos conhecidos)
    \item Execute weakspot detection (descoberta exploratória)
    \item Priorize issues que aparecem em ambos
\end{enumerate}

\textbf{Exemplo}:
\begin{itemize}
    \item Fairness: Detecta disparate impact em gender
    \item Weakspot: Detecta degradação em (gender=F, age<30, income<\$40K)
    \item Insight: Problema é interseccional, não apenas gender
\end{itemize}
